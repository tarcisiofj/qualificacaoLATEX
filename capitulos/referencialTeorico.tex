\chapter{Referencial Teórico}\label{cap:refTeor}

\lipsum[34]

\section{Trabalhos Correlatos}\label{sec:primTrab}

Eu queria testar notas\footnote{Referente \cite{1}  } de rodapé\cite[Seilá]{3}. Ainda não sei como fazer.


\lipsum[34]
\section{Aprendizado de Máquina}\label{sec:aprendMaq}

Aprendizagem de máquina é a capacidade do aprendizado por utilização de algoritmos atuando em cima de uma base de dados.  Diz-se que o computador está aprendendo quando existe uma melhora de desempenho de tarefas que ele utilizou como exemplo \cite{Mitchell1997}. Pode-se citar como exemplo a realização do reconhecimento facial de uma pessoa utilizando aprendizado de máquina. Não seria necessário a criação de várias linhas de código informando a cor do olho azul, orelha grande e cabelo grande seriam de uma certa pessoa. Ao invés disso é observada várias fotos tituladas de uma certa pessoa, e após vários exemplos o computador seria capaz de predizer uma foto nova, se é, ou não, da determinada pessoa através de aprendizado anterior.

Existem alguns motivos, onde justificam, que não é possível simplismente exigir que o projetista implemente melhorias no sistema de forma que ele esteja robusto bastante para lidar com todas as situações (Russel, Stuart. Norvig, 2013). Um  desses motivos seria a incapacidade da antecipação de  todas as situações possíveis de implementação por parte do programador. Fazendo um resumo, aprendizado de máquina seriam algoritmos capazes de aprender automaticamente através de  determinados exemplos, ou comportamentos. 

A partir desta síntese, tem-se uma observação. Para exemplificar uma classificação de dados no contexto de aprendizado de máquina, dois pilares são observados. Um, seriam os dados a serem classificados, e outro o algoritmo que irá atuar nessa base de dados. Como exemplo desses algoritmos podem-se citar: redes neurais, árvores de decisão, Suport Vector Machine – SVM, etc. Qualquer um destes algoritmos poderam ser utilizados para solucionar essa classificação. E a escolha apropriada se dará através de métricas que avaliaram o desempenho de cada algoritmo, e a melhor métrica será a escolha do algoritmo apropriado para aquele problema de classificação de dados. 

(Uma analogia pode ser feita referente do que foi dito acima.) Através do parágrafo acima poderia ser feita uma analogia. Um “problema” qualquer seria um “motor”, e os algoritmos disponíveis seriam as ferramentas para concertar esse motor. A partir daí a ferramenta que fosse mais eficaz, considerando métricas de desempenho, para fazer o motor funcionar, seria a ferramenta(algoritmo) escolhida. Tendo assim a escolha certa para um determinado problema.

\subsection{Aprendizado Supervisionado}\label{ssec:aprendSup}

Nesta sessão será abordado Aqui os dados são classificados e a partir dessa classificação será feita uma predição através de algoritmos.



