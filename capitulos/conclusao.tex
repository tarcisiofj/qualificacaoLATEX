\chapter{Conclusões, Trabalhos Futuros}\label{cap:conclusao} 

Neste capítulo serão apresentadas as conclusões acerca deste trabalho, bem como a apresentação dos benefícios da pesquisa realizada, que teve como base a rotulação de algoritmos com vistas a analisar o seu comportamento considerando algumas bases de dados. Também serão apresentadas sugestões para continuidade deste trabalho já efetuado. 

\section{Conclusão}\label{cond}




Assim, diante do problema de rotulação, que visa encontrar características relevantes em grupos de dados ao ponto de identificar esses grupos, e também de outra pesquisa já concluída \cite{Lopes2016}, que trata deste problema, esta pesquisa teve como diferencial aplicação de novos algoritmos com paradigmas diferentes ainda não testados, , Naive Bayes (estatístico), Classification and Regression Trees - CART (árvore de decisão) e K-Nearest Neighbor - KNN (baseado em instância), apresentando a viabilidade do método de rotulação de dados com algoritmos de aprendizado supervisionados e comparando os resultados através das acurácias dos rótulos encontradas em relação as bases já classificadas (separadas por grupos).

Esta pesquisa tem como referência o trabalho de \citeonline{Lopes2016} utilizando algumas bases de dados iguais como também a ferramenta de programação, MATLAB. Para simular o ambiente e fazer testes comparativos com este trabalho, foi também empregue os mesmos  métodos de discretização e valores das faixas para poder compará-los, embora as respostas não tenham sido conclusivas visto que os grupos utilizados por \cite{Lopes2016} não poderem ser reproduzidos fielmente, e sem essa fidelidade, não há como compará-los de forma justa, tendo em vista que os resultados desta pesquisa são totalmente dependentes dos clusters utilizados.


As análises foram feitas a partir das bases de dados: Seeds, Iris, Glass e Wine, e o resultado ao aplicar cada algoritmo nas bases e encontrar os rótulos foram satisfatórios. A avaliação da qualidade destes rótulos foi feita da seguinte forma: inicialmente, escolheram-se as bases de dados já classificadas; feito isso, foi possível saber quantas amostras pertencem a determinado grupo; em seguida, foi realizado um comparativo dos resultados dos rótulos através dos erros encontrados e nos grupos com as amostras originais das bases de dados, desta forma foi possível mensurar a acurácia de cada rótulo. 
%Então, para avaliar a qualidade do rótulo quando existe um empate no número de erros

%Existe duas situações que podem acontecer para escolha do rótuloPara a escolha do melhor rótulo quando existir um empate entre os algoritmos
O funcionamento dos algoritmos foi também marcado pelas bases utilizadas, a base de dados 1 - Seeds, que é uma base de dados balanceada, e neste caso possui um número de exemplos iguais para as três classes (70 elementos cada classe), possuindo grupos bem distribuídos conforme gráfico exposto. Sua acurácia foi alta, considerando que a menor acurácia parcial de grupo chegou a 80\%, e a maior, aproximadamente 92\%. A base de dados 2 - Iris teve rótulos idênticos nos três algoritmos, Naive Bayes, CART e KNN. As acurácias foram altas e foi percebido a importância da largura da pétala (\textbf{petalwidth}) que foi o atributo que se repetio entre os clusters alterando somente seus intervalos.

Na base de dados 3 - Glass, os rótulos foram bastante satisfatórios ao se analisarem as acurácias dos algoritmos, chegando-se em alguns \textit{clusters} a acurácia de 100\%; contudo, os rótulos do Naive Bayes e CART, em específico no \textit{cluster} 4 (grupo \textbf{recipientes}), mesmo tendo poucos erros, a acurácia chegou em 77\%, sendo o valor mais baixo. Vale ressaltar que essa base não é classificada como uma base balanceada e no grupo recipientes conta com somente treze elementos. Já a base de dados 4 - Wine possui um total de 178 (cento e setenta e oito) registros, de forma balanceada, pois possui três grupos; um número mínimo de 48 (quarenta e oito) elementos em um grupo, e nos outros dois, 59 (cinquenta e nove) e 71 (setenta e um). Nesta base, foi a que teve acurácias mais baixas por rótulos, chegando a 43\% de acurácia nos \textit{clusters} 3 do CART e KNN, e mais alta no \textit{cluster} 2 do Naive Bayes com acurácia de 95\%.
Esta base é a que contém mais atributos e o tipo do vinho (classe 3) tem seus valores de atributos  bem distribuídos entre as classes 1 e 2 (pode ser visualizado no gráfico), refletindo em acurácias mais baixas na classe 3 em todos os algoritmos.

Outro ponto que merece destaque é a tomada de decisão através dos rótulos encontrados neste trabalho, isto é, acurácias mais altas resultam em boa confiabilidade dos rótulos, portanto, quando um especialista da área verificar um rótulo de um grupo, esse será capaz de perceber o que é importante para o grupo, podendo fazer uma tomada de decisão mais eficiente a partir destes resultados. 



%O funcionamento dos algoritmos foi também marcado pelas bases utilizadas, a base de dados 1 - Seeds, que é uma base de dados balanceada, e neste caso possui um número de exemplos iguais para as três classes (70 elementos cada classe), possuindo grupos bem distribuídos conforme gráfico exposto. Sua acurácia foi boa, considerando que a menor acurácia parcial de grupo chegou a 80\%, e a maior, aproximadamente 92\%. A base de dados 2 - Iris teve rótulos idênticos em dois algoritmos, Naive Bayes e KNN, e no CART a diferença ficou somente no \textit{cluster} 2 ( \textit{Iris-versicolor}), que deu maior importância para a largura da pétala (\textbf{petalwidth}) ao invés do comprimento da pétala (\textbf{petallength}) encontrada nos outros dois algoritmos, porém, a escolha do \textbf{petalwidth}, pelo CART, ocasionou um número de erros um pouco maior em comparação com os outros dois algoritmos (Naive Bayes e KNN). 

%Na base de dados 3 - Glass, os rótulos foram bastante satisfatórios ao se analisarem as acurácias dos algoritmos, chegando-se em vários \textit{clusters} a acurácia de 100\%; contudo, os rótulos do Naive Bayes e CART, em específico no \textit{cluster} 4 (grupo \textbf{recipientes}), mesmo tendo poucos erros, a acurácia chegou em 77\% em alguns atributos rótulos. Vale ressaltar que essa base não é classificada como uma base balanceada e no grupo recipientes conta com somente treze elementos. Já a base de dados 4 - Wine possui um total de 178 (cento e setenta e oito) registros, de forma balanceada, pois possui um número mínimo de 48 (quarenta e oito) elementos no grupo. Nesta base, os rótulos foram bastante semelhantes e diferenciados apenas no \textit{cluster} 1 do KNN, que obteve um atributo rótulo diferente dos outros algoritmos, porém o número de erros de todos os rótulos foram iguais, afirmando que os três possuem o mesmo poder de representar os \textit{clusters} com seus rótulos. 






%É importante destacar que este trabalho buscou analisar a aplicação de algoritmos com paradigmas diferentes, a partir das bases citadas, portanto, foi possível conhecer a acurácia de cada algoritmo testado, demonstrando que todos possuem alto grau de confiança, já que variaram de 82.5\% a 100\% na acurácia média, métrica esta calculada de acordo com os grupos extraído da própria fonte\footnote{UCI - Machine Learning Repository. http://archive.ics.uci.edu/ml/}, e assim confirmar que é possível realizar rotulação de dados. 

%Assim, uma análise desse trabalho pode ser descrita da seguinte forma: i) De acordo com a proposta de encontrar rótulos com algoritmos não antes testados em bases de dados já utilizadas em outros trabalhos, foi fielmente cumprido e 

%a análise aqui apresentada é eficiente perante os resultados, porém foi detectado que há  discretização e o número de faixas é essencial para  os resultado dos algoritmos na produção dos rótulos. A técnica utilizada nesta pesquisa segue a característica da aprendizagem supervisionada, ou seja, atributos de entrada se correlacionam para gerar uma saída desejável através de um atributo classe. Então, se os dados contém um grande número de características independentes com alta correlação com a classe, isso ajuda no aprendizado, contudo, se o combinação entre as variáveis é muito complexo, acaba dificultando o aprendizado dessa relação. Isto posto, percebe-se que a discretização tem influência direta nos rótulos encontrados, portanto, se um registro em um determinado método de discretização pertence a uma faixa de valor ao modificar esse método poderá também o registro mudar sua faixa, então, o que antes era rótulo pode não ser mais. 



%\section*{Trabalhos Futuros}
%\addcontentsline{toc}{chapter}{Trabalhos Futuros}
\section{Trabalhos Futuros}\label{cap:fut}



Espera-se realizar estudos mais profundos nos métodos de discretização, pois estes têm influência comprovada na geração dos rótulos nos grupos de dados e também no número de faixas, por estarem diretamente ligado aos métodos de discretização. A melhor discretização e o melhor número de faixas a serem utilizados estão relacionados aos valores das bases de dados utilizadas e, portanto, será necessário um estudo aprofundado para comparar os resultados com os deste trabalho. 



\begin{itemize}
 \item fgsd
 \item ddfs
\end{itemize}


