\chapter{Conclusões, Trabalhos Futuros}\label{cap:conclusao} 

Neste capítulo serão apresentadas as considerações finais da dissertação, bem como a apresentação dos benefícios da pesquisa realizada, que teve como base a rotulação de algoritmos com vistas a analisar o seu comportamento considerando algumas bases de dados consolidadas. Também serão apresentadas sugestões para continuidade deste trabalho já efetuado.

\section{Conclusão}\label{cond}

O presente trabalho dedicou-se a estudar a aplicabilidade de algoritmos supervisionados com bases de dados distintas, as bases utilizadas foram: Seeds, Iris, Glass e Wine, com a finalidade de definir a tupla atributo/valor de maior importância nos \textit{clusters}, determinando a sua rotulação. A formação do referido problema nasceu de outra pesquisa já concluída \cite{Lopes2016}, entretanto, teve como diferencial a realização de rotulação de grupos de dados a partir de algoritmos supervisionados não testados, Naive Bayes, Classification and Regression Trees - CART e K-Nearest Neighbor - KNN, assim, ao final foi possível demonstrar o comparativo de resultados.


As análises foram feitas a partir das bases de dados mencionadas acima, e o comportamento ao aplicar cada algoritmo nas bases e encontrar os rótulos foram satisfatórios. A avalição da qualidade destes rótulos foi feita da seguinte forma: inicialmente, escolheu-se as bases de dados já classificadas; feito isso foi possível saber quantas amostras pertencem a determinado grupo; em seguida, foi realizado um comparativo dos resultados dos rótulos através dos erros encontrados, e nos grupos com as amostras originais das bases de dados, desta forma foi possível mensurar a acurácia de cada rótulo. 
%Então, para avaliar a qualidade do rótulo quando existe um empate no número de erros

%Existe duas situações que podem acontecer para escolha do rótuloPara a escolha do melhor rótulo quando existir um empate entre os algoritmos

O funcionamento dos algoritmos foram também  marcados pelas bases utilizadas, a base de dados 1 - Seeds, é uma base de dados balanceada, que neste caso, possui um número de exemplos iguais para as três classes (70 elementos cada classe), destacando-se que a mesma obteve rótulos diferentes entre os três algoritmos, sua acurácia foi alta, considerando que a menor acurácia parcial de grupo chegou a 80\%, e a maior, aproximadamente 94\%. A base de dados 2 - Iris, tiveram rótulos idênticos em dois algoritmos, Naive Bayes e KNN, e no CART a diferença ficou somente no \textit{cluster} 2 ( \textit{Iris-versicolor}), que deu maior importância para o largura da pétala (\textit{petalwidth}) ao invés do comprimento da pétala (\textit{petallength}) encontrada nos outros dois algoritmos, porém, a escolha do \textit{petalwidth}, pelo CART, ocasionou um número de erros um pouco maior em comparação com os outros dois algoritmos (Naive Bayes e KNN). 

Na base de dados 3 - Glass, os rótulos foram bastantes satisfatórios ao  analisar as acurácias dos algoritmos chegando em vários \textit{clusters} a acurácias de 100\%, contudo, os rótulos do Naive Bayes e CART em específico no \textit{cluster} 4 (grupo recipientes), mesmo tendo poucos erros a acurácia chegou em 77\% em alguns atributos rótulos. Vale ressaltar que essa base não é classificada como uma base balanceada e no grupo, recipientes, conta com somente treze elementos. Já na base de dados 4 - Wine, possui um total de 178 (cento e setenta e oito) registros, de forma balanceada, pois possui um número mínimo de 48 (quarenta e oito) elementos no grupo. Nesta base os rótulos foram bastantes semelhantes, e diferenciados apenas no \textit{cluster} 1 do KNN que obteve um atributo rótulo diferentes dos outros algoritmos, porém o número de erros de todos os rótulos foram iguais afirmando que os três possuem o mesmo poder de representar os clusters com seus rótulos.


Outro ponto que merece destaque é a tomada de decisão através dos rótulos encontrados neste trabalho, isto é, acurácias altas resultam em boa confiabilidade dos rótulos, portanto quando um especialista da área verificar um rótulo de um grupo, este será capaz de perceber o que é importante para o grupo, podendo fazer uma tomada de decisão mais eficiente a partir destes resultados.

Assim, diante do problema de rotulação, que visa encontrar informações relevantes em grupos de dados ao ponto de identificar esses grupos, e considerando que \citeonline{Lopes2016} utilizou algoritmo supervisionado com paradigma conexionista (Redes Neurais) neste problema, esta pesquisa apresentou aplicação de novos algoritmos com paradigmas diferentes, ainda não testados, mostrando a viabilidade do método de rotulação de dados com algoritmos de aprendizado supervisionados, e comparando os resultados através das acurácias dos rótulos encontradas em cada \textit{cluster}.


%É importante destacar que este trabalho buscou analisar a aplicação de algoritmos com paradigmas diferentes, a partir das bases citadas, portanto, foi possível conhecer a acurácia de cada algoritmo testado, demonstrando que todos possuem alto grau de confiança, já que variaram de 82.5\% a 100\% na acurácia média, métrica esta calculada de acordo com os grupos extraído da própria fonte\footnote{UCI - Machine Learning Repository. http://archive.ics.uci.edu/ml/}, e assim confirmar que é possível realizar rotulação de dados. 

%Assim, uma análise desse trabalho pode ser descrita da seguinte forma: i) De acordo com a proposta de encontrar rótulos com algoritmos não antes testados em bases de dados já utilizadas em outros trabalhos, foi fielmente cumprido e 

%a análise aqui apresentada é eficiente perante os resultados, porém foi detectado que há  discretização e o número de faixas é essencial para  os resultado dos algoritmos na produção dos rótulos. A técnica utilizada nesta pesquisa segue a característica da aprendizagem supervisionada, ou seja, atributos de entrada se correlacionam para gerar uma saída desejável através de um atributo classe. Então, se os dados contém um grande número de características independentes com alta correlação com a classe, isso ajuda no aprendizado, contudo, se o combinação entre as variáveis é muito complexo, acaba dificultando o aprendizado dessa relação. Isto posto, percebe-se que a discretização tem influência direta nos rótulos encontrados, portanto, se um registro em um determinado método de discretização pertence a uma faixa de valor ao modificar esse método poderá também o registro mudar sua faixa, então, o que antes era rótulo pode não ser mais. 



%\section*{Trabalhos Futuros}
%\addcontentsline{toc}{chapter}{Trabalhos Futuros}
\section{Trabalhos Futuros}\label{cap:fut}



Espera-se realizar estudos mais profundos nos métodos de discretização, pois estes, tem influência comprovada na geração dos rótulos nos grupos de dados, e também no número de faixas, por está diretamente ligado aos métodos de discretização. A melhor discretização e o melhor número de faixas a ser utilizado está relacionado aos valores das bases de dados utilizada, e portanto  será necessário um estudo aprofundado para comparar os  resultados com os deste trabalho.

