% ----------------------------------------------------------
% Introdução 
% Capítulo sem numeração, mas presente no Sumário
% ----------------------------------------------------------

%\chapter*[Introdução]{Introdução}
%\addcontentsline{toc}{chapter}{Introdução}

% ----------------------------------------------------------
% Introdução 
% Capítulo com numeração, mas presente no Sumário
% ----------------------------------------------------------
\chapter{Introdução} \label{cap:introd}

Com a popularização da internet e mídias sociais, cada vez mais dados são processados, transportados e produzidos. E hoje, termo como Big Data, faz parte do cotidiano de empresas e pessoas. De acordo com o autor \citeonline{Montgomery2013} Big Data são os dados que excedem a capacidade de sistemas de banco de dados. É nesse cenário, com grandes volumes de dados, que não só a formação de grupos ganha importância, mas também a compreensão dos mesmos, pois a interpretação dos grupos fornecerá informações úteis para análises desses clusters.

Agrupamento de dados, ou clustering, é o termo que se usa para identificar dois ou mais objetos pertencentes ao mesmo grupo que compartilham um conceito em comum \cite{Kumar2013}. Cluster é um termo bastante pesquisado no aprendizado não-supervisionado (subárea do aprendizado de máquina) e  aplicada em vários contextos como segmentação de imagens, recuperação de informação e reconhecimento de objetos. Os algoritmos de agrupamento, conforme \citeonline{Kumar2013}, são aplicados em diferentes campos: Biologia (classificação de plantas e animais), Marketing (encontrar grupos de clientes com comportamentos semelhantes), planejamento de cidades (identificação de casas de acordo com seu tipo, valor e localização geográfica), entre outros.

O grau de escalabilidade dos dados gradativamente aumenta no decorrer dos anos, e embora os estudos sobre o problema de agrupamento de dados estejam avançados, fica cada vez mais complexo o entendimento dos clusters formados pela razão do número crescentes de grupos criados. Quanto maiores são os números de grupos produzidos mais  difícil são suas interpretações. 

Diante desse contexto é que se extrai a temática desta proposta de mestrado qual seja - ''Rotulação automática de grupos através de algoritimos supervisionados baseados em árvores e estatísticos'' - o estudo em questão dedica-se na aplicabilidade de dois algoritimos supervisionados, com paradigmas diferentes e bases de dados distintas, a fim de  definir a tupla atributo/valor de maior importância nos clusters, determinando um significado para estes clusters (rotulação).


A formação do problema desta pesquisa nasce a partir do trabalho realizado por \citeonline{LOPES2014},  que se dedicou a estudar a possibilidade de realização de  rotulação automática de grupos utilizando  algoritmo não-supervisionado (K-means) para formação de grupos e algoritmo supervisionado (Redes Neurais) para a rotulação. Assim, partindo deste estudo já realizado, este trabalho questiona-se:  É possível realizar rotulação de grupos de dados a partir de outros algoritmos supervisionados não testados, em específico Naive Bayes e CART?

Acredita-se que o resultado de tal problemática será positivo, considerando que os algoritmos Naives Bayes e CART também são categorizados como supervisionados. Além disso, é necessário mensurar ainda, a acurácia de cada resultado através do percentual de acertos dos atributos que são representados pelos rótulos gerados. Assim, se for possível demonstrar que a acurácia é de pelo menos 60\% ficará então comprovada a possibilidade de se fazer rotulação dos grupos de dados utilizando os algoritimos supervisionados Naive Bayes e CART, pois percentuais menores do rótulo não representariam o grupo. Importante destacar, que este trabalho não se preocupa em criar grupos, mas dar maior relevância à rotulação dos mesmos, isto é, compreender os grupos de dados já formados.

O termo rotulação, neste trabalho, segue a definição conforme \citeonline{LOPES2014}: 

%\newtheorem{teorema}{Definição}
    \begin{teorema}
    Dado um conjunto de clusters ${C=\{c_1,...,c_k | K \geqslant 1\} }$, de modo que cada cluster contém um conjunto de elementos ${c_i=\{\vec{e}_1,..,\vec{e}_{n^{(c_i)}}|n^{(c_i)} \geqslant 1 \}}$ que podem ser representados por um vetor de atributos definidos em ${\mathbb{R}^m }$ e expresso por ${ \vec{e}^{c_i}=(a_1,..,a_m)  }$ e ainda que  com ${ c_i \cap c_{i'}=\{\emptyset\} }$ com ${ 1 \leqslant i, i \leqslant K  }$ e ${ i \neq i' }$; o objetivo consite em apresentar um conjunto de rótulos ${ R=\{ r_{c1},...,r_{ck} \} }$, no qual cada rótulo específico é dados por um conjunto de pares de valores, atributo e seu respectivo intervalo, ${ r_{ci}=\{ (a_1,[p_1,q_1]),...,(a_{m^{(c_i)}}, ]p_{m^{(c_i)}},q_{m^{(c_i)}}]) \} }$ capaz de melhor expressar o cluster ${c_i}$ associado.
        %\footnotemark 
        %\footnotetext{Definição retirada de \cite{LOPES2014}}
        \begin{itemize}[noitemsep]
            \item ${K}$ é o número de clusters;
            \item ${c_i}$ é o i-ésimo cluster qualquer;
            \item ${n^{c_i}}$ é o número de elementos do cluster ${c_i}$;
            \item ${\vec{e}_{n^{(c_i)}}}$ se refere ao j-ésimo elemento pertencente ao cluster ${c_i}$;
            \item ${m}$ é a dimensão do problema;
            \item ${r_{c_i}}$ é o rótulo referente ao cluster ${c_i}$;
            \item ${]p_{m^{(c_i)}},q_{m^{(c_i)}}]}$ representa o intervalo de valores do atributo ${a_{m^{(c_i)}} }$, onde ${ p_{m^{(c_i)}} }$  é o limite inferior e ${ q_{m^{(c_i)}} }$ é o limite superior;
            \item ${m}$ é a dimensão do problema;
        \end{itemize}
    \label{teo:lopes:problema}
    \end{teorema}

Pode-se destacar um exemplo no qual a base de dados possui classes já definidas: macho, fêmea ou raça X, Y, Z, etc. E que ao criar esses grupos sabe-se que existe uma correlação das características dos grupos, acabando por não deixar visível qual característica se apresenta mais significativa dentro desses grupos. Tem-se na rotulação a intenção de definir algum significado para estes grupos, gerando um tipo de rótulo, ${ R=\{ r_{c1},...,r_{ck} \} }$, para melhor expressar o cluster ${c_i}$ associado (Definição \ref{teo:lopes:problema}).

Tecnicamente a informação do rótulo aplicada no cluster pode ajudar na tomada de decisão em algum contexto. A exemplo disso, supõe-se uma situação empregada na área urbana, onde pessoas circulam na cidade e imagina-se que os dados de controle de seus celulares estão sendo capturados pelas células das torres, e gravados em uma base de dados pelas operadoras. Uma vez em posse desses dados, são criados clusters podendo ser aplicado rotulação nestes grupos. E através dos rótulos pode-se personalizar alguns serviços para esses grupos já formados. 

Seguindo o exemplo dos dados capturados do celular, caso o rótulo (${r_{c_i}}$) de um cluster (${c_i}$) fosse o  atributo localização, e os valores  desse atributo escolhido para compor o rótulo, fossem as coordenadas geográficas, o qual definiriam o tipo de localização. Logo percebe-se que os participantes desse grupo possuem característica de frequentar alguma localização em comum. A interpretação deste rótulo poderá implicar em uma tomada de decisão personalizada para este grupo, objetivando otimizar um problema.

O trabalho em questão tem como objetivo principal demonstrar a  possibilidade de fazer rotulação de dados, em bases de dados com grupos já formados, utilizando dois algoritimos supervisionados distintos com paradigmas diferentes. Sendo este um algoritmo com paradigma estatístico - Naive Bayes -  e outro com paradigma simbólico - Classification And Regression Tree (CART).
 
% Para alcançar tal objetivo é necessário ???? (falar do objetivo de cada capítulo)foi estruturada mediante a codificação por intermédio de uma linguagem de natureza técnica, onde fez uso de módulos de aprendizado de máquina, atuando em bases de dados e obtendo como saída deste programa, os rótulos dos grupos. Uma vez que estes grupos já possuem informações do provedor das bases de dados de como foram criados.

Para alcançar tal objetivo é necessário conhecer as técnicas e tecnologias utilizadas nessa pesquisa. Uma das técnicas utilizada é a discretização (seção \ref{cap:refTeor:sec:discret}) onde ocorre a representação do tipo de variáveis contínuas em discretas. Outra técnica é a correlação entre atributos (seção \ref{cap:ferramentas:sec:tecnica}), visto que nesse processo é aplicado os algoritimos referentes da pesquisa. Todas essas técnicas foram estruturadas mediante a codificação por intermédio de uma linguagem de natureza técnica, onde fez uso de módulos de aprendizado de máquina, atuando em bases de dados e obtendo como saída deste programa, os rótulos dos grupos. 

Esta pesquisa é eminentemente quantitativa, pois se utiliza de algoritmos supervisionados para selecionar os atributos de maior relevância nos clusters, através de um percentual de correlação entre atributos, isto é, quanto maior esse percentual maior será a relevância desse atributo em relação aos outros. Além disso faz uma análise da base de dados de forma subjetiva para definir o número de faixas que serão divididos os valores, para realização da discretização. Uma vez escolhido o atributo de maior relevância e selecionada a faixa de valor que mais se repete nesse atributo, o resultado será o rótulo composto pela tupla: atributo mais importante e faixa selecionada. 

O trabalho será disposto em cinco capítulos já incluso  a Introdução e Conclusão, capítulos \ref{cap:introd} e \ref{cap:conclusao} respectivamente.  

O Referencial Teórico abordado no capítulo \ref{cap:refTeor} é responsável em esclarecer as tecnologias utilizadas nesta pesquisa sendo dividida em três seções. Inicialmente na seção \ref{cap:refTeor:sec:aprendMaq}, tem-se uma  explanação sobre aprendizado de máquina e quais os aprendizados indutivos são mais relevantes para este trabalho, ademais, a explicação dos dois algoritmos supervisionados utilizados para fazer rotulação de dados. Já na seção \ref{cap:refTeor:sec:discret} é realizado a divisão das faixas de valores de cada atributo, chamada de discretização. E logo na seção \ref{cap:refTeor:sec:trabcorrel} são apresentas pesquisas já consolidadas  referentes ao assunto de rotulação de clusters.

Na capítulo \ref{cap:ferramentas} é abordada a definição do problema da pesquisa. A partir dessa definição um modelo de resolução é definido e apresentado um fluxograma exibindo os processos a serem seguidos. Logo na seção \ref{cap:ferramentas:sec:tecnica} é demonstrado o funcionamento da técnica de correlação entre atributos. E na seção \ref{cap:ferramentas:sec:exebasemodfic} uma base de dados fictícia é utilizada para exemplificar a execução dos processos do modelo de resolução: discretização da base de dados no Processos (I), no Processo (II) é aplicado o algoritmo supervisionado e no Processo (III) o resultado da rotulação. 

%No capítulo \ref{cap:resultados} os resultados são apresentados separados por cada base de dados. Sendo que em cada algoritmo testado o resultado é dividio em cluster, atributo rótulo desse cluster, faixa de valores compondo o rótulo e mais dois campos expondo o grau de relevância, em porcentagem, de cada atributo em relação aos outros, junto com o número de elementos que não são representados pelo rótulo escolhido. A partir destas informações é retirado o rótulo o qual representará o cluster.

No capítulo \ref{cap:resultados} os resultados são separados por base de dados. Em cada seção referente a uma base de dados testada é criada duas subseções referentes aos algoritmos utilizados. Cada algoritmo apresenta uma tabela com informações desde o número do cluster, rótulos, relevância do atributo até acurácia do rótulo no cluster. É também exibida uma tabela possuindo os valores de relevância dos atributos por cluster, posto que esses valores desta tabela servirão de apoio para entender como o quão os atributos estão bem correlacionados.

%Sendo que em cada algoritmo testado o resultado é dividio em cluster, atributo rótulo desse cluster, faixa de valores compondo o rótulo e mais dois campos expondo o grau de relevância, em porcentagem, de cada atributo em relação aos outros, junto com o número de elementos que não são representados pelo rótulo escolhido. A partir destas informações é retirado o rótulo o qual representará o cluster.

Diante de todo o exposto fica claro que esta pesquisa além de dar continuidade a um tema específico aplicado na interpretação de agrupamento de dados, também serve como ponto de partida para outra pesquisa mais aprofundada, onde poderá esta tentar comprovar a possibilidade de fazer rotulação de dados utilizando qualquer algoritmo supervisionado.

% \section*{Motivação}\label{sec:motivacao}
% \addcontentsline{toc}{section}{Motivação}

% \lipsum[35]

% \section*{Objetivos}\label{sec:objetivos}
% \addcontentsline{toc}{section}{Objetivos}

% \lipsum[36]
