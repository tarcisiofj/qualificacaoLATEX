% ----------------------------------------------------------
% Introdução 
% Capítulo sem numeração, mas presente no Sumário
% ----------------------------------------------------------

\chapter*[Introdução]{Introdução}
\addcontentsline{toc}{chapter}{Introdução}

Esta pesquisa apresenta uma proposta de mestrado envolvendo como tema principal a rotulação de dados com algoritmos supervisionados. O foco é apresentar através de amostragem, que é possível fazer rotulação de dados com qualqueis algoritmos supervisionados. Sabendo da inviabilidade deste trabalho em  testar todos os algoritmos supervisionados, foram realizados, nesta proposta de mestrado, testes utilizando algoritmos com paradigmas diferentes para poder confirmar a proposta deste estudo.

Embora rotulação de dados seja uma área bem definida, o modo de abordagem aplicado nesta pesquisa vem alterando a maneira de como Aprendizagem de Máquina define este termo. Em pesquisas realizadas neste área sob supervisão do orientador desta proposta, vários trabalhos estão definindo Rotulação sendo algo diferente da Classificação dos dados. Apesar de várias literaturas \cite{Barber2011,Mitchell1997} entre outras citarem o termo rotulação como um sinônimo de classificação, neste departamento não é tratado asssim. Muitos trabalhos feitos aqui neste laboratório estão redefinindo o termo rotulação como algo mais completo e que possui propriedade diferente a apresentada na classificação.

A classificação é dada com um identificador do registro conforme suas características. Um especialista através de seu conhecimento técnico avalia esses registros e os define por algum classificador: macho ou fêmea, raça X, Y ou Z, etc. Mas na rotulação a intenção é pegar estes dados já classificados e definir algum significado desses grupos. 

Os clusters foram formados por algum tipo de similaridade entre os atributos que o compõe e tão diferentes quanto os atributos que pertencem a  clusters distintos. E embora os atributos já estivessem agrupados, seria importante esses grupos possuírem algum tipo de rótulo, onde o próprio analista ao percebe-los pudesse tecnicamente notar um significado nos grupos, e com essa informação, poder tomar alguma decisão sobre eles. A exemplo disso, na área de turismo onde pessoas circulam na cidade e imagina-se que os dados de seus celulares estão sendo armazenados pelas células das torres e armazenados em um base de dados. Pode-se criar clusters através do perfil desses dados e através desses clusters aplicar a rotulação de dados. Então em um cluster com um rótulo com característica de  localização, e o valor de maior repetição fosse coordenadas de uma academia. Com isso alguma decisão inteligente poderia ser feita para atender esse grupo. 

Muito se tem escrito sobre algoritmos de agrupamentos de dados, mas poucos falam em dar significado aos grupos formados. Apesar de parecer  lógico que a formação do cluster é por algum tipo de correlação das características do grupo, não fica claro qual característica se apresenta mais significativa dentro desse grupo. E este trabalho tem como cerne apresentar essa característica de maior importância no grupo , aplicando a técnica de rotulação de dados.

A rotulação é dividida em duas etapas distintas: correlação de atributos e discretização. Nessa técnica o rótulo é composto pelo resultado dessas duas etapas. Na correlação de atributos é onde acontece a aplicação do algoritmo supervisionado na base de dados, e gerado uma tabela como resultado. Essa tabela contém um valor expresso em porcentagem informando o grau de relevância dele em relação aos outro atributos. Na segunda etapa os valores passam por uma técnica de discretização e após isso, já com o atributo escolhido, verifica-se qual valor da faixa mais se repete neste atributo. Uma vez escolhido a faixa, é apresentado os limites dessa faixa, e com esses valores o rótulo é completado por o atributo de maior relevância junto com a faixa que mais se repete. 

Mesmo com cunho semelhante o trabalho de \cite{Lopes} fez o método de rotulação utilizando um algoritmo de paradigma conexionista, e aqui nesta pesquisa será aplicado dois algoritmos com paradigmas diferentes para poder servir de amostra e provar que a técnica de rotulação funciona para qualquer algoritmo supervisionado. Os algoritmos escolhidos foram: Naive Bayes (estatístico) e CART (simbólico).

Em consideração do problema exposto na sessão \ref{cap:metodologia} é proposto um modelo de resolução onde é realizada a rotulação de dados. 
...(explica o que foi feito no trabalho)...
% \section*{Motivação}\label{sec:motivacao}
% \addcontentsline{toc}{section}{Motivação}

% \lipsum[35]

% \section*{Objetivos}\label{sec:objetivos}
% \addcontentsline{toc}{section}{Objetivos}

% \lipsum[36]
