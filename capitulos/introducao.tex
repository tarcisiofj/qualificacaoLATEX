% ----------------------------------------------------------
% Introdução 
% Capítulo sem numeração, mas presente no Sumário
% ----------------------------------------------------------

\chapter*[Introdução]{Introdução}
\addcontentsline{toc}{chapter}{Introdução}

Esta pesquisa apresenta uma proposta de mestrado envolvendo como tema principal a rotulação de dados com algoritmos supervisionados. O foco é apresentar através de amostragem, que é possível fazer rotulação de dados com qualqueis algoritmos supervisionados. Sabendo da inviabilidade deste trabalho em  testar todos os algoritmos supervisionados, foram realizados, nesta proposta, testes utilizando algoritmos com paradigmas diferentes para poder confirmar a proposta deste estudo.

Embora rotulação de dados seja uma área bem definida, o modo de abordagem aplicado nesta pesquisa vem alterando a maneira de como Aprendizagem de Máquina define este termo, chamado rotulação. Em pesquisas realizadas neste área, sob supervisão do orientador deste trabalho, vários estudos estão definindo rotulação sendo algo diferente da classificação dos dados. Apesar de várias literaturas \cite{Barber2011,Mitchell1997} entre outras citarem o termo rotulação como um sinônimo de classificação, neste departamento não é tratado assim. Muitos trabalhos feitos aqui neste laboratório estão redefinindo o termo rotulação como complemento e que possui propriedade diferente a apresentada na classificação.

A classificação é dada com um identificador do registro conforme suas características. Um especialista através de seu conhecimento técnico avalia esses registros e os define por algum classificador: macho ou fêmea, raça X, Y ou Z, etc. Mas na rotulação a intenção é pegar estes dados já classificados e definir algum significado para esses grupos. 

Os clusters são formados por algum tipo de similaridade entre os atributos que o compõe, e tão diferentes quanto os atributos que pertencem a  clusters distintos. E embora os atributos já estivessem agrupados, seria importante esses grupos possuírem algum tipo de rótulo, onde o próprio analista ao percebê-los pudesse tecnicamente notar um significado nos grupos, e com essa informação, poder tomar alguma decisão sobre eles. 

A exemplo disso, supõe-se uma situação empregada na área urbana, onde pessoas circulam na cidade e imagina-se que os dados de controle de seus celulares estão sendo capturados pelas células das torres, e gravados em um base de dados pelas operadoras. Uma vez em posse desses dados, são criados clusters. E seria interessante a aplicabilidade da rotulação para personalizar alguns serviços para esses grupos já formados. Então, caso o rótulo de um cluster fosse o atributo localização, e os valores  desse atributo escolhido para compor o rótulo, fossem as coordenadas geográficas, que definiriam a localização de uma academia. Logo percebe-se que os participante desse grupo possuem característica de frequentar academias. Já em posse dessa informação  seria inteligente, da parte do analista, tomar alguma decisão personalizada para este grupo.

Muito se tem escrito sobre algoritmos de agrupamentos de dados, mas poucos falam em dar significado aos grupos formados. Apesar de parecer  lógico que a formação do cluster é por algum tipo de correlação das características do grupo, não fica claro qual característica se apresenta mais significativa dentro desse grupo. E este trabalho tem como cerne apresentar essa característica de maior importância no grupo , aplicando a técnica de rotulação de dados.

A rotulação é dividida em duas etapas distintas: correlação de atributos e discretização. Nessa técnica o rótulo é composto pelo resultado dessas duas etapas. Na correlação de atributos é onde acontece a aplicação do algoritmo supervisionado na base de dados, e gerado uma tabela como resultado. Essa tabela contém um valor expresso em porcentagem informando o grau de relevância dele em relação aos outro atributos. Na segunda etapa os valores passam por uma técnica de discretização e após isso, já com o atributo escolhido através da tabela de correlação, verifica-se qual valor da faixa mais se repete neste atributo. Uma vez escolhido a faixa, é apresentado os limites dessa faixa, e com esses valores o rótulo é apresentado por um conjunto de tuplas formadas pelo atributo, de maior relevância, e com os limites da faixa que mais se repetem. 

Mesmo com cunho semelhante, o trabalho de \cite{Lopes} fez o método de rotulação utilizando um algoritmo com paradigma conexionista, diferente ao empregado aqui. Mas nesta pesquisa, será aplicado dois algoritmos com paradigmas diferentes servindo de amostragem para provar que a técnica de rotulação funciona para quaisquer algoritmos supervisionados. Os algoritmos escolhidos foram: Naive Bayes (paradigma estatístico) e CART (paradigma simbólico).

Na sessão ~\ref{cap:ferramentas:sec:considproblema} é abordada a definição do problema da pesquisa. A partir da definição do problema um modelo de resolução é apresentado através do fluxograma, figura ~\ref{fig:modeloresolucao}. Neste modelo é aplicado as duas etapas principais: Processos I e II. Essas fases serão utilizadas no processo de rotulação (III). No processo I  o cluster é submetido a uma técnica de discretização, dentre as duas apresentadas, Discretização por Frequências Iguais - EFD ou Discretização por Larguras Iguais. A outra etapa, processo II, é aplicado o algoritmo supervisionado na base de dados e gerado uma tabela com os valores em porcentagem dos relacioanamentos entre os atributos, onde quanto maior valor mais bem correlacionado o atributo é em relação aos outro.

A rotulação acontece na composição dessas duas etapas,..


% \section*{Motivação}\label{sec:motivacao}
% \addcontentsline{toc}{section}{Motivação}

% \lipsum[35]

% \section*{Objetivos}\label{sec:objetivos}
% \addcontentsline{toc}{section}{Objetivos}

% \lipsum[36]
