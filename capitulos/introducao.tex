% ----------------------------------------------------------
% Introdução 
% Capítulo sem numeração, mas presente no Sumário
% ----------------------------------------------------------

%\chapter*[Introdução]{Introdução}
%\addcontentsline{toc}{chapter}{Introdução}

% ----------------------------------------------------------
% Introdução 
% Capítulo com numeração, mas presente no Sumário
% ----------------------------------------------------------
\chapter{Introdução} \label{cap:introd}

Agrupamento de dados, ou \textit{clustering}, é o termo usado para identificar dois ou  mais objetos pertencentes ao mesmo grupo que compartilham um conceito em comum    \cite{Kumar2013}. \textit{Cluster} é um termo bastante pesquisado no aprendizado não-supervisionado (subárea do aprendizado de máquina) e aplicado em vários contextos, como segmentação de imagens, recuperação de informação e reconhecimento de objetos. Os algoritmos de agrupamento, conforme \citeonline{Kumar2013}, são aplicados em diferentes campos: Biologia (classificação de plantas e animais); Marketing (encontrar grupos de clientes com comportamentos semelhantes); Planejamento de cidades (identificação de casas de acordo com seu tipo, valor e localização geográfica), entre outros. 


Com a popularização da internet e mídias sociais, cada vez mais dados são processados, transportados e produzidos. É nesse cenário, com grandes volumes de dados, que não só a formação de grupos ganha importância, mas também torna-se importante sua compreensão, pois a interpretação dos grupos fornecerá informações úteis para análise desses \textit{clusters}. 

O grau de escalabilidade dos dados aumenta gradativamente no decorrer dos anos e, embora os estudos sobre o problema de agrupamento de dados estejam avançados, fica cada vez mais complexo entender como são formados esses \textit{clusters} em razão do número crescente de grupos e suas características dificultando suas interpretações. 

Diante desse contexto é que se extrai a temática desta proposta de dissertaão. O estudo em questão dedica-se à aplicabilidade de algoritmos supervisionados com paradigmas diferentes em bases de dados distintas, a fim de definir a tupla, atributo/faixa de valor, de maior importância nos \textit{clusters}, determinando um significado para estes \textit{clusters} (rotulação). 

A rotulação dita neste trabalho segue a própria definição da palavra, que serve para informar sobre algo. Então, a partir de um grupo de dados, seria possível destacar neste grupo uma informação que o represente e uma forma seria encontrar por meio de técnicas uma tupla, atributo/faixa de valor, em que o atributo selecionado seria o que teria maior taxa de acerto resultante da aplicação do algoritmo de aprendizado supervisionado no grupo, a ponto de representá-lo, e faixa de valor seria um intervalo que mais houvesse ocorrência de dados do atributo. Poderá também haver no grupo mais de um atributo com sua respectiva faixa representando o rótulo. 



A formação do problema desta pesquisa nasce a partir do trabalho realizado por \citeonline{Lopes2016}, que se dedicou a estudar a possibilidade de realização de rotulação automática de grupos, utilizando, para isso, dois algoritmos: i) Um para realizar a formação de grupos por meio de algoritmo não-supervisionado (K-means); e ii) utiliza o algoritmo supervisionado (Redes Neurais Artificiais - RNAs) para fazer a rotulação de grupos. Assim, partindo do estudo já realizado, este trabalho se dedica a realizar rotulação de grupos de dados a partir de outros algoritmos supervisionados de paradigmas diferentes não testados e realizando um comparativo entre eles comprovando a eficácia do método. 

Nesta pesquisa foi aferida a acurácia de cada resultado por meio do percentual de acertos dos atributos que são representados pelos rótulos gerados, sendo este cálculo possível em virtude da comparação dos registros representados pelos rótulos com os registros que participam do grupo. Isto posto, no desenvolvimento deste trabalho não há preocupação na criação de grupos e sim na rotulação dos mesmos, isto é, compreender os grupos de dados já formados. 


Quando se analisam grupos que já estão formados, sabe-se que há uma correlação das características pelos quais seus dados se mantêm agrupados. Acontece que, com grandes números de grupos e características sendo criadas, isso acaba por não deixar visível qual característica se apresenta mais significativa dentro desses grupos.

Tecnicamente, a informação do rótulo aplicada no \textit{cluster} pode ajudar na tomada de decisão em algum contexto. O exemplo \ref{ex:cel} mostra uma situação que pode ser empregada a rotulação de grupos de dados em uma tomada de decisão.

\begin{exemplo} \label{ex:cel}
Supõe-se uma situação empregada na área urbana, onde pessoas circulam, e imagina-se que os dados de controle de seus celulares estão sendo capturados pelas células das torres e gravados em uma base de dados pelas operadoras. Uma vez em posse desses dados, são criados clusters, podendo ser aplicada rotulação nestes grupos e por meio disso pode-se personalizar alguns serviços para os grupos já formados. Caso o rótulo (${r_{c_i}}$) de um \textit{cluster} (${c_i}$) fosse o atributo localização e os valores desse atributo escolhido para compor o rótulo fossem as coordenadas geográficas, definir-se-ia a região de localização. Logo, percebe-se que os participantes desse grupo possuem a característica de frequentar alguma localização em comum. A interpretação deste rótulo poderá implicar uma tomada de decisão personalizada para o grupo, objetivando otimizar um problema. 
\end{exemplo}
% (objetivo) O trabalho em questão tem como objetivo principal demonstrar a possibilidade de fazer rotulação de dados, em bases de dados com grupos já formados, utilizando dois algoritmos supervisionados distintos. 
 

%Para alcançar tal objetivo é necessário conhecer as técnicas e tecnologias utilizadas nessa pesquisa. Uma das técnicas utilizada é a discretização (seção \ref{cap:refTeor:sec:discret}) onde ocorre a representação do tipo de variáveis contínuas em discretas. Outra técnica é a correlação entre atributos (seção \ref{cap:ferramentas:sec:tecnica}), visto que nesse processo é aplicado os algoritimos referentes da pesquisa. Todas essas técnicas foram estruturadas mediante a codificação por intermédio de uma linguagem de natureza técnica, onde fez uso de módulos de aprendizado de máquina, atuando em bases de dados e obtendo como saída deste programa, os rótulos dos grupos. 

% Esta pesquisa se utiliza de algoritmos supervisionados para selecionar os atributos de maior relevância nos clusters, através de um percentual de correlação entre atributos, isto é, quanto maior esse percentual maior será a relevância desse atributo em relação aos outros. Além disso faz uma análise da base de dados de forma subjetiva para definir o número de faixas que serão divididos os valores, para realização da discretização. Uma vez escolhido o atributo de maior relevância e selecionada a faixa de valor que mais se repete nesse atributo, o resultado será o rótulo composto pela tupla: atributo mais importante e faixa selecionada. 

O trabalho será disposto em cinco capítulos, já inclusas a Introdução e Conclusão nos capítulos \ref{cap:introd} e \ref{cap:conclusao} respectivamente. O Referencial Teórico, abordado no capítulo \ref{cap:refTeor}, esclarece as tecnologias utilizadas nesta pesquisa, que é dividida em três seções. No capítulo \ref{cap:ferramentas} é abordada a definição do problema da pesquisa. A partir dessa definição, um modelo de resolução é definido e apresentado um fluxograma exibindo os processos a serem seguidos. E  no capítulo \ref{cap:resultados},os resultados são separados por base de dados. Em cada seção referente a uma base de dados testada são criadas subseções referentes aos algoritmos utilizados  e depois uma subseção de avaliação dos resultados dos algoritmos.

%Inicialmente na seção \ref{cap:refTeor:sec:aprendMaq}, em-se uma explanação sobre aprendizado de máquina e quais os aprendizados indutivos são mais relevantes para este trabalho; ademais, a explicação dos algoritmos supervisionados utilizados para fazer rotulação de dados. Já na seção \ref{cap:refTeor:sec:discret} é realizada a divisão das faixas de valores de cada atributo, chamada de discretização. E logo na seção \ref{cap:refTeor:sec:trabcorrel} são apresentadas pesquisas já consolidadas referentes a assuntos como aprendizado de máquina, classificação, agrupamentos e rotulação de dados. 

%Logo na seção \ref{cap:ferramentas:sec:tecnica}é demonstrado o funcionamento da técnica de correlação entre atributos. E na seção \ref{cap:ferramentas:sec:exebasemodfic} uma base de dados fictícia é utilizada para exemplificar a execução dos processos do modelo de resolução nas seguintes etapas: discretização da base de dados, a aplicação do algoritmo supervisionado e resultado da rotulação. Cada algoritmo apresenta uma tabela com informações desde o número do \textit{cluster}, rótulos, relevância do atributo até acurácia do rótulo no \textit{cluster}. É também exibida uma tabela possuindo os valores de relevância dos atributos por \textit{cluster}, posto que os valores desta tabela servirão de apoio para entender o quão os atributos estão bem correlacionados. 

%Sendo que em cada algoritmo testado o resultado é dividio em cluster, atributo rótulo desse cluster, faixa de valores compondo o rótulo e mais dois campos expondo o grau de relevância, em porcentagem, de cada atributo em relação aos outros, junto com o número de elementos que não são representados pelo rótulo escolhido. A partir destas informações é retirado o rótulo o qual representará o cluster.

%Diante do exposto fica claro que esta pesquisa além de dar continuidade, visto que novos algoritmos são adicionados, a um tema específico aplicado na interpretação de agrupamento de dados, também serve como ponto de partida para outra pesquisa mais aprofundada. Pesquisa esta que poderá comprovar a possibilidade de fazer rotulação de dados utilizando qualquer algoritmo supervisionado.
% \section*{Motivação}\label{sec:motivacao}
% \addcontentsline{toc}{section}{Motivação}

% \lipsum[35]

% \section*{Objetivos}\label{sec:objetivos}
% \addcontentsline{toc}{section}{Objetivos}

% \lipsum[36]
