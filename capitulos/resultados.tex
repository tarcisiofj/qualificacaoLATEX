\chapter{Resultados}\label{cap:resultados}

Os resultados obtidos aqui neste capítulo foram referentes a aplicação do método de rotulação em ${3(três)}$ bases de dados distintas. Um dos primeiros passos na análise de aprendizagem de máquina é quando o analista prepara os dados para poder utilizar  um método de aprendizagem apropriado. 

Então  a escolha da base de dados também tem influência direta em bons resultados. E sabendo disso a escolha dos conjuntos de dados utilizados nesta pesquisa foi por conta delas apresentarem características diferentes, e também por serem conhecidas, facilitando a análise e servindo de amostra a outras base.

\section{Implementação}

Para conseguir gerar os resultados aqui escritos foram feitas implementações utilizando a ferramenta MATLAB \footnote{http://www.mathworks.com/products/matlab/}, onde junto  a ela é possível utilizar suas funções de aprendizado de máquina já prontas. MATLAB possui uma linguagem técnica, e de fácil implementação por já possuir uma gama de funções\footnote{versão: R2016a(9.0.0.341360); 64-bit (glnxa64)} preparadas para aprendizado de máquina. Por esses motivos essa ferramenta foi escolhida para colocar em prática essa pesquisa.

Foram realizados vários testes com o intuito de tentar otimizar resultados e poder comparálos a outras pesquisas já escritas. Seguindo essa linha foi determinado a escolha de ${3(três)}$ bases de dados já conhecidas, onde na implementação de cada uma delas surgiu algumas alterações, dependendo da base, na variável(V), quantidade de faixas(R) e método de discretização(EWD,EFD). Essas mudanças para cada base servirão para otimizar os resultados. 

Cada base de dados será aplicado dois algoritmos de aprendizado supervisionado que possuem paradigmas diferentes para servir de amostra e poder assim tirar conclusões sobre a rotulação em quaisquer algoritmos supervisionados.  

Os algoritmos utilizados foram o Naive Bayes, sessão \ref{cap:refTeor:sssec:nbayes}, com paradigma estatístico. E também o algoritmo Classification e Regression Trees - CART, \ref{cap:refTeor:sssec:cart}, com paradigma simbólico de  árvore de decisão.

\section{Seeds - Identificação de Tipos de Semente}
Essa base foi extraída da UCI Machine Learning\footnote{http://archive.ics.uci.edu/ml/}, composta por ${7(sete)}$ atributos definindo suas características e mais uma definindo sua classificação, sendo este último, um atributo classe  responsável por identificar o tipo de semente. Possuindo um total de 210 registros classificados em ${3(três)}$ categorias:
\begin{itemize}[noitemsep]
 \item 70 elementos do tipo Kama;
 \item 70 elementos do tipo Rosa;
 \item 70 elementos do tipo Canadian.
\end{itemize}
Na configuração de implementação foi utilizado o método EFD de discretização com divisão em três faixas, ${R=3}$ para todos os atributos, e inserido o valor de variação ${V=3\%}$.

Na tabela \ref{tab:rot:seeds:nb} e tabela \ref{tab:rot:seeds:cart} são  apresentados os resultados com a execução do algoritmo Naive Bayes e CART respectivamente. Elas são formadas por uma coluna informando os \textbf{Clusters}, \textbf{Rótulos}  compostos pelo \textbf{Atributo} e sua \textbf{Faixa} de valor. Junto também a coluna \textbf{Relevância} exibindo a resposta do algoritmo em porcentagem, da correlação do atributo em relação aos outros atributos do cluster, retirado da tabela \ref{tab:matrelevancia:seeds:nb} e da tabela \ref{tab:matrelevancia:seeds:cart} respectivamente. E por último a coluna \textbf{Elem Fora da Faixa} que mostra a quantidade de elementos que não estão dentro da faixa do rótulo. Essa última coluna tem a função de exibir em números a quantidade de valores que não estão participando da porcentagem da coluna de \textbf{Relevância}. Para quem está analisando a tabela é interessante mais essa dado, pois pode compara com o total de elementos do grupo.

\subsection{Naive Bayes} \label{cap:resultados:ssec:seed:nb}
% Please add the following required packages to your document preamble:
% \usepackage{multirow}
% \usepackage[table,xcdraw]{xcolor}
% If you use beamer only pass "xcolor=table" option, i.e. \documentclss[xcolor=table]{beamer}
\begin{table}[!h]
\centering
\caption{Resultado da aplicação do algoritmo Naive Bayes}
\label{tab:rot:seeds:nb}
\begin{tabular}{llcrc}
\hline
\multicolumn{1}{c}{\cellcolor[HTML]{FFFFFF}} & \multicolumn{2}{c}{Rótulos}                & \multicolumn{1}{r}{}               & \\ \cline{2-3}
Cluster                                      & Atributos      & \multicolumn{1}{c}{Faixa} & \multicolumn{1}{c}{Relevância(\%)} & Elem fora da Faixa\\ \hline \hline
1                                            & area           & ] 12.78 $\sim$  16.14 ]   & 92\%                               & 14\\  \hline
                                             & area           & ] 16.14 $\sim$  21.18 ]   & 95\%                               & 6\\ 
\multirow{-2}{*}{2}                          & lkernel        & ] 5.826 $\sim$  6.675 ]   & 92\%                               & 6\\  \hline
3                                            & perimetro      & [ 12.41 $\sim$  13.73 ]   & 95\%                               & 5\\ \hline \hline
\end{tabular}
\end{table}



Analisando a coluna rótulo da tabela \ref{tab:rot:seeds:nb}, nota-se que o atributo \textbf{area} aparece tanto no  cluster 1 como também no cluster 2. A técnica envolve não só o rótulo como também a faixa que os valores mais se repetem dentro do atributo. Nesse caso pode-se observar que o atributo se repete entre os clusters. Mas no cluster 1, a faixa de valores difere do cluster 2, sem comentar que no cluster 2 existe outro atributo compondo o rótulo, \textbf{lkernel}.

A seleção dos atributos rótulos acontece da diferença da variável ${V=3\%}$ em relação ao atributo de maior relevência. Caso essa variável tenha o valor alterado, os rótulos dos clusters poderão sofrer mudanças, pois poderá aumentar  ou diminuir o número de atributos dos rótulos, dependendo do valor inserido em ${V}$. Através da tabela \ref{tab:matrelevancia:seeds:nb} é possível analisar todos os valores de relevância gerados para os atributos e analisar qual valor pode-se inserir em ${V}$ para montar o rótulo.

\begin{table}[!h]
    
    \caption{Resultado da Correlação dos atributos pelo Naive Bayes; Legenda dos Atributos: (A)area, (B)perimetro, (C)compacteness, (D)Lkernel, (E)Wkernel, (F)asymetry, (G)lkgroove}    
    \centering
   \small\addtolength{\tabcolsep}{+2pt}
    \begin{tabular}{|cl|c|c|c|c|c|c|c|}
        \hline \hline
                                &   & \multicolumn{7}{c|}{Atributos}          \\ \cline{3-9} 
        \multicolumn{1}{|l}{}                            &   & A    & B & C & D & E & F & G \\ \hline
        \multicolumn{1}{|c|}{}                           & 1 & 92.8 & 87.1   & 50.0      & 75.7 & 85.7 & 60.0   & 65.7   \\ \cline{2-9} 
        \multicolumn{1}{|c|}{}                           & 2 & 95.7 & 91.4   & 47.1      & 92.8 & 90.0 & 28.5  & 85.7  \\ \cline{2-9} 
        \multicolumn{1}{|c|}{\multirow{-3}{*}{Clusters}} & 3 & 91.4 & 95.7   & 71.4      & 85.7 & 91.4 & 64.2  & 58.5  \\ \hline
    \end{tabular}
    \label{tab:matrelevancia:seeds:nb} 
\end{table}

A tabela \ref{tab:matrelevancia:seeds:nb} é formada por clusters representado pelas linhas, e colunas representado por atributos. Essa tabela é fruto da implementação do Naive Bayes em cima dessa base de dados, e foi  gerada para auxiliar a retirada dos atributos rótulos. Uma análise pode ser feita através desses dados e ajudar a definir um valor para a variável ${V}$. Percebe-se que algumas características são mais bem correlacionadas que  outras, através de seus valores mais altos. Isso indica o grau de relacionamento entre os atributos após a aplicação do algoritmo. 


\begin{table}[!h]
\caption{Resultado de ${4(quatro)}$ execuções do algoritmo Naive Bayes; Legenda dos Atributos: (A)area, (B)perimetro, (C)compacteness, (D)Lkernel, (E)Wkernel, (F)asymetry, (G)lkgroove}
 \begin{tabular}{ll}
%\rule{0}{50}

  
   %\scalebox{0.5}{%
   \small\addtolength{\tabcolsep}{-4pt}
     \begin{tabular}{|cl|c|c|c|c|c|c|c|}
        \hline \hline
            {\tiny 1a. Execução}     &   & \multicolumn{7}{c|}{Atributos}                                               \\ \cline{3-9} 
       \multicolumn{1}{|l}{}                            &   & A    & B & C & D & E & F & G \\ \hline
        \multicolumn{1}{|c|}{}                           & 1 & 92.8 & 87.1   & 48.5      & 77.1 & 82.8 & 57.1   & 65.7   \\ \cline{2-9} 
        \multicolumn{1}{|c|}{}                           & 2 & 94.2 & 90.0   & 45.7      & 92.8 & 90.0 & 38.5  & 87.1  \\ \cline{2-9} 
        \multicolumn{1}{|c|}{\multirow{-3}{*}{Clusters}} & 3 & 91.4 & 95.7   & 72.8      & 85.7 & 91.4 & 64.2  & 60.0  \\ \hline
      \end{tabular}
    %}
 &
 %\hspace{1cm} %altera o espaçamento entre as tabelas
 
   
 %\scalebox{0.5}{%
   \small\addtolength{\tabcolsep}{-4pt}
   \begin{tabular}{|cl|c|c|c|c|c|c|c|}
        \hline \hline
             {\tiny 2a. Execução }       &   & \multicolumn{7}{c|}{Atributos}                                               \\ \cline{3-9} 
       \multicolumn{1}{|l}{}                            &   & A    & B & C & D & E & F & G \\ \hline
        \multicolumn{1}{|c|}{}                           & 1 & 92.8 & 87.1   & 47.1      & 77.1 & 87.1 & 60.0   & 65.7   \\ \cline{2-9} 
        \multicolumn{1}{|c|}{}                           & 2 & 94.2 & 90.0   & 47.1      & 92.8 & 91.4 & 32.8  & 87.1  \\ \cline{2-9} 
        \multicolumn{1}{|c|}{\multirow{-3}{*}{Clusters}} & 3 & 91.4 & 95.7   & 72.8      & 85.7 & 92.8 & 64.2  & 60.0  \\ \hline
      \end{tabular}
  %}
  \\  [8ex]
 %\hspace{1cm} %altera o espaçamento entre as tabelas
 %\rule{0}{50}
 
 %\scalebox{0.5}{%
   \small\addtolength{\tabcolsep}{-4pt}
   \begin{tabular}{|cl|c|c|c|c|c|c|c|}
        \hline \hline
          {\tiny 3a. Execução}     &   & \multicolumn{7}{c|}{Atributos}                                               \\ \cline{3-9} 
       \multicolumn{1}{|l}{}                            &   & A    & B & C & D & E & F & G \\ \hline
        \multicolumn{1}{|c|}{}                           & 1 & 94.2 & 85.7   & 48.5      & 77.1 & 82.8 & 61.4   & 65.7   \\ \cline{2-9} 
        \multicolumn{1}{|c|}{}                           & 2 & 92.8 & 90.0   & 50.0      & 92.8 & 90.0 & 32.8  & 87.1  \\ \cline{2-9} 
        \multicolumn{1}{|c|}{\multirow{-3}{*}{Clusters}} & 3 & 91.4 & 95.7   & 72.8      & 85.7 & 92.8 & 64.2  & 60.0  \\ \hline
   \end{tabular}
    
    &
    
       \small\addtolength{\tabcolsep}{-4pt}
   \begin{tabular}{|cl|c|c|c|c|c|c|c|}
        \hline \hline
            {\tiny 4a. Execução }   &   & \multicolumn{7}{c|}{Atributos}                                               \\ \cline{3-9} 
       \multicolumn{1}{|l}{}                            &   & A    & B & C & D & E & F & G \\ \hline
        \multicolumn{1}{|c|}{}                           & 1 & 91.4 & 88.5   & 54.2      & 75.7 & 85.7 & 62.8   & 61.4   \\ \cline{2-9} 
        \multicolumn{1}{|c|}{}                           & 2 & 95.7 & 90.0   & 50.0      & 92.8 & 90.0 & 38.5  & 85.7  \\ \cline{2-9} 
        \multicolumn{1}{|c|}{\multirow{-3}{*}{Clusters}} & 3 & 91.4 & 95.7   & 72.8      & 85.7 & 94.2 & 64.2  & 57.1  \\ \hline
   \end{tabular}
   \\
 
 \end{tabular}
 \label{tab:execucoes:seed:nb}
\end{table}

Para conseguir ter uma idéia mais ampla dessas informações, na tabela \ref{tab:execucoes:seed:nb} é exposto o resultado de ${4(quatro)}$ execuções do Algoritmo Naive Bayes, e pode-se constatar que mesmo havendo algumas alterações em seus valores nos atributos em cada execução, a correlação entre os atributos não oferece muita alteração. Como exemplo, o atributo \textbf{area}, possui o melhor grau de correlacionamento em seu grupo, mesmo testado em quatro execuções, como mostrado na tabela \ref{tab:execucoes:seed:nb}.

Segue abaixo o resultado do algoritmo Naive Bayes na base de dados \textbf{Seeds} com seus rótulos: 
\begin{itemize}[noitemsep]
 \item ${r_{c_1}=\{ (area, ]12.78 \sim 16.14]) \} }$  
 \item ${r_{c_2}=\{ (area, ]16.14 \sim 21.18]), (Lkernel, ]5.826 \sim 6.675]) \} }$
 \item ${r_{c_3}=\{ (perimetro, [12.41 \sim 13.73])\} }$
\end{itemize}


\subsection{CART}\label{cap:resultados:ssec:seed:cart}


Já na tabela \ref{tab:rot:seeds:cart}, tem-se o resultado da aplicação do algoritmo supervisionado CART. Ele é implementado para solucionar casos de árvore de decisão pelo MATLAB. O intuito é testar a base de dados no paradigma simbólico. 

\begin{table}[!h]
\centering
\caption{Resultado da aplicação do algoritmo CART}
\label{tab:rot:seeds:cart}
\begin{tabular}{llcrc}\hline

\multicolumn{1}{c}{\cellcolor[HTML]{FFFFFF}} & \multicolumn{2}{c}{Rótulos}                      & \multicolumn{1}{r}{}            \\ \cline{2-3}
Cluster                                      & Atributos      & \multicolumn{1}{c}{Faixa}       & \multicolumn{1}{c}{Relevância(\%)} & Elem fora da Faixa \\ \hline \hline
                                             & area           & ] 12.78 $\sim$  16.14 ]         & 91\%          & 14 \\  
\multirow{-2}{*}{1}                          & perimetro      & [ 13.73 $\sim$ 15.18 ]          & 94\%          & 14\\ \hline
                                             & area           & ] 16.14 $\sim$  21.18 ]          & 95\%         & 6 \\ 
\multirow{-2}{*}{2}                          & perimetro      & ] 15.18 $\sim$  17.25 ]          & 98\%         & 7\\  \hline
                                             & perimetro      & [ 12.41 $\sim$  13.73 ]         & 95\%          & 5 \\
\multirow{-2}{*}{3}                          & wkernel        & [ 2.63 $\sim$  3.049 ]         & 97\%           & 9\\ \hline \hline
\end{tabular}
\end{table}

Pode-se verificar na tabela \ref{tab:rot:seeds:cart} que os clusters 1 e 2 possuem o mesmo conjunto de atributos selecionados no campo de rótulo. Mas isso não implica dizer que os dois grupos são identificados pelo mesmo rótulo. O rótulo é composto pelos atributos e pelas faixas, onde a faixa é escolhida é a faixa onde se tem o maior número de valores que se repetem nessa faixa. Então, caso exista um vetor de elementos já discretizados, ${\vec{e}_{(c_i)}=\{1,1,1,2,2,2,2,3,3\}}$. Neste vetor o valor que mais se repete é o ${2}$ , então a faixa ${2}$ foi a que mais se repetiu e com isso é a escolhida para compor o rótulo com o atributo mais relevante.

Para entender a escolha desses atributos no campo de rótulos, a tabela \ref{tab:matrelevancia:seeds:cart} exibe o resultado gerado na execução do algoritmo em cima da base. Cada valor desses é o resultado da aplicação do algoritmo enquanto o atributo era a classe da vez coforme figura ~\ref{fig:tecnicamodelocomp}, sessão \ref{cap:ferramentas:ssec:algsuper}. O atributo de maior valor junto com os atributos da diferença de ${V}$ com o mais relevante, são escolhidos para ser rótulos. Na linha(cluster) 1 o maior valor é o atributo perimetro. Pega o valor encontrado em perimetro, e subtrai de ${V=3}$. A partir daí o(s) atributo(s) que possuí(rem) um valor que está entre este resultado até o mais alto, irá compor o rótulo.

\begin{table}[!h]
    
    \caption{Resultado da Correlação dos atributos pelo CART; Legenda dos Atributos: (A)area, (B)perimetro, (C)compacteness, (D)Lkernel, (E)Wkernel, (F)asymetry, (G)lkgroove}    
    \centering
   \small\addtolength{\tabcolsep}{+2pt}
    \begin{tabular}{|cl|c|c|c|c|c|c|c|}
        \hline \hline
                                &   & \multicolumn{7}{c|}{Atributos}          \\ \cline{3-9} 
        \multicolumn{1}{|l}{}                            &   & A    & B & C & D & E & F & G \\ \hline
        \multicolumn{1}{|c|}{}                           & 1 & 91.4 & 94.2   & 58.5      & 80.0 & 81.4 & 61.4   & 61.4   \\ \cline{2-9} 
        \multicolumn{1}{|c|}{}                           & 2 & 98.5 & 98.5   & 51.4      & 90.0 & 88.5 & 42.8  & 88.5  \\ \cline{2-9} 
        \multicolumn{1}{|c|}{\multirow{-3}{*}{Clusters}} & 3 & 92.7 & 95.7   & 80.0      & 88.5 & 97.1 & 58.5  & 78.5  \\ \hline
    \end{tabular}
    \label{tab:matrelevancia:seeds:cart} 
\end{table}

Foram realizadas vários teste, onde alguns deles estão na tabela \ref{tab:execucoes:seed:cart}. Essas operações foram execuções do algoritmo CART em cima da base, para provar que a técnica de correlação de atributos, \ref{cap:ferramentas:sec:tecnica}, é funcional para este algoritmo. O mesmo pode ser visto no algoritmo de paradigma estatístico, sessão \ref{cap:resultados:ssec:seed:nb}, realizado nessa pesquisa. O  comportamento de ambos foram bem semelhantes, pois eles seguem o padrão de valores os quais não se alteram muito a cada iteração.

\begin{table}[!h]
\caption{Resultado de ${4(quatro)}$ iterações do algoritmo CART; Legenda dos Atributos: (A)area, (B)perimetro, (C)compacteness, (D)Lkernel, (E)Wkernel, (F)asymetry, (G)lkgroove}
 \begin{tabular}{ll}
%\rule{0}{50}

  
   %\scalebox{0.5}{%
   \small\addtolength{\tabcolsep}{-4pt}
     \begin{tabular}{|cl|c|c|c|c|c|c|c|}
        \hline \hline
           {\tiny  1a. Execução}      &   & \multicolumn{7}{c|}{Atributos}                                               \\ \cline{3-9} 
       \multicolumn{1}{|l}{}                            &   & A    & B & C & D & E & F & G \\ \hline
        \multicolumn{1}{|c|}{}                           & 1 & 91.4 & 94.2   & 58.5      & 80.0 & 74.2 & 55.7   & 60.0   \\ \cline{2-9} 
        \multicolumn{1}{|c|}{}                           & 2 & 98.5 & 98.5   & 50.0      & 90.0 & 88.5 & 41.4  & 90.0  \\ \cline{2-9} 
        \multicolumn{1}{|c|}{\multirow{-3}{*}{Clusters}} & 3 & 92.8 & 95.7   & 80.0      & 88.5 & 97.1 & 55.7  & 77.1  \\ \hline
      \end{tabular}
    %}
 &
 %\hspace{1cm} %altera o espaçamento entre as tabelas
 
   
 %\scalebox{0.5}{%
   \small\addtolength{\tabcolsep}{-4pt}
   \begin{tabular}{|cl|c|c|c|c|c|c|c|}
        \hline \hline
         {\tiny  2a. Execução} &   & \multicolumn{7}{c|}{Atributos}                                               \\ \cline{3-9} 
       \multicolumn{1}{|l}{}                            &   & A    & B & C & D & E & F & G \\ \hline
        \multicolumn{1}{|c|}{}                           & 1 &  91.4 & 94.2   & 62.8      & 78.5 & 81.4 & 61.4   & 57.1   \\ \cline{2-9} 
        \multicolumn{1}{|c|}{}                           & 2 & 98.5 & 98.5   & 54.2      & 90.0 & 88.5 & 40.0  & 90.0  \\ \cline{2-9} 
        \multicolumn{1}{|c|}{\multirow{-3}{*}{Clusters}} & 3 & 92.8 & 95.7   & 80.0      & 88.5 & 97.1 & 60.0  & 77.1  \\ \hline
      \end{tabular}
  %}
  \\  [8ex]
 %\hspace{1cm} %altera o espaçamento entre as tabelas
 %\rule{0}{50}
 
 %\scalebox{0.5}{%
   \small\addtolength{\tabcolsep}{-4pt}
   \begin{tabular}{|cl|c|c|c|c|c|c|c|}
        \hline \hline
          {\tiny  3a. Execução}   &   & \multicolumn{7}{c|}{Atributos}                                               \\ \cline{3-9} 
       \multicolumn{1}{|l}{}                            &   & A    & B & C & D & E & F & G \\ \hline
        \multicolumn{1}{|c|}{}                           & 1 & 93.8 & 93.6   & 61.8      & 83.2 & 89.2 & 53.2   & 71.0   \\ \cline{2-9} 
        \multicolumn{1}{|c|}{}                           & 2 & 98.2 & 98.3   & 61.9      & 93.0 & 90.5 & 25.2  & 90.1  \\ \cline{2-9} 
        \multicolumn{1}{|c|}{\multirow{-3}{*}{Clusters}} & 3 & 95.5 & 96.3   & 82.4      & 90.9 & 97.7 & 59.3  & 77.0  \\ \hline
   \end{tabular}
    
    &
    
       \small\addtolength{\tabcolsep}{-4pt}
   \begin{tabular}{|cl|c|c|c|c|c|c|c|}
        \hline \hline
         {\tiny 4a. Execução}       &   & \multicolumn{7}{c|}{Atributos}                                               \\ \cline{3-9} 
       \multicolumn{1}{|l}{}                            &   & A    & B & C & D & E & F & G \\ \hline
        \multicolumn{1}{|c|}{}                           & 1 & 92.8 & 94.2   & 60.0      & 80.0 & 84.2 & 64.2   & 60.0   \\ \cline{2-9} 
        \multicolumn{1}{|c|}{}                           & 2 & 98.5 & 98.5   & 47.1      & 91.4 & 90.0 & 42.8  & 88.5  \\ \cline{2-9} 
        \multicolumn{1}{|c|}{\multirow{-3}{*}{Clusters}} & 3 & 91.4 & 95.7   & 80.0      & 88.5 & 97.1 & 55.7  & 77.1  \\ \hline
   \end{tabular}
   \\
 
 \end{tabular}
 \label{tab:execucoes:seed:cart}
\end{table}

O resultado do algoritmo CART na base de dados \textbf{Seeds} tem como rótulos: 
\begin{itemize}[noitemsep]
 \item ${r_{c_1}=\{ (area, ]12.78 \sim 16.14]), (perimetro, ]13.73 \sim 15.18]) \} }$
 \item ${r_{c_2}=\{ (area, ]16.14 \sim 21.18]), (perimetro, ]15.18 \sim 17.25]) \} }$
 \item ${r_{c_3}=\{ (perimetro, [12.41 \sim 13.73]),  (wkernet, [2.63 \sim 3.049]) \} }$
\end{itemize}


\section{Iris - Identificação de Tipos de Plantas}


Essa base de dados utilizada neste trabalho, extraída do UCI\footnote{https://archive.ics.uci.edu/ml/} Machine Learning, também já é uma base conhecida em outras pesquisas\footnote{\cite{Lopes}, \cite{kotsiantis2005logitboost}, \cite{Filho2015} e outros}. Possui 150 registros de amostra de plantas com um total de 4 atributos  definindo as características das plantas e mais 1 atributo classe. Este último atributo classifica o tipo de planta em 3 tipos, segundo \cite{runkler2012} :
\begin{itemize}[noitemsep]
 \item 50 elementos da classe Iris-setosa ;
 \item 50 elementos da classe Iris-versicolour;
 \item 50 elementos da classe Iris-virginica.
\end{itemize}
Os atributos correspondentes são comprimento da sepala - SL, largura da sepala - SW, comprimento da pétala - PL e
largura da pétala - PW. Através dessas características há uma classificação para dizer qual tipo de planta.

Foi aplicado na configuração de implementação o método EFD\footnote{sessão \ref{cap:refTeor:subsec:efd}} de discretização com divisão em três faixas, R = 3 para todos os atributos, e inserido o valor de variação ${V=3\%}$. Mais uma vez, o valor ${V}$ é subjetivo do pesquisador e influenciado pelos valores de correlação dos atributos nos grupos, tabela \ref{tab:execucoes:iris:nb}.

Seguindo a análise, semelhante da base de dados anterior, será realizado testes utilizando dois algoritmos\footnote{sessões \ref{cap:resultados:ssec:seed:nb},\ref{cap:resultados:ssec:seed:cart}} e cada resultado será exibido em tabelas. Portando as colunas são formadas por \textbf{Clusters},\textbf{Rótulos}, \textbf{Relevância} e \textbf{Elem fora da Faixa} representando os valores que não estão dentro da faixa escolhida como rótulo. Também foi posto em tabelas o resultado das correlações entre os atributos de cada grupo, servindo de informação para decisão do valor de ${V}$. E também apresentado os resultados de outras iterações de cada algoritmo, para mostrar o comportamento dos atributos entre eles no grupo.

\subsection{Naive Bayes} \label{cap:resultados:ssec:iris:nb}

Segue a tabela \ref{tab:rot:iris:nb} com os resultados da rotulação após a aplicação do algoritmo. Com essa base de dados nota-se que no cluster 1 houve um acerto de 100\% da rotulação. O cluster 2 obteve os mesmo atributos do cluster 1, mas as faixas de valores são diferentes. E no cluster 3 somente um atributo foi selecionado, petalwidth, conseguindo ter uma relevância entre os outros atributos de 90\%.

\begin{table}[!h]
\centering
\caption{Resultado da aplicação do algoritmo Naive Bayes}
\label{tab:rot:iris:nb}
\begin{tabular}{llcrc} \hline
 
\multicolumn{1}{c}{\cellcolor[HTML]{FFFFFF}} & \multicolumn{2}{c}{Rótulos}                & \multicolumn{1}{r}{}               & \\ \cline{2-3}
Cluster                                      & Atributos      & \multicolumn{1}{c}{Faixa} & \multicolumn{1}{c}{Relevância(\%)} & Elem fora da Faixa\\ \hline \hline
                                             & petallength    & [ 1.0 $\sim$  3.7 ]       & 100\%                               & 0 \\  
\multirow{-2}{*}{1}                          & petalwidth     & [ 0.1 $\sim$  1.0 ]       & 100\%                               & 0 \\  \hline
                                             & petallength    & ] 3.7 $\sim$  5.1 ]       & 84\%                               & 7\\ 
\multirow{-2}{*}{2}                          & petalwidth     & ] 1.0 $\sim$  1.7 ]       & 82\%                               & 8\\  \hline
3                                            & petalwidth     & ] 1.7 $\sim$  2.5 ]       & 90\%                               & 5\\ \hline \hline
\end{tabular}
\end{table}

Os valore na coluna de relevância não podem ser analisados isoladamente. Para isso  a tabela \ref{tab:execucoes:iris:nb} possui os valores de todos os atributos no momento que ele são classes. Os valores são em porcentagem para melhor análise do grau de relacionamento entre os outros atributos.

\begin{table}[!h]
\caption{Resultado de ${4(quatro)}$ execuções do algoritmo Naive Bayes; Legenda dos Atributos: (SL)sepallength,(SW)sepalwidth,(PL)petallength,(PW)petalwidth}
 \begin{tabular}{ll}
%\rule{0}{50}

  
   %\scalebox{0.5}{%
   \small\addtolength{\tabcolsep}{-1pt}
     \begin{tabular}{|cl|c|c|c|c|}
        \hline \hline
                  1a. Execução   &   & \multicolumn{4}{c|}{Atributos}                                               \\ \cline{3-6} 
       \multicolumn{1}{|l}{}                             &   & SL   & SW     & PL    & PW      \\ \hline
        \multicolumn{1}{|c|}{}                           & 1 & 80   & 68     & \textbf{100}   & \textbf{100}       \\ \cline{2-6} 
        \multicolumn{1}{|c|}{}                           & 2 & 72 & 76   & \textbf{84}  & \textbf{82}     \\ \cline{2-6} 
        \multicolumn{1}{|c|}{\multirow{-3}{*}{Clusters}} & 3 & 76 & 74   & 68  & \textbf{90}     \\ \hline
      \end{tabular}
    %}
 &
 %\hspace{1cm} %altera o espaçamento entre as tabelas
 
   
 %\scalebox{0.5}{%
  \small\addtolength{\tabcolsep}{-1pt}
     \begin{tabular}{|cl|c|c|c|c|}
        \hline \hline
         2a. Execução         &   & \multicolumn{4}{c|}{Atributos}                                               \\ \cline{3-6} 
       \multicolumn{1}{|l}{}                             &   & SL   & SW     & PL    & PW      \\ \hline
        \multicolumn{1}{|c|}{}                           & 1 & 80 & 68   & 100 &  100      \\ \cline{2-6} 
        \multicolumn{1}{|c|}{}                           & 2 & 72 & 76   & 88  &    84  \\ \cline{2-6} 
        \multicolumn{1}{|c|}{\multirow{-3}{*}{Clusters}} & 3 & 70 & 74   & 70  &  90    \\ \hline
      \end{tabular}
  %}
  \\  [8ex]
 %\hspace{1cm} %altera o espaçamento entre as tabelas
 %\rule{0}{50}
 
 %\scalebox{0.5}{%
   \small\addtolength{\tabcolsep}{-1pt}
     \begin{tabular}{|cl|c|c|c|c|}
        \hline \hline
         3a. Execução           &   & \multicolumn{4}{c|}{Atributos}                                               \\ \cline{3-6} 
       \multicolumn{1}{|l}{}                             &   & SL   & SW     & PL    & PW      \\ \hline
        \multicolumn{1}{|c|}{}                           & 1 & 80 & 68   & 100  & 100       \\ \cline{2-6} 
        \multicolumn{1}{|c|}{}                           & 2 & 72 & 74   & 84  &  84    \\ \cline{2-6} 
        \multicolumn{1}{|c|}{\multirow{-3}{*}{Clusters}} & 3 & 74 & 74   & 68  &   90   \\ \hline
      \end{tabular}
    
    &
    
 \small\addtolength{\tabcolsep}{-1pt}
     \begin{tabular}{|cl|c|c|c|c|}
        \hline \hline
        4a. Execução      &   & \multicolumn{4}{c|}{Atributos}                                               \\ \cline{3-6} 
       \multicolumn{1}{|l}{}                             &   & SL   & SW     & PL    & PW      \\ \hline
        \multicolumn{1}{|c|}{}                           & 1 & 80 & 68   & 100  &   100     \\ \cline{2-6} 
        \multicolumn{1}{|c|}{}                           & 2 & 72 & 74   & 86  &   82   \\ \cline{2-6} 
        \multicolumn{1}{|c|}{\multirow{-3}{*}{Clusters}} & 3 & 70 & 74   & 70  & 92     \\ \hline
      \end{tabular}
   \\
 
 \end{tabular}
 \label{tab:execucoes:iris:nb}
\end{table}

Nessa tabela \ref{tab:execucoes:iris:nb} foram inseridas quatro tabelas com os resultados de cada execução. Foi escolhida na tabela \ref{tab:execucoes:iris:nb} a 1a. execução para montar a tabela de rótulos, tabela \ref{tab:rot:iris:nb}. A partir dela o pesquisador poderá arbitrá sobre o valor de ${V}$ para melhor adaptá-lo a base.
Das várias execuções expostas na tabela \ref{tab:execucoes:iris:nb}, percebe-se que não há muita diferença entre esses valores em cada execução. Isso mostra um padrão de valores de acordo com a base. No caso da 1a. execução os valores escolhidos como rótulo estão destacados em cada cluster.

Se a tabela escolhida fosse a da 4a. execução, os valores de rótulos seriam modificados, em virtude da diferença do atributo de maior valor com a variável ${V}$. O atributo de maior valor, no cluster 2, é o PL com 86\%. Então a diferença desse valor com ${V=3}$ chega em 83\%. Tanto SL, SW e PW possuem valores inferiores a 83\%, com isso o atributo escolhido para compor o rótulo  no cluster 2 seria só o PL, diferente das outras tabelas que são o PL e PW.

Os rótulos com o algoritmo CART na base de dados \textbf{Iris} são dados abaixo:
\begin{itemize}[noitemsep]
 \item ${r_{c_1}=\{ (petallength, [ 1.0 \sim 3.7]), (petalwidth,[ 0.1 \sim 1.0 ] ) \} }$  
 \item ${r_{c_2}=\{ (petallength, ] 3.7 \sim 5.1]), (petalwidth,] 1.0 \sim 1.7 ] )\} }$
 \item ${r_{c_3}=\{ (petalwidth, ] 1.7 \sim 2.5 ]) \} }$
\end{itemize}

\subsection{CART} \label{cap:resultados:ssec:iris:cart}

A aplicação do algoritmo CART na base de dados Iris gerou a tabela \ref{tab:rot:iris:cart} como resultado, e ao examinar pode-se observar uma semelhante com a sessão anterior \ref{cap:resultados:ssec:iris:nb} onde foi aplicado o Naive Bayes. 

\begin{table}[!h]
\centering
\caption{Resultado da aplicação do algoritmo CART}
\label{tab:rot:iris:cart}
\begin{tabular}{llcrc} \hline
 
\multicolumn{1}{c}{\cellcolor[HTML]{FFFFFF}} & \multicolumn{2}{c}{Rótulos}                & \multicolumn{1}{r}{}               & \\ \cline{2-3}
Cluster                                      & Atributos      & \multicolumn{1}{c}{Faixa} & \multicolumn{1}{c}{Relevância(\%)} & Elem fora da Faixa\\ \hline \hline
                                             & petallength    & [ 1.0 $\sim$  3.7 ]       & 100\%                               & 0 \\  
\multirow{-2}{*}{1}                          & petalwidth     & [ 0.1 $\sim$  1.0 ]       & 100\%                               & 0 \\  \hline
                                             & petallength    & ] 3.7 $\sim$  5.1 ]       & 88\%                               & 7\\ 
\multirow{-2}{*}{2}                          & petalwidth     & ] 1.0 $\sim$  1.7 ]       & 90\%                               & 8\\  \hline
3                                            & petalwidth     & ] 1.7 $\sim$  2.5 ]       & 90\%                               & 5\\ \hline \hline
\end{tabular}
\end{table}

Ao observar a tabela, \ref{tab:rot:iris:cart}, percebe-se que o resultado de rotulação é idêntico ao do algoritmo anterior, mas na coluna de \textbf{Relevência} existe uma diferença no cluster 2, contudo essa diferença não chega  a modificar a rotulação desta base de dados.
Iris 
Fazendo uma análise dessa diferença, na coluna de \textbf{Relevância}, valor esse, adquirido conforme a 1a. Execução da tabela \ref{tab:execucoes:iris:cart}, e já sabendo que o a escolha do rótulo tem a influência do valor de relevância, e também do valor de ${V}$. É constatado que os valores podem até ser maiores, nessa caso aqui específico, e mesmo assim a rotulação seria a mesma. Então mesmo alterando o valor de ${V}$ até 12 , os rótulos continuariam os mesmo.

Com o valor de ${V=3}$ ou até com ${V=12}$, como exemplo, e analisando as execuções da tabela \ref{tab:execucoes:iris:cart}, pode-se perceber que a diferença dos números dos rótulos são altos em relação aos outros atributos. Então a influência dos atributos escolhidos são bem fortes em relação aos que não são atributos. Com estes resultados percebe-se que os atributos são bem correlacionado com as classes, gerando uma boa distinção dos rótulos.  


\begin{table}[!h]
\caption{Resultado de ${4(quatro)}$ iterações do algoritmo CART; Legenda dos Atributos: (SL)sepallength,(SW)sepalwidth,(PL)petallength,(PW)petalwidth}
 \begin{tabular}{ll}
%\rule{0}{50}

  
   %\scalebox{0.5}{%
   \small\addtolength{\tabcolsep}{-1pt}
     \begin{tabular}{|cl|c|c|c|c|}
        \hline \hline
                  1a. Execução   &   & \multicolumn{4}{c|}{Atributos}                                               \\ \cline{3-6} 
       \multicolumn{1}{|l}{}                             &   & SL   & SW     & PL    & PW      \\ \hline
        \multicolumn{1}{|c|}{}                           & 1 & 80   & 68     & \textbf{100}   & \textbf{100}       \\ \cline{2-6} 
        \multicolumn{1}{|c|}{}                           & 2 & 74 & 76   & \textbf{88}  & \textbf{90}     \\ \cline{2-6} 
        \multicolumn{1}{|c|}{\multirow{-3}{*}{Clusters}} & 3 & 68 & 68   & 74  & \textbf{90}     \\ \hline
      \end{tabular}
    %}
 &
 %\hspace{1cm} %altera o espaçamento entre as tabelas
 
   
 %\scalebox{0.5}{%
  \small\addtolength{\tabcolsep}{-1pt}
     \begin{tabular}{|cl|c|c|c|c|}
        \hline \hline
         2a. Execução         &   & \multicolumn{4}{c|}{Atributos}                                               \\ \cline{3-6} 
       \multicolumn{1}{|l}{}                             &   & SL   & SW     & PL    & PW      \\ \hline
        \multicolumn{1}{|c|}{}                           & 1 & 80 & 68   & 100 &  100      \\ \cline{2-6} 
        \multicolumn{1}{|c|}{}                           & 2 & 74 & 76   & 88  &    90  \\ \cline{2-6} 
        \multicolumn{1}{|c|}{\multirow{-3}{*}{Clusters}} & 3 & 70 & 70   & 74  &  90    \\ \hline
      \end{tabular}
  %}
  \\  [8ex]
 %\hspace{1cm} %altera o espaçamento entre as tabelas
 %\rule{0}{50}
 
 %\scalebox{0.5}{%
   \small\addtolength{\tabcolsep}{-1pt}
     \begin{tabular}{|cl|c|c|c|c|}
        \hline \hline
         3a. Execução           &   & \multicolumn{4}{c|}{Atributos}                                               \\ \cline{3-6} 
       \multicolumn{1}{|l}{}                             &   & SL   & SW     & PL    & PW      \\ \hline
        \multicolumn{1}{|c|}{}                           & 1 & 80 & 68   & 100  & 100       \\ \cline{2-6} 
        \multicolumn{1}{|c|}{}                           & 2 & 74 & 76   & 86  &  90    \\ \cline{2-6} 
        \multicolumn{1}{|c|}{\multirow{-3}{*}{Clusters}} & 3 & 70 & 66   & 78  &   90   \\ \hline
      \end{tabular}
    
    &
    
 \small\addtolength{\tabcolsep}{-1pt}
     \begin{tabular}{|cl|c|c|c|c|}
        \hline \hline
        4a. Execução      &   & \multicolumn{4}{c|}{Atributos}                                               \\ \cline{3-6} 
       \multicolumn{1}{|l}{}                             &   & SL   & SW     & PL    & PW      \\ \hline
        \multicolumn{1}{|c|}{}                           & 1 & 80 & 68   & 100  &   100     \\ \cline{2-6} 
        \multicolumn{1}{|c|}{}                           & 2 & 72 & 74   & 86  &   90   \\ \cline{2-6} 
        \multicolumn{1}{|c|}{\multirow{-3}{*}{Clusters}} & 3 & 68 & 66   & 78  & 90     \\ \hline
      \end{tabular}
   \\
 
 \end{tabular}
 \label{tab:execucoes:iris:cart}
\end{table}


Segue abaixo os rótulos na base de dados \textbf{Iris} aplicado no algoritmo CART:
\begin{itemize}[noitemsep]
 \item ${r_{c_1}=\{ (petallength, [ 1.0 \sim 3.7]), (petalwidth,[ 0.1 \sim 1.0 ] ) \} }$  
 \item ${r_{c_2}=\{ (petallength, ] 3.7 \sim 5.1]), (petalwidth,] 1.0 \sim 1.7 ] )\} }$
 \item ${r_{c_3}=\{ (petalwidth, ] 1.7 \sim 2.5 ]) \} }$
\end{itemize}

  
 
%\end{table}

%\section{Glass - Identificação de Tipos de Vidro}
