\chapter{Resultados}\label{cap:resultados}

Os resultados obtidos aqui neste capítulo foram referentes a aplicação do método de rotulação em ${3(três)}$ bases de dados distintas. Um dos primeiros passos na análise de aprendizagem de máquina é quando o analista prepara os dados para poder utilizar  um método de aprendizagem apropriado. 

Então  a escolha da base de dados também tem influência direta em bons resultados. E sabendo disso a escolha dos conjuntos de dados utilizados nesta pesquisa foi por conta delas apresentarem características diferentes, e também por serem conhecidas, a análise acaba ficando mais clara servindo de amostras a outras base.

\section{Implementação}

Para conseguir gerar os resultados aqui escritos foram feitas implementações utilizando a ferramenta MATLAB \footnote{http://www.mathworks.com/products/matlab/}, onde junto  a ela é possível utilizar suas funções de aprendizado de máquina já prontas. MATLAB possui uma linguagem técnica, e de fácil implementação por já possuir uma gama de funções\footnote{versão: R2016a(9.0.0.341360); 64-bit (glnxa64)} preparadas para aprendizado de máquina. Por esses motivos essa ferramenta foi escolhida para colocar em prática essa pesquisa.

Foram realizados vários testes com o intuito de tentar otimizar resultados e poder comparálos a outras pesquisas já escritas. Seguindo essa linha foi determinado a escolha de ${3(três)}$ bases de dados já conhecidas, onde na implementação de cada uma delas surgiu algumas alterações, dependendo da base, na variável(V), quantidade de faixas(R) e método de discretização. 

\section{Seeds - Identificação de Tipos de Semente}

\section{Iris - Identificação de Tipos de Plantas}

\section{Glass - Identificação de Tipos de Vidro}


glass
