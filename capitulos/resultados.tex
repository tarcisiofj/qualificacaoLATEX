\chapter{Resultados}\label{cap:resultados}

Foram realizados testes com algumas bases de dados da UCI Machine Learning\footnote{http://archive.ics.uci.edu/ml/}, um repositório de dados a serviço da comunidade de aprendizado de máquina. Criado por estudantes de pós-graduação na UC Irvine em 1987, essas bases são utilizadas por estudantes mas e por educadores e pesquisadores como fonte primária de aplicações de aprendizado de máquina. 

* lembrar que as bases são essas (iris, ...) por serem já estudadas, e de prévio conhecimento, onde a partir de estudos em bases assim pode-se ter resultados conclusivos e expandir para outras bases. É como se testase com base conhecida para depois pegar uma nova.


As bases de dados foram escolhidas não só por critério comparativo de outros trabalhos que também já as utilizaram servindo  de referência para os resultados, como também, um cuidado de só escolher bases que estão classificadas, uma vez que esta pesquisa trabalhará com os clusters já formados e não na criação de grupos.

A divisão deste capítulo iniciará por uma explanação da implementação do trabalho explicando as ferramentas  utilizadas no desenvolvimento, e quais configurações  de algumas variáveis. Cada seção refere-se a uma base de dados utilizada, sendo esta, dividida em subseções para os algoritmos: Naive Bayes, CART e KNN.

\section{Implementação}\label{cap:resultados:sec:implement}

Para  gerar os resultados aqui escritos foram realizadas implementações utilizando a ferramenta MATLAB\footnote{http://www.mathworks.com/products/matlab/ ; versão: R2016a(9.0.0.341360); 64-bit (glnxa64)}, sendo possível utilizar suas funções de aprendizado de máquina já implementadas na \textit{Statistics and Machine Learning Toolbox }. Por aprensentar  linguagem técnica e funcões já prontas direcionada para aprendizado de máquina essa ferramenta foi escolhida para colocar em prática essa pesquisa.

Ao longo da pesquisa foram realizados vários testes, porém, houveram alterações de algumas variáveis e métodos de discretização, sempre com o objetivo de obter os melhores resultados. As variáveis e métodos alterados são: variância ``V'', número de faixas ``R'', e métodos de discretização ``R'' (EWD,EFD). 

Como dito na subseção \ref{cap:ferramentas:ssec:algsuper} a variação ${V}$ existe para evitar a ambiguidade dos rótulos, ou seja, quando rótulos apresentarem os mesmos resultados: atributo e faixa de valor. Além de evitar a ambiguidade dos rótulos a variável ${V}$ pode ser utilizada também para selecionar mais de um atributo para ser o rótulo do cluster.

%A utilização da variação ${V}$ para escolha de rótulos acontece após uma análise da tabela de correlação dos atributos (seção \ref{cap:ferramentas:sec:tecnica}), a exemplo da tabela \ref{tab:matrelevancia:seeds:nb}. E existindo valores muito próximos em relação a outros (cabe a uma análise se necessário), poderá utilizar esses atributos também como rótulos para melhor definir o cluster. A variável ${V}$ pode ser configurada com um valor que possa abranger esses atributos que possuam valores de porcentagens próximos ao do atributo de maior valor na tabela (mais relevante). O valor de ${V}$ é subjetivo e sua adição é condicionada a análise da aplicação do algoritmo na bases de dados.

%É importante ressaltar a criação da tabela de correlação de atributos (Matriz de Atributos Importantes). Essa tabela é implementada conforme a técnica de correlação entre atributos com algoritmos supervisionados, seção \ref{cap:ferramentas:ssec:algsuper}, onde cada célula da tabela é preenchida através da execução de um algoritmo supervisionado. Estas execuções são realizadas em todos os atributos da cada cluster existente na base de dados.

%Após a tabela preenchida, o atributo rótulo será selecionado a partir do maior valor em relação aos outros atributos do grupo, que é representado pela linha da tabela (matriz de atributos). Também poderá ser selecionado como rótulo os atributos que possuam o valor entre a diferença de ${V}$ com o atributo de maior valor (mais relevante). Por exemplo, se o valor de ${V=5\%}$ e o atributo de maior valor é \textbf{95\%}, então todos os atributos que possuírem o valor a partir de \textbf{90\%} serão considerados rótulos também.

%a cada iteração do algoritmo é preenchida uma célula da tabela com o valor de correlação do primeiro atributo até o último atributo de um cluster, e depois iniciado o mesmo procedimento a outro cluster até não haver mais clusters. Após a tabela estar montada o atributo rótulo será selecionado a partir do maior valor em relação aos outros atributos do grupo, e caso seja necessário também é selecionado como rótulo os atributos que possuam o valor entre a diferença de ${V}$ com o atributo de maior valor (mais relevante). 


%A cada base de dados descritas nas seções seguintes, são configuradas algumas variáveis, método de discretização e implementado dois algoritmos de aprendizado supervisionado com paradigmas diferentes para fazer rotulação. Cada algoritmo terá como resultado o rótulo por cluster de dados.


\section{Seeds - Identificação de Tipos de Semente}
Essa base é composta por sete  atributos definindo suas características e mais um atributo classe  responsável por identificar os tipos de sementes \cite{Charytanowicz2010}. Em seus atributos  seus valores são todos contínuos e não existem valores em branco,  possuindo um total de 210 registros classificados em três categorias:
\begin{itemize}[noitemsep]
 \item 70 elementos do tipo Kama;
 \item 70 elementos do tipo Rosa;
 \item 70 elementos do tipo Canadian.
\end{itemize}
Para classificar as sementes, como Kama, Rosa e Canadian foi utilizada uma técnica de raio X, que é relativamente mais barata que outras técnicas de imagem, como microscopia ou tecnologia a laser. O material foi colhido de campos experimentais, explorados no Instituo de Agrofísica da Academia Polonês de Ciências em Lublin.

Como já mencionado neste capítulo, na seção \ref{cap:resultados:sec:implement}, antes de executar o algoritmo algumas configuração são necessárias. A primeira, é a configuração do método de discretização para o tipo EFD, e a segunda é a divisão dos valores dos atributos em faixas, com ${R=3}$ para todos os atributos. Tanto a discretização como o valor do número de faixas foram testados e escolhidos para alcançar melhores resultados.

%e também o valor de variação ${V=0\%}$,  não obstante, esta variável ${V}$ só assumirar valor maior que zero após análise dos resultados caso haja ambiguidade.


\subsection{Naive Bayes} \label{cap:resultados:ssec:seed:nb}
% Please add the following required packages to your document preamble:
% \usepackage{multirow}
% \usepackage[table,xcdraw]{xcolor}
% If you use beamer only pass "xcolor=table" option, i.e. \documentclss[xcolor=table]{beamer}


\begin{table}[!h]
\centering
\caption{Resultado da rotulação com o algoritmo Naive Bayes}
\label{tab:rot:seeds:nb}
\scalebox{0.8}{
\begin{tabular}{llcrcc}
\hline \hline
\multicolumn{1}{c}{\cellcolor[HTML]{FFFFFF}} & \multicolumn{2}{c}{Rótulos}                & \multicolumn{1}{r}{}               & & \\ \cline{2-3}
Cluster                                      & Atributos      & \multicolumn{1}{c}{Faixa} & \multicolumn{1}{c}{Relevância(\%)} & Fora da Faixa & Acurácia Cluster(\%) \\ \hline \hline
1                                            & area           & ] 12.78 $\sim$  16.14 ]   & 92\%                               & 14 & 80\% \\  \hline
2                                            & area           & ] 16.14 $\sim$  21.18 ]   & 95\%                               & 6 & 91,4\%\\ \hline
%\multirow{-2}{*}{2}                          & lkernel        & ] 5.826 $\sim$  6.675 ]   & 92\%                               & 6\\  \hline
3                                            & perimetro      & [ 12.41 $\sim$  13.73 ]   & 95\%                               & 5 & 92,8\% \\ \hline \hline
\end{tabular}
}
\end{table}


Na tabela \ref{tab:rot:seeds:nb} é apresentado os resultados de rotulação do algoritmo Naive Bayes. Essa tabela é composta por colunas que informam os \textbf{Clusters}, \textbf{Rótulos}  que integram \textbf{Atributo} e sua \textbf{Faixa} de valor, além da coluna \textbf{Relevância} exibida em porcentagem, bem como as  colunas \textbf{Fora da Faixa} e \textbf{Acurácia Cluster} que exibem a quantidade de elementos que não estão dentro da faixa designada pelo do rótulo encontrado, e a acurácia do cada cluster respectivamente.

A coluna \textbf{Relevância}  demonstra o maior valor entre os atributos de cada  cluster, e caso esses valores sejam ambíguos, serão exibidos na coluna todos estes  atributos. Para ter maior clareza na escolha desses atributos foi inseridaa a tabela \ref{tab:matrelevancia:seeds:nb} que exibem os valores de correlação entre eles.

Já na coluna, \textbf{Fora da Faixa}, tem a função de exibir, em números, a quantidade de valores que não estão participando da faixa definida pelo rótulo. Através de experimentos percebeu-se o mérito de apresentar em números a quantidade de elementos que não estão sendo representados pelo rótulo gerando mais realidade as informações, ao invés de exibir em porcentagem.

Na última coluna, \textbf{Acurácia Cluster}, apresenta em porcentagem o grau de acerto, por cluster, dos registros que são representados pelo rótulo. Foi possível expor estas informações visto que cada cluster já apresenta a quantidade e quais registros fazem parte de cada cluster.

Analisando a coluna Rótulos da tabela \ref{tab:rot:seeds:nb}, nota-se que o atributo \textbf{area} aparece tanto no  cluster 1 como também no cluster 2. A rotulação de dados envolve não só o atributo mais relevante, como também, a faixa de valores que mais se repete dentro do atributo. Nesse caso pode-se observar que na coluna \textbf{Atributo} o atributo \textbf{area} se repete entre os cluster  1 e 2, mas no cluster 1 a faixa de valores difere do cluster 2, sendo considerados rótulos distintos.


%Caso os resultados gerados na tabela \ref{tab:matrelevancia:seeds:nb} expusessem clusters com rótulos ambíguos, poderia ser utilizado a variação de ${V}$. Quando houver ambiguidade dos rótulos, a seleção dos atributos que compõem os rótulos, acontecerá da diferença da variável ${V}$ em relação ao atributo de maior relevência do cluster. Caso essa variável tenha o valor alterado, os rótulos dos clusters poderão sofrer mudanças, pois poderia aumentar  ou diminuir o número de atributos dos rótulos, dependendo do valor inserido em ${V}$. Através da tabela \ref{tab:matrelevancia:seeds:nb} é possível analisar todos os valores de relevância gerados para os atributos e analisar qual valor pode-se inserir em ${V}$ para montar o rótulo.

%Para exemplificar a utilização da variável ${V}$ pode-se utilizar como exemplo os dados do cluster 2 da tabela \ref{tab:matrelevancia:seeds:nb} e adotando ${V=3\%}$.  Neste exemplo não só o atributo de maior relevância, \textbf{area} com ${95.7\%}$ seria escolhido como rótulo, mas também o atributo \textbf{lkernel} com valor ${92.8\%}$, pois a diferença entre o valor de \textbf{area} com ${V}$ resultaria em ${92.7\%}$. Através dessa diferença todos os atributos que estivessem na faixa de ${92.7\%}$ a  ${95.7\%}$ seriam selecionados como atributos do rótulo.

\begin{table}[!h]
    
    \caption{Resultado da Correlação dos atributos pelo Naive Bayes; Legenda dos Atributos: (A)area, (B)perimetro, (C)compacteness, (D)Lkernel, (E)Wkernel, (F)asymetry, (G)lkgroove}    
    \centering
   \small\addtolength{\tabcolsep}{+2pt}
    \begin{tabular}{|cl|c|c|c|c|c|c|c|}
        \hline \hline
                                &   & \multicolumn{7}{c|}{Atributos}          \\ \cline{3-9} 
        \multicolumn{1}{|l}{}                            &   & A    & B & C & D & E & F & G \\ \hline
        \multicolumn{1}{|c|}{}                           & 1 & 92.8 & 87.1   & 50.0      & 75.7 & 85.7 & 60.0   & 65.7   \\ \cline{2-9} 
        \multicolumn{1}{|c|}{}                           & 2 & 95.7 & 91.4   & 47.1      & 92.8 & 90.0 & 28.5  & 85.7  \\ \cline{2-9} 
        \multicolumn{1}{|c|}{\multirow{-3}{*}{Clusters}} & 3 & 91.4 & 95.7   & 71.4      & 85.7 & 91.4 & 64.2  & 58.5  \\ \hline
    \end{tabular}
    \label{tab:matrelevancia:seeds:nb} 
\end{table}

A tabela \ref{tab:matrelevancia:seeds:nb} é um exemplo da matriz de atributos importantes gerada pela técnica de correlação entre atributos. É formada por clusters representado pelas linhas, e atributos representado por colunas, onde esses valores são representados em porcentagem (\%). 

Essa tabela é fruto da aplicação do Naive Bayes na base de dados \textbf{Seeds}, e a partir dela é retirado o(s) atributo(s) rótulo(s). Uma análise pode ser feita através desses dados e ajudar a definir um valor para a variável ${V}$ caso necessário. Percebe-se que algumas características são mais bem correlacionadas que  outras, através dos valores mais altos indicando o grau de relacionamento entre os atributos após a aplicação do algoritmo. 


\begin{table}[!h]
\caption{Resultado de 4 (quatro) execuções do algoritmo Naive Bayes; Legenda dos Atributos: (A)area, (B)perimetro, (C)compacteness, (D)Lkernel, (E)Wkernel, (F)asymetry, (G)lkgroove}
 %\begin{tabular}{ll}
%\rule{0}{50}
\centering
   \subfloat[1a. Execução]{ \label{tab:execucoes:seed:nb:1exec}
   %\scalebox{0.5}{%
   \small\addtolength{\tabcolsep}{-2pt} 
     \begin{tabular}{|cl|c|c|c|c|c|c|c|}
        \hline \hline
            {\tiny 1a. Execução}     &   & \multicolumn{7}{c|}{Atributos}                                               \\ \cline{3-9} 
       \multicolumn{1}{|l}{}                            &   & A    & B & C & D & E & F & G \\ \hline
        \multicolumn{1}{|c|}{}                           & 1 & 92.8 & 87.1   & 48.5      & 77.1 & 82.8 & 57.1   & 65.7   \\ \cline{2-9} 
        \multicolumn{1}{|c|}{}                           & 2 & 94.2 & 90.0   & 45.7      & 92.8 & 90.0 & 38.5  & 87.1  \\ \cline{2-9} 
        \multicolumn{1}{|c|}{\multirow{-3}{*}{Clusters}} & 3 & 91.4 & 95.7   & 72.8      & 85.7 & 91.4 & 64.2  & 60.0  \\ \hline
      \end{tabular}
    %}
    }
 %&
 %\hspace{1cm} %altera o espaçamento entre as tabelas
 
   \subfloat[2a. Execução]{ \label{tab:execucoes:seed:nb:2exec}
 %\scalebox{0.5}{%
   \small\addtolength{\tabcolsep}{-2pt}
   \begin{tabular}{|cl|c|c|c|c|c|c|c|}
        \hline \hline
             {\tiny 2a. Execução }       &   & \multicolumn{7}{c|}{Atributos}                                               \\ \cline{3-9} 
       \multicolumn{1}{|l}{}                            &   & A    & B & C & D & E & F & G \\ \hline
        \multicolumn{1}{|c|}{}                           & 1 & 92.8 & 87.1   & 47.1      & 77.1 & 87.1 & 60.0   & 65.7   \\ \cline{2-9} 
        \multicolumn{1}{|c|}{}                           & 2 & 94.2 & 90.0   & 47.1      & 92.8 & 91.4 & 32.8  & 87.1  \\ \cline{2-9} 
        \multicolumn{1}{|c|}{\multirow{-3}{*}{Clusters}} & 3 & 91.4 & 95.7   & 72.8      & 85.7 & 92.8 & 64.2  & 60.0  \\ \hline
      \end{tabular}
  %}
  %\\  [8ex]
    }
    
    \subfloat[3a. Execução]{ \label{tab:execucoes:seed:nb:3exec}
 %\scalebox{0.5}{%
   \small\addtolength{\tabcolsep}{-2pt} 
   \begin{tabular}{|cl|c|c|c|c|c|c|c|}
        \hline \hline
          {\tiny 3a. Execução}     &   & \multicolumn{7}{c|}{Atributos}                                               \\ \cline{3-9} 
       \multicolumn{1}{|l}{}                            &   & A    & B & C & D & E & F & G \\ \hline
        \multicolumn{1}{|c|}{}                           & 1 & 94.2 & 85.7   & 48.5      & 77.1 & 82.8 & 61.4   & 65.7   \\ \cline{2-9} 
        \multicolumn{1}{|c|}{}                           & 2 & 92.8 & 90.0   & 50.0      & 92.8 & 90.0 & 32.8  & 87.1  \\ \cline{2-9} 
        \multicolumn{1}{|c|}{\multirow{-3}{*}{Clusters}} & 3 & 91.4 & 95.7   & 72.8      & 85.7 & 92.8 & 64.2  & 60.0  \\ \hline
   \end{tabular}
    }
    %&
    
    \subfloat[4a. Execução]{  \label{tab:execucoes:seed:nb:4exec}
       \small\addtolength{\tabcolsep}{-2pt}
        \begin{tabular}{|cl|c|c|c|c|c|c|c|}
        \hline \hline
            {\tiny 4a. Execução }   &   & \multicolumn{7}{c|}{Atributos}                                               \\ \cline{3-9} 
       \multicolumn{1}{|l}{}                            &   & A    & B & C & D & E & F & G \\ \hline
        \multicolumn{1}{|c|}{}                           & 1 & 91.4 & 88.5   & 54.2      & 75.7 & 85.7 & 62.8   & 61.4   \\ \cline{2-9} 
        \multicolumn{1}{|c|}{}                           & 2 & 95.7 & 90.0   & 50.0      & 92.8 & 90.0 & 38.5  & 85.7  \\ \cline{2-9} 
        \multicolumn{1}{|c|}{\multirow{-3}{*}{Clusters}} & 3 & 91.4 & 95.7   & 72.8      & 85.7 & 94.2 & 64.2  & 57.1  \\ \hline
        \end{tabular}
   %\\
    }
    
 %\end{tabular}
 \label{tab:execucoes:seed:nb}
\end{table}

%% TESTE DE CRIAÇÃO DE TABELA
% \begin{table}[!h]
% \caption{Resultado de 4 (quatro) execuções do algoritmo Naive Bayes; Legenda dos Atributos: (A)area, (B)perimetro, (C)compacteness, (D)Lkernel, (E)Wkernel, (F)asymetry, (G)lkgroove}
%  \centering
%  \subfloat[Tabela A]{
%  
%  \small\addtolength{\tabcolsep}{-5pt}
%  \begin{tabular}{|cl|c|c|c|c|c|c|c|}
%         \hline \hline
%             {\tiny 1a. Execução}     &   & \multicolumn{7}{c|}{Atributos}                                               \\ \cline{3-9} 
%        \multicolumn{1}{|l}{}                            &   & A    & B & C & D & E & F & G \\ \hline
%         \multicolumn{1}{|c|}{}                           & 1 & 92.8 & 87.1   & 48.5      & 77.1 & 82.8 & 57.1   & 65.7   \\ \cline{2-9} 
%         \multicolumn{1}{|c|}{}                           & 2 & 94.2 & 90.0   & 45.7      & 92.8 & 90.0 & 38.5  & 87.1  \\ \cline{2-9} 
%         \multicolumn{1}{|c|}{\multirow{-3}{*}{Clusters}} & 3 & 91.4 & 95.7   & 72.8      & 85.7 & 91.4 & 64.2  & 60.0  \\ \hline
%       \end{tabular}
%  }
%  
%  %\hspace{3cm}
%  
%  \subfloat[Tabela B]{
%  
%  \small\addtolength{\tabcolsep}{-5pt}
%  \begin{tabular}{|cl|c|c|c|c|c|c|c|}
%         \hline \hline
%             {\tiny 1a. Execução}     &   & \multicolumn{7}{c|}{Atributos}                                               \\ \cline{3-9} 
%        \multicolumn{1}{|l}{}                            &   & A    & B & C & D & E & F & G \\ \hline
%         \multicolumn{1}{|c|}{}                           & 1 & 92.8 & 87.1   & 48.5      & 77.1 & 82.8 & 57.1   & 65.7   \\ \cline{2-9} 
%         \multicolumn{1}{|c|}{}                           & 2 & 94.2 & 90.0   & 45.7      & 92.8 & 90.0 & 38.5  & 87.1  \\ \cline{2-9} 
%         \multicolumn{1}{|c|}{\multirow{-3}{*}{Clusters}} & 3 & 91.4 & 95.7   & 72.8      & 85.7 & 91.4 & 64.2  & 60.0  \\ \hline
%       \end{tabular}
%  }
%  
% \end{table}



Para provar empiricamente os resultados, na tabela \ref{tab:execucoes:seed:nb} é exposto  4 (quatro) execuções do Algoritmo Naive Bayes. Pode-se constatar que mesmo havendo algumas alterações nos valores dos atributos em cada execução, a correlação entre os atributos não oferece muita alteração. Como exemplo, o atributo \textbf{area} nos clusters 1 e 2, possuem o melhor grau de correlacionamento em seus grupos, mesmo nas quatro execuções, como mostrado na tabela \ref{tab:execucoes:seed:nb}.

A informação passada pela tabela \ref{tab:execucoes:seed:nb} tem a intensão de mostrar para o analista que estiver aplicando a rotulação de dados, como seus atributos se comportam em cada cluster, ao utilizar  o método de rotulação de dados.

Segue abaixo o resultado do algoritmo Naive Bayes na base de dados \textbf{Seeds} com seus rótulos: 
\begin{itemize}[noitemsep]
 \item ${r_{c_1}=\{ (area, ]12.78 \sim 16.14]) \} }$  
 \item ${r_{c_2}=\{ (area, ]16.14 \sim 21.18]) \} }$
 \item ${r_{c_3}=\{ (perimetro, [12.41 \sim 13.73])\} }$
\end{itemize}


\subsection{CART}\label{cap:resultados:ssec:seed:cart}

Na tabela \ref{tab:rot:seeds:cart}, que segue o mesmo modelo da tabela \ref{tab:rot:seeds:nb}, tem-se o resultado da aplicação do algoritmo supervisionado na base Seeds. O CART é utilizado pela toolbox do MATLAB como algoritmo de classificação de árvore de decisão. O que se pretende fazer é seguir a pesquisa e testar a base de dados com um paradigma  diferente para fazer rotulação nos clusters.

\begin{table}[!h]
\centering
\caption{Resultado da aplicação do algoritmo CART}
\label{tab:rot:seeds:cart}
\scalebox{0.8}{
\begin{tabular}{llcrcc}\hline \hline

\multicolumn{1}{c}{\cellcolor[HTML]{FFFFFF}} & \multicolumn{2}{c}{Rótulos}                      & \multicolumn{1}{r}{}            \\ \cline{2-3}
Cluster                                      & Atributos      & \multicolumn{1}{c}{Faixa}       & \multicolumn{1}{c}{Relevância(\%)} & Fora da Faixa & Acurácia Cluster(\%)\\ \hline \hline
%                                             & area           & ] 12.78 $\sim$  16.14 ]         & 91\%          & 14 \\  
1                                            & perimetro      & [ 13.73 $\sim$ 15.18 ]          & 94\%          & 14 & 80\%\\ \hline
                                             & area           & ] 16.14 $\sim$  21.18 ]          & 98\%         & 6 & \\ 
\multirow{-2}{*}{2}                          & perimetro      & ] 15.18 $\sim$  17.25 ]          & 98\%         & 7 & \multirow{-2}{*}{90\%} \\  \hline
%                                             & perimetro      & [ 12.41 $\sim$  13.73 ]         & 95\%          & 5 \\
3                                            & wkernel        & [ 2.63 $\sim$  3.049 ]         & 97\%           & 9 & 87,1\%\\ \hline \hline
\end{tabular}}
\end{table}

Já na tabela \ref{tab:execucoes:seed:cart} são exibidas algumas execuções do algoritmo CART na base de dados. O mesmo comportamento entre execuções pode ser visto no algoritmo de paradigma estatístico, subseção \ref{cap:resultados:ssec:seed:nb}, realizado nessa pesquisa. O  comportamento de ambos os algoritmos foram bem semelhantes, como também, seus valores nas execuções que não se alteraram muito a cada iteração.

% 
% \begin{table}[!h]
%     
%     \caption{Resultado da Correlação dos atributos pelo CART; Legenda dos Atributos: (A)area, (B)perimetro, (C)compacteness, (D)Lkernel, (E)Wkernel, (F)asymetry, (G)lkgroove}    
%     \centering
%    \small\addtolength{\tabcolsep}{1pt}
%     \begin{tabular}{|cl|c|c|c|c|c|c|c|}
%         \hline \hline
%                                 &   & \multicolumn{7}{c|}{Atributos}          \\ \cline{3-9} 
%         \multicolumn{1}{|l}{}                            &   & A    & B & C & D & E & F & G \\ \hline
%         \multicolumn{1}{|c|}{}                           & 1 & 91.4 & 94.2   & 58.5      & 80.0 & 81.4 & 61.4   & 61.4   \\ \cline{2-9} 
%         \multicolumn{1}{|c|}{}                           & 2 & 98.5 & 98.5   & 51.4      & 90.0 & 88.5 & 42.8  & 88.5  \\ \cline{2-9} 
%         \multicolumn{1}{|c|}{\multirow{-3}{*}{Clusters}} & 3 & 92.7 & 95.7   & 80.0      & 88.5 & 97.1 & 58.5  & 78.5  \\ \hline
%     \end{tabular}
%     \label{tab:matrelevancia:seeds:cart} 
% \end{table}

%%%%%%%%%%%%%%%%%%%%%%%%%%%%%%%%%%%%%%%%%%%%%%%%%%%%%%%%%%%%%%%%%%%%%

\begin{table}[!h]
\caption{Resultado de 4 (quatro) execuções do algoritmo Naive Bayes; Legenda dos Atributos: (A)area, (B)perimetro, (C)compacteness, (D)Lkernel, (E)Wkernel, (F)asymetry, (G)lkgroove}
 %\begin{tabular}{ll}
%\rule{0}{50}
\centering
   \subfloat[1a. Execução]{ \label{tab:execucoes:seed:cart:1exec}
   %\scalebox{0.5}{%
   \small\addtolength{\tabcolsep}{-2pt} 
    \begin{tabular}{|cl|c|c|c|c|c|c|c|}
        \hline \hline
           {\tiny  1a. Execução}      &   & \multicolumn{7}{c|}{Atributos}                                               \\ \cline{3-9} 
       \multicolumn{1}{|l}{}                            &   & A    & B & C & D & E & F & G \\ \hline
        \multicolumn{1}{|c|}{}                           & 1 & 91.4 & 94.2   & 58.5      & 80.0 & 74.2 & 55.7   & 60.0   \\ \cline{2-9} 
        \multicolumn{1}{|c|}{}                           & 2 & 98.5 & 98.5   & 50.0      & 90.0 & 88.5 & 41.4  & 90.0  \\ \cline{2-9} 
        \multicolumn{1}{|c|}{\multirow{-3}{*}{Clusters}} & 3 & 92.8 & 95.7   & 80.0      & 88.5 & 97.1 & 55.7  & 77.1  \\ \hline
      \end{tabular}
    %}
    }
 %&
 %\hspace{1cm} %altera o espaçamento entre as tabelas
 
   \subfloat[2a. Execução]{ \label{tab:execucoes:seed:cart:2exec}
 %\scalebox{0.5}{%
   \small\addtolength{\tabcolsep}{-2pt}
  \begin{tabular}{|cl|c|c|c|c|c|c|c|}
        \hline \hline
         {\tiny  2a. Execução} &   & \multicolumn{7}{c|}{Atributos}                                               \\ \cline{3-9} 
       \multicolumn{1}{|l}{}                            &   & A    & B & C & D & E & F & G \\ \hline
        \multicolumn{1}{|c|}{}                           & 1 &  91.4 & 94.2   & 62.8      & 78.5 & 81.4 & 61.4   & 57.1   \\ \cline{2-9} 
        \multicolumn{1}{|c|}{}                           & 2 & 98.5 & 98.5   & 54.2      & 90.0 & 88.5 & 40.0  & 90.0  \\ \cline{2-9} 
        \multicolumn{1}{|c|}{\multirow{-3}{*}{Clusters}} & 3 & 92.8 & 95.7   & 80.0      & 88.5 & 97.1 & 60.0  & 77.1  \\ \hline
      \end{tabular}
  %}
  %\\  [8ex]
    }
    
    \subfloat[3a. Execução]{ \label{tab:execucoes:seed:cart:3exec}
 %\scalebox{0.5}{%
   \small\addtolength{\tabcolsep}{-2pt} 
   \begin{tabular}{|cl|c|c|c|c|c|c|c|}
        \hline \hline
          {\tiny  3a. Execução}   &   & \multicolumn{7}{c|}{Atributos}                                               \\ \cline{3-9} 
       \multicolumn{1}{|l}{}                            &   & A    & B & C & D & E & F & G \\ \hline
        \multicolumn{1}{|c|}{}                           & 1 & 93.8 & 93.6   & 61.8      & 83.2 & 89.2 & 53.2   & 71.0   \\ \cline{2-9} 
        \multicolumn{1}{|c|}{}                           & 2 & 98.2 & 98.3   & 61.9      & 93.0 & 90.5 & 25.2  & 90.1  \\ \cline{2-9} 
        \multicolumn{1}{|c|}{\multirow{-3}{*}{Clusters}} & 3 & 95.5 & 96.3   & 82.4      & 90.9 & 97.7 & 59.3  & 77.0  \\ \hline
   \end{tabular}
    }
    %&
    
    \subfloat[4a. Execução]{  \label{tab:execucoes:seed:cart:4exec}
       \small\addtolength{\tabcolsep}{-2pt}
   \begin{tabular}{|cl|c|c|c|c|c|c|c|}
        \hline \hline
         {\tiny 4a. Execução}       &   & \multicolumn{7}{c|}{Atributos}                                               \\ \cline{3-9} 
       \multicolumn{1}{|l}{}                            &   & A    & B & C & D & E & F & G \\ \hline
        \multicolumn{1}{|c|}{}                           & 1 & 92.8 & 94.2   & 60.0      & 80.0 & 84.2 & 64.2   & 60.0   \\ \cline{2-9} 
        \multicolumn{1}{|c|}{}                           & 2 & 98.5 & 98.5   & 47.1      & 91.4 & 90.0 & 42.8  & 88.5  \\ \cline{2-9} 
        \multicolumn{1}{|c|}{\multirow{-3}{*}{Clusters}} & 3 & 91.4 & 95.7   & 80.0      & 88.5 & 97.1 & 55.7  & 77.1  \\ \hline
   \end{tabular}
   %\\
    }
    
 %\end{tabular}
 \label{tab:execucoes:seed:cart}
\end{table}
%%%%%%%%%%%%%%%%%%%%%%%%%%%%%%%%%%%%%%%%%%%%%%%%%%%%%%%%%%%%%%%%%%%%

% 
% \begin{table}[!h]
% \caption{Resultado de 4 (quatro) iterações do algoritmo CART; Legenda dos Atributos: (A)area, (B)perimetro, (C)compacteness, (D)Lkernel, (E)Wkernel, (F)asymetry, (G)lkgroove}
%  \begin{tabular}{ll}
% %\rule{0}{50}
% 
%   
%    %\scalebox{0.5}{%
%    \small\addtolength{\tabcolsep}{-5pt}
%      \begin{tabular}{|cl|c|c|c|c|c|c|c|}
%         \hline \hline
%            {\tiny  1a. Execução}      &   & \multicolumn{7}{c|}{Atributos}                                               \\ \cline{3-9} 
%        \multicolumn{1}{|l}{}                            &   & A    & B & C & D & E & F & G \\ \hline
%         \multicolumn{1}{|c|}{}                           & 1 & 91.4 & 94.2   & 58.5      & 80.0 & 74.2 & 55.7   & 60.0   \\ \cline{2-9} 
%         \multicolumn{1}{|c|}{}                           & 2 & 98.5 & 98.5   & 50.0      & 90.0 & 88.5 & 41.4  & 90.0  \\ \cline{2-9} 
%         \multicolumn{1}{|c|}{\multirow{-3}{*}{Clusters}} & 3 & 92.8 & 95.7   & 80.0      & 88.5 & 97.1 & 55.7  & 77.1  \\ \hline
%       \end{tabular}
%     %}
%  &
%  %\hspace{1cm} %altera o espaçamento entre as tabelas
%  
%    
%  %\scalebox{0.5}{%
%    \small\addtolength{\tabcolsep}{-5pt}
%    \begin{tabular}{|cl|c|c|c|c|c|c|c|}
%         \hline \hline
%          {\tiny  2a. Execução} &   & \multicolumn{7}{c|}{Atributos}                                               \\ \cline{3-9} 
%        \multicolumn{1}{|l}{}                            &   & A    & B & C & D & E & F & G \\ \hline
%         \multicolumn{1}{|c|}{}                           & 1 &  91.4 & 94.2   & 62.8      & 78.5 & 81.4 & 61.4   & 57.1   \\ \cline{2-9} 
%         \multicolumn{1}{|c|}{}                           & 2 & 98.5 & 98.5   & 54.2      & 90.0 & 88.5 & 40.0  & 90.0  \\ \cline{2-9} 
%         \multicolumn{1}{|c|}{\multirow{-3}{*}{Clusters}} & 3 & 92.8 & 95.7   & 80.0      & 88.5 & 97.1 & 60.0  & 77.1  \\ \hline
%       \end{tabular}
%   %}
%   \\  [8ex]
%  %\hspace{1cm} %altera o espaçamento entre as tabelas
%  %\rule{0}{50}
%  
%  %\scalebox{0.5}{%
%    \small\addtolength{\tabcolsep}{-5pt}
%    \begin{tabular}{|cl|c|c|c|c|c|c|c|}
%         \hline \hline
%           {\tiny  3a. Execução}   &   & \multicolumn{7}{c|}{Atributos}                                               \\ \cline{3-9} 
%        \multicolumn{1}{|l}{}                            &   & A    & B & C & D & E & F & G \\ \hline
%         \multicolumn{1}{|c|}{}                           & 1 & 93.8 & 93.6   & 61.8      & 83.2 & 89.2 & 53.2   & 71.0   \\ \cline{2-9} 
%         \multicolumn{1}{|c|}{}                           & 2 & 98.2 & 98.3   & 61.9      & 93.0 & 90.5 & 25.2  & 90.1  \\ \cline{2-9} 
%         \multicolumn{1}{|c|}{\multirow{-3}{*}{Clusters}} & 3 & 95.5 & 96.3   & 82.4      & 90.9 & 97.7 & 59.3  & 77.0  \\ \hline
%    \end{tabular}
%     
%     &
%     
%        \small\addtolength{\tabcolsep}{-5pt}
%    \begin{tabular}{|cl|c|c|c|c|c|c|c|}
%         \hline \hline
%          {\tiny 4a. Execução}       &   & \multicolumn{7}{c|}{Atributos}                                               \\ \cline{3-9} 
%        \multicolumn{1}{|l}{}                            &   & A    & B & C & D & E & F & G \\ \hline
%         \multicolumn{1}{|c|}{}                           & 1 & 92.8 & 94.2   & 60.0      & 80.0 & 84.2 & 64.2   & 60.0   \\ \cline{2-9} 
%         \multicolumn{1}{|c|}{}                           & 2 & 98.5 & 98.5   & 47.1      & 91.4 & 90.0 & 42.8  & 88.5  \\ \cline{2-9} 
%         \multicolumn{1}{|c|}{\multirow{-3}{*}{Clusters}} & 3 & 91.4 & 95.7   & 80.0      & 88.5 & 97.1 & 55.7  & 77.1  \\ \hline
%    \end{tabular}
%    \\
%  
%  \end{tabular}
%  \label{tab:execucoes:seed:cart}
% \end{table}

O resultado da rotulação utilizando o algoritmo CART na base de dados \textbf{Seeds} tem como rótulos: 
\begin{itemize}[noitemsep]
 \item ${r_{c_1}=\{ (perimetro, ]13.73 \sim 15.18]) \} }$
 \item ${r_{c_2}=\{ (area, ]16.14 \sim 21.18]), (perimetro, ]15.18 \sim 17.25]) \} }$
 \item ${r_{c_3}=\{ (wkernel, [2.63 \sim 3.049]) \} }$
\end{itemize}


\section{Iris - Identificação de Tipos de Plantas}


A base de dados \textbf{Iris}, também pertencente a UCI Machine Learning, é muito conhecida em outras pesquisas \citeonline{LOPES2014,Filho2015}, como também na literatura em reconhecimentos de padrões, por utilizar classes de plantas bem definidas. Contêm 3 classes de 50 instâncias cada, totalizando  150 registros de amostra de plantas. O atributo classe classifica o tipo de planta em 3 tipos \cite{FISHER1936}:

\begin{itemize}[noitemsep]
 \item 50 elementos da classe Iris-setosa ;
 \item 50 elementos da classe Iris-versicolour;
 \item 50 elementos da classe Iris-virginica.
\end{itemize}

Os atributos correspondentes são comprimento da sepala - SL, largura da sepala - SW, comprimento da pétala - PL e largura da pétala - PW. Através dessas características há uma classificação para dizer qual tipo de planta.

Para alcançar os resultados do algoritmo na base de dados, foram aplicadas algumas  configurações. Estas configurações foram o método de discretização, tipo EFD, seção \ref{cap:refTeor:subsec:efd}, e a divisão em três faixas de valores ${R = 3}$ para todos os atributos, e inserido o valor de variação ${V=0\%}$, tabela \ref{tab:execucoes:iris:nb}.

Seguindo a análise, semelhante da base de dados anterior, serão realizados testes utilizando os algoritmos, Naive Bayes e CART. Seus resultados serão exibidos em tabelas. Também foi posto nas tabelas \ref{tab:execucoes:iris:nb} e \ref{tab:execucoes:iris:cart} os resultados da técnica de correlações entre os atributos de cada grupo, servindo de informação para decisão do valor de ${V}$, caso fosse necessário. E também apresentado os resultados de outras iterações de cada algoritmo, para mostrar o comportamento dos atributos entre eles no grupo.

\subsection{Naive Bayes} \label{cap:resultados:ssec:iris:nb}

Através da tabela \ref{tab:rot:iris:nb} os resultados da rotulação são exibidos após a aplicação do algoritmo. Com essa base de dados nota-se que no cluster 1 houve um acerto de 100\% da rotulação. O cluster 2 e cluster 3 obtiveram rótulos distintos, cada um com com grau de relevância acima de 80\% em relação aos outros atributos de cada grupo.

\begin{table}[!h]
\centering
\caption{Resultado da aplicação do algoritmo Naive Bayes}
\label{tab:rot:iris:nb}
\scalebox{0.8}{
\begin{tabular}{llcrcc} \hline \hline
 
\multicolumn{1}{c}{\cellcolor[HTML]{FFFFFF}} & \multicolumn{2}{c}{Rótulos}                & \multicolumn{1}{r}{}               & \\ \cline{2-3}
Cluster                                      & Atributos      & \multicolumn{1}{c}{Faixa} & \multicolumn{1}{c}{Relevância(\%)} & Fora da Faixa & Acurácia Cluster(\%)\\ \hline \hline
                                             & petallength    & [ 1.0 $\sim$  3.7 ]       & 100\%                               & 0 & \\  
\multirow{-2}{*}{1}                          & petalwidth     & [ 0.1 $\sim$  1.0 ]       & 100\%                               & 0 & \multirow{-2}{*}{100\%} \\  \hline
2                                             & petallength    & ] 3.7 $\sim$  5.1 ]       & 84\%                               & 7 & 86\% \\ \hline
%\multirow{-2}{*}{2}                          & petalwidth     & ] 1.0 $\sim$  1.7 ]       & 82\%                               & 8\\  \hline
3                                            & petalwidth     & ] 1.7 $\sim$  2.5 ]       & 90\%                               & 5 & 90\% \\ \hline \hline
\end{tabular}}
\end{table}

A porcentagem representada na coluna de relevância não pode ser analisada isoladamente. Para isso  a tabela \ref{tab:execucoes:iris:nb} possui os valores de correlação de todos os atributos. Todos os números estão representados em porcentagem para melhor análise do grau de relacionamento entre os outros atributos.
% 
% \begin{table}[!h]
% \caption{Resultado (em \%) de ${4(quatro)}$ execuções do algoritmo Naive Bayes; Legenda dos Atributos: (SL)sepallength, (SW)sepalwidth, (PL)petallength, (PW)petalwidth}
%  \begin{tabular}{ll}
% %\rule{0}{50}
% 
%   
%    %\scalebox{0.5}{%
%    \small\addtolength{\tabcolsep}{-1pt}
%      \begin{tabular}{|cl|c|c|c|c|}
%         \hline \hline
%                   1a. Execução   &   & \multicolumn{4}{c|}{Atributos}                                               \\ \cline{3-6} 
%        \multicolumn{1}{|l}{}                             &   & SL   & SW     & PL    & PW      \\ \hline
%         \multicolumn{1}{|c|}{}                           & 1 & 80   & 68     & \textbf{100}   & \textbf{100}       \\ \cline{2-6} 
%         \multicolumn{1}{|c|}{}                           & 2 & 72 & 76   & \textbf{84}  & \textbf{82}     \\ \cline{2-6} 
%         \multicolumn{1}{|c|}{\multirow{-3}{*}{Clusters}} & 3 & 76 & 74   & 68  & \textbf{90}     \\ \hline
%       \end{tabular}
%     %}
%  &
%  %\hspace{1cm} %altera o espaçamento entre as tabelas
%  
%    
%  %\scalebox{0.5}{%
%   \small\addtolength{\tabcolsep}{-1pt}
%      \begin{tabular}{|cl|c|c|c|c|}
%         \hline \hline
%          2a. Execução         &   & \multicolumn{4}{c|}{Atributos}                                               \\ \cline{3-6} 
%        \multicolumn{1}{|l}{}                             &   & SL   & SW     & PL    & PW      \\ \hline
%         \multicolumn{1}{|c|}{}                           & 1 & 80 & 68   & 100 &  100      \\ \cline{2-6} 
%         \multicolumn{1}{|c|}{}                           & 2 & 72 & 76   & 88  &    84  \\ \cline{2-6} 
%         \multicolumn{1}{|c|}{\multirow{-3}{*}{Clusters}} & 3 & 70 & 74   & 70  &  90    \\ \hline
%       \end{tabular}
%   %}
%   \\  [8ex]
%  %\hspace{1cm} %altera o espaçamento entre as tabelas
%  %\rule{0}{50}
%  
%  %\scalebox{0.5}{%
%    \small\addtolength{\tabcolsep}{-1pt}
%      \begin{tabular}{|cl|c|c|c|c|}
%         \hline \hline
%          3a. Execução           &   & \multicolumn{4}{c|}{Atributos}                                               \\ \cline{3-6} 
%        \multicolumn{1}{|l}{}                             &   & SL   & SW     & PL    & PW      \\ \hline
%         \multicolumn{1}{|c|}{}                           & 1 & 80 & 68   & 100  & 100       \\ \cline{2-6} 
%         \multicolumn{1}{|c|}{}                           & 2 & 72 & 74   & 84  &  84    \\ \cline{2-6} 
%         \multicolumn{1}{|c|}{\multirow{-3}{*}{Clusters}} & 3 & 74 & 74   & 68  &   90   \\ \hline
%       \end{tabular}
%     
%     &
%     
%  \small\addtolength{\tabcolsep}{-1pt}
%      \begin{tabular}{|cl|c|c|c|c|}
%         \hline \hline
%         4a. Execução      &   & \multicolumn{4}{c|}{Atributos}                                               \\ \cline{3-6} 
%        \multicolumn{1}{|l}{}                             &   & SL   & SW     & PL    & PW      \\ \hline
%         \multicolumn{1}{|c|}{}                           & 1 & 80 & 68   & 100  &   100     \\ \cline{2-6} 
%         \multicolumn{1}{|c|}{}                           & 2 & 72 & 74   & 86  &   82   \\ \cline{2-6} 
%         \multicolumn{1}{|c|}{\multirow{-3}{*}{Clusters}} & 3 & 70 & 74   & 70  & 92     \\ \hline
%       \end{tabular}
%    \\
%  
%  \end{tabular}
%  \label{tab:execucoes:iris:nb}
% \end{table}



\begin{table}[!h]
\caption{Resultado (em \%) de 4 (quatro) execuções do algoritmo Naive Bayes; Legenda dos Atributos: (SL)sepallength, (SW)sepalwidth, (PL)petallength, (PW)petalwidth}
 %\begin{tabular}{ll}
%\rule{0}{50}
\centering
   \subfloat[1a. Execução]{ \label{tab:execucoes:iris:nb:1exec}
   %\scalebox{0.5}{%
   \small\addtolength{\tabcolsep}{-2pt} 
     \begin{tabular}{|cl|c|c|c|c|}
        \hline \hline
                  1a. Execução   &   & \multicolumn{4}{c|}{Atributos}                                               \\ \cline{3-6} 
       \multicolumn{1}{|l}{}                             &   & SL   & SW     & PL    & PW      \\ \hline
        \multicolumn{1}{|c|}{}                           & 1 & 80   & 68     & \textbf{100}   & \textbf{100}       \\ \cline{2-6} 
        \multicolumn{1}{|c|}{}                           & 2 & 72 & 76   & \textbf{84}  & 82     \\ \cline{2-6} 
        \multicolumn{1}{|c|}{\multirow{-3}{*}{Clusters}} & 3 & 76 & 74   & 68  & \textbf{90}     \\ \hline
      \end{tabular}
    %}
    }
 %&
 %\hspace{1cm} %altera o espaçamento entre as tabelas
 \hspace{1cm}
   \subfloat[2a. Execução]{ \label{tab:execucoes:iris:nb:2exec}
 %\scalebox{0.5}{%
   \small\addtolength{\tabcolsep}{-2pt}
      \begin{tabular}{|cl|c|c|c|c|}
        \hline \hline
         2a. Execução         &   & \multicolumn{4}{c|}{Atributos}                                               \\ \cline{3-6} 
       \multicolumn{1}{|l}{}                             &   & SL   & SW     & PL    & PW      \\ \hline
        \multicolumn{1}{|c|}{}                           & 1 & 80 & 68   & 100 &  100      \\ \cline{2-6} 
        \multicolumn{1}{|c|}{}                           & 2 & 72 & 76   & 88  &    84  \\ \cline{2-6} 
        \multicolumn{1}{|c|}{\multirow{-3}{*}{Clusters}} & 3 & 70 & 74   & 70  &  90    \\ \hline
      \end{tabular}
  %}
  %\\  [8ex]
    }
    
    \subfloat[3a. Execução]{ \label{tab:execucoes:iris:nb:3exec}
 %\scalebox{0.5}{%
   \small\addtolength{\tabcolsep}{-2pt} 
     \begin{tabular}{|cl|c|c|c|c|}
        \hline \hline
         3a. Execução           &   & \multicolumn{4}{c|}{Atributos}                                               \\ \cline{3-6} 
       \multicolumn{1}{|l}{}                             &   & SL   & SW     & PL    & PW      \\ \hline
        \multicolumn{1}{|c|}{}                           & 1 & 80 & 68   & 100  & 100       \\ \cline{2-6} 
        \multicolumn{1}{|c|}{}                           & 2 & 72 & 74   & 84  &  84    \\ \cline{2-6} 
        \multicolumn{1}{|c|}{\multirow{-3}{*}{Clusters}} & 3 & 74 & 74   & 68  &   90   \\ \hline
      \end{tabular}
    }
    %&
    \hspace{1cm}
    \subfloat[4a. Execução]{  \label{tab:execucoes:iris:nb:4exec}
       \small\addtolength{\tabcolsep}{-2pt}
     \begin{tabular}{|cl|c|c|c|c|}
        \hline \hline
        4a. Execução      &   & \multicolumn{4}{c|}{Atributos}                                               \\ \cline{3-6} 
       \multicolumn{1}{|l}{}                             &   & SL   & SW     & PL    & PW      \\ \hline
        \multicolumn{1}{|c|}{}                           & 1 & 80 & 68   & 100  &   100     \\ \cline{2-6} 
        \multicolumn{1}{|c|}{}                           & 2 & 72 & 74   & 86  &   82   \\ \cline{2-6} 
        \multicolumn{1}{|c|}{\multirow{-3}{*}{Clusters}} & 3 & 70 & 74   & 70  & 92     \\ \hline
      \end{tabular}
   %\\
    }
    
 %\end{tabular}
 \label{tab:execucoes:iris:nb}
\end{table}

Na tabela \ref{tab:execucoes:iris:nb} foram inseridas quatro resultados de execuções do algoritmo. Foi escolhida na tabela \ref{tab:execucoes:iris:nb:1exec} a 1a. execução para montar a tabela de rótulos, tabela \ref{tab:rot:iris:nb}. A partir dessas execuções o pesquisador poderá arbitrá sobre o valor de ${V}$ para melhor adaptá-lo a base. Das várias execuções expostas na tabela \ref{tab:execucoes:iris:nb}, percebe-se que não há muita diferença entre os valores de cada execução. Isso mostra um padrão de valores de acordo com a base. No caso da 1a. execução (tabela \ref{tab:execucoes:iris:nb:1exec}) os valores escolhidos como rótulo estão destacados em cada cluster.

%Se na tabela \ref{tab:execucoes:iris:nb} fosse escolhida a 3a. execução, os valores de rótulos seriam modificados, em virtude dos valores mais altos serem iguais, fazendo que o rótulo assumisse dois atributos: PL e PW. Em análise do cluster 2 perecebe-se que os valores de PL e PW nas quatro execuções são bem próximos e até idênticos na terceira execução, como já dito anteriormente, então caso fosse necessário inserir um valor de variação ${V}$, um valor aceitável seria ${V=3}$. Desta maneira manteria os rótulos dos clusters 1 e 3 sem alteração, e um novo atributo seria incluído no cluster 2, assumindo o novo rótulo com dois atributos: PL e PW.

Os rótulos com o algoritmo Naive Bayes na base de dados \textbf{Iris} são dados abaixo:
\begin{itemize}[noitemsep]
 \item ${r_{c_1}=\{ (petallength, [ 1.0 \sim 3.7]), (petalwidth,[ 0.1 \sim 1.0 ] ) \} }$  
 \item ${r_{c_2}=\{ (petallength, ] 3.7 \sim 5.1]) \} }$
 \item ${r_{c_3}=\{ (petalwidth, ] 1.7 \sim 2.5 ]) \} }$
\end{itemize}

\subsection{CART} \label{cap:resultados:ssec:iris:cart}

A aplicação do algoritmo CART na base de dados \textbf{Iris} gerou a tabela \ref{tab:rot:iris:cart} como resultado, e ao examinar pode-se observar uma semelhança com a subseção anterior onde foi aplicado o Naive Bayes. 

\begin{table}[!h]
\centering
\caption{Resultado da aplicação do algoritmo CART}
\label{tab:rot:iris:cart}
\scalebox{0.8}{
\begin{tabular}{llcrcc} \hline \hline
 
\multicolumn{1}{c}{\cellcolor[HTML]{FFFFFF}} & \multicolumn{2}{c}{Rótulos}                & \multicolumn{1}{r}{}               & \\ \cline{2-3}
Cluster                                      & Atributos      & \multicolumn{1}{c}{Faixa} & \multicolumn{1}{c}{Relevância(\%)} & Fora da Faixa & Acurácia Cluster(\%)\\ \hline \hline
                                             & petallength    & [ 1.0 $\sim$  3.7 ]       & 100\%                               & 0 & \\  
\multirow{-2}{*}{1}                          & petalwidth     & [ 0.1 $\sim$  1.0 ]       & 100\%                               & 0 & \multirow{-2}{*}{100\%}\\  \hline
%                                             & petallength    & ] 3.7 $\sim$  5.1 ]       & 88\%                               & 7\\ 
2                                            & petalwidth     & ] 1.0 $\sim$  1.7 ]       & 90\%                               & 8 & 84\%\\  \hline
3                                            & petalwidth     & ] 1.7 $\sim$  2.5 ]       & 90\%                               & 5 & 90\%\\ \hline \hline
\end{tabular}}
\end{table}

Ao observar a tabela \ref{tab:rot:iris:cart} percebe-se que o resultado de rotulação no cluster 1 e 3 são idênticos ao do algoritmo apresentado anteriormente, mas  no cluster 2 o rótulo é diferenciado pelo atributo petalwidth que atinge valores mais altos em todas as execuções, como mostra a tabela \ref{tab:execucoes:iris:cart}.

%%%%%%%%%%%%%%%%%%%%%%%%%%%%%%%%%%%%%%%%%%%%%%%%%%%%%%%%%%%%%%%%%%%%%%%%%%%%%%%%%%%%%%%%%%%%%%%%%%%%%%%%%%%%%%%%%%%%%%%%%%%%%%%%%%%%%%%%%%%%%%%%%%%%%%%%%%%%%%%%



\begin{table}[!h]
\caption{Resultado de 4 (quatro) iterações do algoritmo CART; Legenda dos Atributos: (SL)sepallength,(SW)sepalwidth,(PL)petallength,(PW)petalwidth}
 %\begin{tabular}{ll}
%\rule{0}{50}
\centering
   \subfloat[1a. Execução]{ \label{tab:execucoes:iris:cart:1exec}
   %\scalebox{0.5}{%
   \small\addtolength{\tabcolsep}{-2pt} 
     \begin{tabular}{|cl|c|c|c|c|}
        \hline \hline
                  1a. Execução   &   & \multicolumn{4}{c|}{Atributos}                                               \\ \cline{3-6} 
       \multicolumn{1}{|l}{}                             &   & SL   & SW     & PL    & PW      \\ \hline
        \multicolumn{1}{|c|}{}                           & 1 & 80   & 68     & \textbf{100}   & \textbf{100}       \\ \cline{2-6} 
        \multicolumn{1}{|c|}{}                           & 2 & 74 & 76   & 88  & \textbf{90}     \\ \cline{2-6} 
        \multicolumn{1}{|c|}{\multirow{-3}{*}{Clusters}} & 3 & 68 & 68   & 74  & \textbf{90}     \\ \hline
      \end{tabular}
    %}
    }
 %&
 %\hspace{1cm} %altera o espaçamento entre as tabelas
 \hspace{1cm}
   \subfloat[2a. Execução]{ \label{tab:execucoes:iris:cart:2exec}
 %\scalebox{0.5}{%
   \small\addtolength{\tabcolsep}{-2pt}
     \begin{tabular}{|cl|c|c|c|c|}
        \hline \hline
         2a. Execução         &   & \multicolumn{4}{c|}{Atributos}                                               \\ \cline{3-6} 
       \multicolumn{1}{|l}{}                             &   & SL   & SW     & PL    & PW      \\ \hline
        \multicolumn{1}{|c|}{}                           & 1 & 80 & 68   & 100 &  100      \\ \cline{2-6} 
        \multicolumn{1}{|c|}{}                           & 2 & 74 & 76   & 88  &    90  \\ \cline{2-6} 
        \multicolumn{1}{|c|}{\multirow{-3}{*}{Clusters}} & 3 & 70 & 70   & 74  &  90    \\ \hline
      \end{tabular}
  %}
  %\\  [8ex]
    }
    
    \subfloat[3a. Execução]{ \label{tab:execucoes:iris:cart:3exec}
 %\scalebox{0.5}{%
   \small\addtolength{\tabcolsep}{-2pt} 
     \begin{tabular}{|cl|c|c|c|c|}
        \hline \hline
         3a. Execução           &   & \multicolumn{4}{c|}{Atributos}                                               \\ \cline{3-6} 
       \multicolumn{1}{|l}{}                             &   & SL   & SW     & PL    & PW      \\ \hline
        \multicolumn{1}{|c|}{}                           & 1 & 80 & 68   & 100  & 100       \\ \cline{2-6} 
        \multicolumn{1}{|c|}{}                           & 2 & 72 & 74   & 84  &  84    \\ \cline{2-6} 
        \multicolumn{1}{|c|}{\multirow{-3}{*}{Clusters}} & 3 & 74 & 74   & 68  &   90   \\ \hline
      \end{tabular}
    }
    %&
    \hspace{1cm}
    \subfloat[4a. Execução]{  \label{tab:execucoes:iris:cart:4exec}
       \small\addtolength{\tabcolsep}{-2pt}
     \begin{tabular}{|cl|c|c|c|c|}
        \hline \hline
        4a. Execução      &   & \multicolumn{4}{c|}{Atributos}                                               \\ \cline{3-6} 
       \multicolumn{1}{|l}{}                             &   & SL   & SW     & PL    & PW      \\ \hline
        \multicolumn{1}{|c|}{}                           & 1 & 80 & 68   & 100  &   100     \\ \cline{2-6} 
        \multicolumn{1}{|c|}{}                           & 2 & 72 & 74   & 86  &   90   \\ \cline{2-6} 
        \multicolumn{1}{|c|}{\multirow{-3}{*}{Clusters}} & 3 & 68 & 66   & 78  & 90     \\ \hline
      \end{tabular}
   %\\
    }
    
 %\end{tabular}
 \label{tab:execucoes:iris:cart}
\end{table}

%%%%%%%%%%%%%%%%%%%%%%%%%%%%%%%%%%%%%%%%%%%%%%%%%%%%%%%%%%%%%%%%%%%%%%%%%%%%%%%%%%%%%%%%%%%%%%%%%%%%%%%%%%%%%%%%%%%%%%%%%%%%%%%%%%%%%%%%%%%%%%%%%%%%%%%%%%%%%%%%
% 
% \begin{table}[!h]
% \caption{Resultado de ${4(quatro)}$ iterações do algoritmo CART; Legenda dos Atributos: (SL)sepallength,(SW)sepalwidth,(PL)petallength,(PW)petalwidth}
%  \begin{tabular}{ll}
% %\rule{0}{50}
% 
%   
%    %\scalebox{0.5}{%
%    \small\addtolength{\tabcolsep}{-1pt}
%      \begin{tabular}{|cl|c|c|c|c|}
%         \hline \hline
%                   1a. Execução   &   & \multicolumn{4}{c|}{Atributos}                                               \\ \cline{3-6} 
%        \multicolumn{1}{|l}{}                             &   & SL   & SW     & PL    & PW      \\ \hline
%         \multicolumn{1}{|c|}{}                           & 1 & 80   & 68     & \textbf{100}   & \textbf{100}       \\ \cline{2-6} 
%         \multicolumn{1}{|c|}{}                           & 2 & 74 & 76   & \textbf{88}  & \textbf{90}     \\ \cline{2-6} 
%         \multicolumn{1}{|c|}{\multirow{-3}{*}{Clusters}} & 3 & 68 & 68   & 74  & \textbf{90}     \\ \hline
%       \end{tabular}
%     %}
%  &
%  %\hspace{1cm} %altera o espaçamento entre as tabelas
%  
%    
%  %\scalebox{0.5}{%
%   \small\addtolength{\tabcolsep}{-1pt}
%      \begin{tabular}{|cl|c|c|c|c|}
%         \hline \hline
%          2a. Execução         &   & \multicolumn{4}{c|}{Atributos}                                               \\ \cline{3-6} 
%        \multicolumn{1}{|l}{}                             &   & SL   & SW     & PL    & PW      \\ \hline
%         \multicolumn{1}{|c|}{}                           & 1 & 80 & 68   & 100 &  100      \\ \cline{2-6} 
%         \multicolumn{1}{|c|}{}                           & 2 & 74 & 76   & 88  &    90  \\ \cline{2-6} 
%         \multicolumn{1}{|c|}{\multirow{-3}{*}{Clusters}} & 3 & 70 & 70   & 74  &  90    \\ \hline
%       \end{tabular}
%   %}
%   \\  [8ex]
%  %\hspace{1cm} %altera o espaçamento entre as tabelas
%  %\rule{0}{50}
%  
%  %\scalebox{0.5}{%
%    \small\addtolength{\tabcolsep}{-1pt}
%      \begin{tabular}{|cl|c|c|c|c|}
%         \hline \hline
%          3a. Execução           &   & \multicolumn{4}{c|}{Atributos}                                               \\ \cline{3-6} 
%        \multicolumn{1}{|l}{}                             &   & SL   & SW     & PL    & PW      \\ \hline
%         \multicolumn{1}{|c|}{}                           & 1 & 80 & 68   & 100  & 100       \\ \cline{2-6} 
%         \multicolumn{1}{|c|}{}                           & 2 & 74 & 76   & 86  &  90    \\ \cline{2-6} 
%         \multicolumn{1}{|c|}{\multirow{-3}{*}{Clusters}} & 3 & 70 & 66   & 78  &   90   \\ \hline
%       \end{tabular}
%     
%     &
%     
%  \small\addtolength{\tabcolsep}{-1pt}
%      \begin{tabular}{|cl|c|c|c|c|}
%         \hline \hline
%         4a. Execução      &   & \multicolumn{4}{c|}{Atributos}                                               \\ \cline{3-6} 
%        \multicolumn{1}{|l}{}                             &   & SL   & SW     & PL    & PW      \\ \hline
%         \multicolumn{1}{|c|}{}                           & 1 & 80 & 68   & 100  &   100     \\ \cline{2-6} 
%         \multicolumn{1}{|c|}{}                           & 2 & 72 & 74   & 86  &   90   \\ \cline{2-6} 
%         \multicolumn{1}{|c|}{\multirow{-3}{*}{Clusters}} & 3 & 68 & 66   & 78  & 90     \\ \hline
%       \end{tabular}
%    \\
%  
%  \end{tabular}
%  \label{tab:execucoes:iris:cart}
% \end{table}


Segue abaixo os rótulos na base de dados \textbf{Iris} aplicado pelo algoritmo CART:
\begin{itemize}[noitemsep]
 \item ${r_{c_1}=\{ (petallength, [ 1.0 \sim 3.7]), (petalwidth,[ 0.1 \sim 1.0 ] ) \} }$  
 \item ${r_{c_2}=\{  (petalwidth,] 1.0 \sim 1.7 ] )\} }$
 \item ${r_{c_3}=\{ (petalwidth, ] 1.7 \sim 2.5 ]) \} }$
\end{itemize}


\section{Glass - Identificação de Tipos de Vidros}\label{cap:resultados:ssec:iris}

Essa base ficou conhecida por Vina Spiehler, Ph.D. da DABFT Diagnostic Products Corporation, onde conduzio pesquisas e testes de comparação em seu sistema baseado em regras determinando, se o tipo de vidro era temperado ou não. Institutos de investigação criminológica motivaram os estudos de classificação de tipos de vidros, porque em uma cena de crime, uma classificação de tipos de vidro corretamente identificada pode ser utilizada como prova, ajudando diretamente na investigação \cite{Evett:1989}.

Possui um total de 214 instancias, caracterizados por 9 atributos (RI, Na, Mg, Al, Si, K, Ca, Ba e Fe), sendo que o atributo \textbf{RI} indica o índice de refração, e quanto aos demais atributos são valores correspondentes a porcentagem do óxido.

Os tipos de vidro (atributo classe) foram divididos em 7 grupos distintos:
\begin{itemize}
 \item 1 janelas de construção - vidro temperado: 70 registros
 \item 2 janelas de construção - vidro não-temperado: 76 registros
 \item 3 janelas de veículos - vidro temperado: 17 registros
 \item 4 janelas de veículos - vidro não-temperado: 0 registro
 \item 5 recipientes: 13 registros
 \item 6 louças de mesa: 9 registros
 \item 7 lâmpadas: 29 registros 
\end{itemize}

Para execução dos algoritmos foram definidos a quantidade de faixas (${R}$) que serão divididos os valores dos atributos, qual o método de discretização e o valor de variação ${V}$ caso haja ambiguidade. Nos teste desenvolvidos nesta pesquisa os valores de referência foram, ${R=3}$ para o número de faixas, o método de discretização EWD e o valor  ${V=0\%}$.


\subsection{Naive Bayes} \label{cap:resultados:ssec:glass:nb}

Ao observar a tabela \ref{tab:rot:glass:nb} percebe-se que a coluna \textbf{Relevância} obteve porcentagens altas, ressaltando nos rótulos de cada grupo os atributos que mais bem se relacionaram. E em específico no  \textbf{cluster 5} atributo \textbf{Na}, o valor da coluna de \textbf{Relevância = 100\%}, mas na coluna, \textbf{Fora da Faixa}, apresentam 2(dois) elementos que não estão sendo representados pelo rótulo. 

Essa situação dita no parágrafo acima segue a Definição \ref{teo:resolucao}, mas é um exemplo prático que não aconteceu em outros testes das outras bases de dados, e por isso segue um esclarecimento. A definição é que cada rótulo específico é dado por um conjunto de pares de valores, tendo como saída um vetor com atributo e seu respectivo intervalo, ${ r_{ci}=\{ (a_1,[p_1,q_1]),...,(a_{m^{(c_i)}}, ]p_{m^{(c_i)}},q_{m^{(c_i)}}]) \} }$ capaz de melhor expressar o cluster ${c_i}$. Então caso a coluna \textbf{Relevância} seja igual a 100\%, isso não implica que todos os elementos tenham que estar dentro da faixa ${ p_{m^{(c_i)}} }$ (limite inferior) e ${ q_{m^{(c_i)}} }$ (limite superior), e sim, a maioria dos elementos, mostrando que o rótulo é capaz de melhor representar o cluster.

Além de apresentar dados desbalenceados o \textbf{Cluster 5} apresentado na tabela \ref{tab:rot:glass:nb} conta com o total de nove elementos, e entre estes, nenhum participa da 1a. faixa, dois estão na 2a. faixa e os restantes (sete) estão na 3a. faixa. Dessa maneira justifica-se o porquê dos dois elementos estarem de fora do rótulo, pois a faixa rótulo escolhida é a 3a. faixa, onde contém a maioria dos elementos, por conseguinte, escolhida para representar o rótulo.

\begin{table}[!h]
\centering
\caption{Resultado da aplicação do algoritmo Naive Bayes}
\label{tab:rot:glass:nb}
\scalebox{0.9}{
\begin{tabular}{llcrcc} 
\hline \hline
 
\multicolumn{1}{c}{\cellcolor[HTML]{FFFFFF}} & \multicolumn{2}{c}{Rótulos}                & \multicolumn{1}{r}{}               & \\ \cline{2-3}
Cluster                                      & Atributos      & \multicolumn{1}{c}{Faixa} & \multicolumn{1}{c}{Relevância(\%)} & Fora da Faixa & Acurácia Cluster(\%)\\ \hline \hline
                                             & Mg    & [ 2.245 $\sim$  4.490     ]       & 100\%                               & 0 & \\
                                             & K     & [ 0.0 $\sim$  1.5525      ]       & 100\%                               & 0 & \\  
\multirow{-3}{*}{1}                          & Ba    & [ 0.0 $\sim$  0.7875     ]       & 100\%                               & 0 & \multirow{-3}{*}{100\%}\\  \hline
%                                             & petallength    & ] 3.7 $\sim$  5.1 ]       & 88\%                               & 7\\ 
2                                            & K     & ] 0.0 $\sim$  1.5525 ]           & 100\%                               & 0 & 100\% \\  \hline
                                            & Mg     & ]  2.245 $\sim$  4.490  ]              & 100\%                               & 0 &\\ 
                                            & K     & ] 0.0 $\sim$  1.5525 ]               & 100\%                               & 0 &\\  
                                            & Ca     & ] 8.12 $\sim$  10.81 ]       & 100\%                               & 0 &\\ 
\multirow{-3}{*}{3}                          & Ba    & [ 0.0 $\sim$  0.7875     ]       & 100\%                               & 0 &  \multirow{-3}{*}{100\%} \\  \hline
                                             & Al    & [ 1.0925 $\sim$  1.895 ]       & 92\%                               & 4 & \\
                                             & K     & [ 0.0 $\sim$  1.5525      ]       & 92\%                               & 3 &\\  
\multirow{-3}{*}{4}                          & Ba    & [ 0.0 $\sim$  0.7875     ]       & 92\%                               & 1 & \multirow{-3}{*}{69,2\%} \\  \hline
                                            & Na    & [14.055 $\sim$ 17.38  ]           & 100\%                             & 2 &\\
                                            & K     & [ 0.0 $\sim$  1.5525      ]       & 100\%                               & 0 & \\  
                                            & Ba    & [ 0.0 $\sim$  0.7875     ]       & 100\%                               & 0 & \\   
\multirow{-4}{*}{5}   & Fe    & [ 0.0 $\sim$  0.1275     ]       & 100\%                               & 0 & \multirow{-4}{*}{77,7\%}\\  \hline
6                                            & Fe    & [ 0.0 $\sim$  0.1275     ]       & 100\%                               & 0 & 100\%\\  \hline

\hline

\end{tabular}
}
\end{table}

Os resultados da tabela \ref{tab:execucoes:glass:nb}, assim como nos resultados de bases anteriores, indicam uma sequência de execuções onde é possível observar o comportamento das variáveis que são escolhidas como rótulo. Nestes exemplos fica claro que não foi necessária a utilização de uma variação ${V}$ para a escolha dos rótulos, logo porque não houve ambiguidade entre eles. Por outro lado, quando testes utilizaram o outro método de discretização, EFD, retornaram rótulos ambíguos obrigando o uso da variação ${V}$. Em consequencia disto foi definindo o método de discretização EWD como padrão para a rotulação de dados.


\begin{table}[!h]
\caption{Resultado de 4 (quatro) execuções do algoritmo Naive Bayes.}
 %\begin{tabular}{ll}
%\rule{0}{50}
\centering
   \subfloat[1a. Execução]{ \label{tab:execucoes:glass:nb:1exec}
   %\scalebox{0.5}{%
   \small\addtolength{\tabcolsep}{-2pt} 
    \begin{tabular}{|cl|c|c|c|c|c|c|c|c|c|}
        \hline \hline
            {\tiny 1a. Exec}     &   & \multicolumn{9}{c|}{\tiny Atributos}                                               \\ \cline{3-11} 
       \multicolumn{1}{|l}{}                            &   & RI    & Na    & Mg  & Al   & Si   & K   & Ca   & Ba  & Fe             \\ \hline
        \multicolumn{1}{|c|}{}                           & 1 & 87.1 & 92.8  & 100 & 81.4 & 82.8 & 100 & 90.0 & 100 & 78.5 \\ \cline{2-11} 
        \multicolumn{1}{|c|}{}                           & 2 & 65.7 & 86.8  & 85.5& 82.8 & 56.5 & 100  & 73.6 &98.6 & 61.8  \\ \cline{2-11} 
        \multicolumn{1}{|c|}{}                           & 3 & 82.3   & 82.3& 100  & 76.4 & 58.8 & 100  & 100 & 100 & 82.3  \\ \cline{2-11}
        \multicolumn{1}{|c|}{}                           & 4 & 84.6   & 69.2& 30.76  & 92.3 & 76.9 & 92.3  & 76.9 & 92.3 & 69.2  \\ \cline{2-11}
        \multicolumn{1}{|c|}{}                           & 5 & 77.7   & 100& 33.3  & 66.6 & 44.4 & 100  & 55.5 & 100 & 100  \\ \cline{2-11}
        \multicolumn{1}{|c|}{\multirow{-3}{*}{\tiny Clusters}} & 6 & 58.6 & 79.3& 79.3  & 72.4 & 79.3 & 93.1  & 93.1 & 13.7 & 100   \\ 
        
        \hline
      \end{tabular}
    %}
    }
 %&
 %\hspace{1cm} %altera o espaçamento entre as tabelas
 \hspace{1cm}
   \subfloat[2a. Execução]{ \label{tab:execucoes:glass:nb:2exec}
 %\scalebox{0.5}{%
   \small\addtolength{\tabcolsep}{-2pt}
    \begin{tabular}{|cl|c|c|c|c|c|c|c|c|c|}
        \hline \hline
            {\tiny 2a. Exec}     &   & \multicolumn{9}{c|}{\tiny Atributos}                                               \\ \cline{3-11} 
       \multicolumn{1}{|l}{}                            &   & RI    & Na    & Mg  & Al   & Si   & K   & Ca   & Ba  & Fe             \\ \hline
        \multicolumn{1}{|c|}{}                           & 1 & 87.1 & 92.8  & 100 & 81.4 & 81.4 & 100 & 90.0 & 100 & 78.5 \\ \cline{2-11} 
        \multicolumn{1}{|c|}{}                           & 2 & 65.7 & 92.1  & 88.1& 82.8 & 63.1 & 100  & 72.3 &97.3 & 61.8  \\ \cline{2-11} 
        \multicolumn{1}{|c|}{}                           & 3 & 72.4  & 82.3& 100  & 76.4 & 47 & 100  & 100 & 100 & 82.3  \\ \cline{2-11}
        \multicolumn{1}{|c|}{}                           & 4 & 84.6   & 69.2& 23  & 92.3 & 76.9 & 92.3  & 76.9 & 92.3 & 61.5  \\ \cline{2-11}
        \multicolumn{1}{|c|}{}                           & 5 & 77.7   & 100& 33.3  & 66.6 & 44.4 & 100  & 55.5 & 100 & 100  \\ \cline{2-11}
        \multicolumn{1}{|c|}{\multirow{-3}{*}{\tiny Clusters}} & 6 & 58.6   & 79.3 & 79.3  & 68.9 & 79.3 & 93.1  & 93.1 & 17.2 & 100  \\ 
        
        \hline
      \end{tabular}
  %}
  %\\  [8ex]
    }
    
    \subfloat[3a. Execução]{ \label{tab:execucoes:glass:nb:3exec}
 %\scalebox{0.5}{%
   \small\addtolength{\tabcolsep}{-2pt} 
     \begin{tabular}{|cl|c|c|c|c|c|c|c|c|c|}
        \hline \hline
            {\tiny 3a. Exec}     &   & \multicolumn{9}{c|}{\tiny Atributos}                                               \\ \cline{3-11} 
       \multicolumn{1}{|l}{}                            &   & RI    & Na    & Mg  & Al   & Si   & K   & Ca   & Ba  & Fe             \\ \hline
        \multicolumn{1}{|c|}{}                           & 1 & 87.1 & 92.8  & 100 & 81.4 & 84.2 & 100 & 90.0 & 100 & 78.5 \\ \cline{2-11} 
        \multicolumn{1}{|c|}{}                           & 2 & 68.4 & 89.4  & 86.8& 84.2 & 60.5 & 100  & 72.3 &98.6 & 64.4  \\ \cline{2-11} 
        \multicolumn{1}{|c|}{}                           & 3 & 76.4   & 82.3& 100  & 76.4 & 52.9 & 100  & 100 & 100 & 82.3  \\ \cline{2-11}
        \multicolumn{1}{|c|}{}                           & 4 & 84.6   & 69.2& 23  & 92.3 & 76.9 & 92.3  & 76.9 & 92.3 & 76.9  \\ \cline{2-11}
        \multicolumn{1}{|c|}{}                           & 5 & 77.7   & 100& 33.3  & 66.6 & 44.4 & 100  & 55.5 & 100 & 100  \\ \cline{2-11}
        \multicolumn{1}{|c|}{\multirow{-3}{*}{\tiny Clusters}} & 6 & 58.6 & 79.3& 79.3  & 68.9 & 79.3 & 93.1  & 89.6 & 13.7 & 100   \\ 
        
        \hline
      \end{tabular}
    }
    %&
    \hspace{1cm}
    \subfloat[4a. Execução]{  \label{tab:execucoes:glass:nb:4exec}
       \small\addtolength{\tabcolsep}{-2pt}
    \begin{tabular}{|cl|c|c|c|c|c|c|c|c|c|}
        \hline \hline
            {\tiny 4a. Exec}     &   & \multicolumn{9}{c|}{\tiny Atributos}                                               \\ \cline{3-11} 
       \multicolumn{1}{|l}{}                            &   & RI    & Na    & Mg  & Al   & Si   & K   & Ca   & Ba  & Fe             \\ \hline
        \multicolumn{1}{|c|}{}                           & 1 & 87.1 & 92.8  & 100 & 81.4 & 84.2 & 100 & 90.0 & 100 & 78.5 \\ \cline{2-11} 
        \multicolumn{1}{|c|}{}                           & 2 & 65.7 & 90.7  & 86.8& 82.8 & 59.2 & 100  & 76.3 &98.6 & 63.1  \\ \cline{2-11} 
        \multicolumn{1}{|c|}{}                           & 3 & 76.4 & 82.3  & 100  & 76.4 & 52.4 & 100  & 100 & 100 & 82.3  \\ \cline{2-11}
        \multicolumn{1}{|c|}{}                           & 4 & 84.6 & 53.8  & 23  & 92.3 & 76.9 & 92.3  & 76.9 & 92.3 & 69.2  \\ \cline{2-11}
        \multicolumn{1}{|c|}{}                           & 5 & 77.7 & 100   & 33.3  & 66.6 & 44.4 & 100  & 55.5 & 100 & 100  \\ \cline{2-11}
        \multicolumn{1}{|c|}{\multirow{-3}{*}{\tiny Clusters}} & 6 & 58.6   & 82.7& 79.3  & 72.4 & 79.3 & 93.1  & 82.7 & 6.8 & 100  \\ 
        
        \hline
      \end{tabular}
   %\\
    }
    
 %\end{tabular}
 \label{tab:execucoes:glass:nb}
\end{table}

%%%%%%%%%%%%%%%%%%%%%%%%%%%%%%%%%%%%%%%%%%%%%%%%%%%%%%%%%%%%%%%%%%%%%%%%%%%%%%%%%%%%%%%%%%%%%%%%%%%%%%%%%%%%%%%%%%%%%%%%%%%%%%%%%%%%%%%%%%%%%%%%
%  
% \begin{table}[!ht]
% \caption{Resultado de ${4(quatro)}$ execuções do algoritmo Naive Bayes.}
%  \begin{tabular}{ll}
% %\rule{0}{50}
% 
%   
%    %\scalebox{0.5}{%
%    \small\addtolength{\tabcolsep}{-5pt}
%     \begin{tabular}{|cl|c|c|c|c|c|c|c|c|c|}
%         \hline \hline
%             {\tiny 1a. Exec}     &   & \multicolumn{9}{c|}{\tiny Atributos}                                               \\ \cline{3-11} 
%        \multicolumn{1}{|l}{}                            &   & RI    & Na    & Mg  & Al   & Si   & K   & Ca   & Ba  & Fe             \\ \hline
%         \multicolumn{1}{|c|}{}                           & 1 & 87.1 & 92.8  & 100 & 81.4 & 82.8 & 100 & 90.0 & 100 & 78.5 \\ \cline{2-11} 
%         \multicolumn{1}{|c|}{}                           & 2 & 65.7 & 86.8  & 85.5& 82.8 & 56.5 & 100  & 73.6 &98.6 & 61.8  \\ \cline{2-11} 
%         \multicolumn{1}{|c|}{}                           & 3 & 82.3   & 82.3& 100  & 76.4 & 58.8 & 100  & 100 & 100 & 82.3  \\ \cline{2-11}
%         \multicolumn{1}{|c|}{}                           & 4 & 84.6   & 69.2& 30.76  & 92.3 & 76.9 & 92.3  & 76.9 & 92.3 & 69.2  \\ \cline{2-11}
%         \multicolumn{1}{|c|}{}                           & 5 & 77.7   & 100& 33.3  & 66.6 & 44.4 & 100  & 55.5 & 100 & 100  \\ \cline{2-11}
%         \multicolumn{1}{|c|}{\multirow{-3}{*}{\tiny Clusters}} & 6 & 58.6 & 79.3& 79.3  & 72.4 & 79.3 & 93.1  & 93.1 & 13.7 & 100   \\ 
%         
%         \hline
%       \end{tabular}
%     %}
%  &
%  %\hspace{1cm} %altera o espaçamento entre as tabelas
%  
%    
%  %\scalebox{0.5}{%
%    \small\addtolength{\tabcolsep}{-5pt}
%     \begin{tabular}{|cl|c|c|c|c|c|c|c|c|c|}
%         \hline \hline
%             {\tiny 2a. Exec}     &   & \multicolumn{9}{c|}{\tiny Atributos}                                               \\ \cline{3-11} 
%        \multicolumn{1}{|l}{}                            &   & RI    & Na    & Mg  & Al   & Si   & K   & Ca   & Ba  & Fe             \\ \hline
%         \multicolumn{1}{|c|}{}                           & 1 & 87.1 & 92.8  & 100 & 81.4 & 81.4 & 100 & 90.0 & 100 & 78.5 \\ \cline{2-11} 
%         \multicolumn{1}{|c|}{}                           & 2 & 65.7 & 92.1  & 88.1& 82.8 & 63.1 & 100  & 72.3 &97.3 & 61.8  \\ \cline{2-11} 
%         \multicolumn{1}{|c|}{}                           & 3 & 72.4  & 82.3& 100  & 76.4 & 47 & 100  & 100 & 100 & 82.3  \\ \cline{2-11}
%         \multicolumn{1}{|c|}{}                           & 4 & 84.6   & 69.2& 23  & 92.3 & 76.9 & 92.3  & 76.9 & 92.3 & 61.5  \\ \cline{2-11}
%         \multicolumn{1}{|c|}{}                           & 5 & 77.7   & 100& 33.3  & 66.6 & 44.4 & 100  & 55.5 & 100 & 100  \\ \cline{2-11}
%         \multicolumn{1}{|c|}{\multirow{-3}{*}{\tiny Clusters}} & 6 & 58.6   & 79.3 & 79.3  & 68.9 & 79.3 & 93.1  & 93.1 & 17.2 & 100  \\ 
%         
%         \hline
%       \end{tabular}
%   %}
%   \\  [15ex]
%  %\hspace{1cm} %altera o espaçamento entre as tabelas
%  %\rule{0}{50}
%  
%  %\scalebox{0.5}{%
%    \small\addtolength{\tabcolsep}{-5pt}
%      \begin{tabular}{|cl|c|c|c|c|c|c|c|c|c|}
%         \hline \hline
%             {\tiny 3a. Exec}     &   & \multicolumn{9}{c|}{\tiny Atributos}                                               \\ \cline{3-11} 
%        \multicolumn{1}{|l}{}                            &   & RI    & Na    & Mg  & Al   & Si   & K   & Ca   & Ba  & Fe             \\ \hline
%         \multicolumn{1}{|c|}{}                           & 1 & 87.1 & 92.8  & 100 & 81.4 & 84.2 & 100 & 90.0 & 100 & 78.5 \\ \cline{2-11} 
%         \multicolumn{1}{|c|}{}                           & 2 & 68.4 & 89.4  & 86.8& 84.2 & 60.5 & 100  & 72.3 &98.6 & 64.4  \\ \cline{2-11} 
%         \multicolumn{1}{|c|}{}                           & 3 & 76.4   & 82.3& 100  & 76.4 & 52.9 & 100  & 100 & 100 & 82.3  \\ \cline{2-11}
%         \multicolumn{1}{|c|}{}                           & 4 & 84.6   & 69.2& 23  & 92.3 & 76.9 & 92.3  & 76.9 & 92.3 & 76.9  \\ \cline{2-11}
%         \multicolumn{1}{|c|}{}                           & 5 & 77.7   & 100& 33.3  & 66.6 & 44.4 & 100  & 55.5 & 100 & 100  \\ \cline{2-11}
%         \multicolumn{1}{|c|}{\multirow{-3}{*}{\tiny Clusters}} & 6 & 58.6 & 79.3& 79.3  & 68.9 & 79.3 & 93.1  & 89.6 & 13.7 & 100   \\ 
%         
%         \hline
%       \end{tabular}
%     
%     &
%     
%        \small\addtolength{\tabcolsep}{-5pt}
%     \begin{tabular}{|cl|c|c|c|c|c|c|c|c|c|}
%         \hline \hline
%             {\tiny 4a. Exec}     &   & \multicolumn{9}{c|}{\tiny Atributos}                                               \\ \cline{3-11} 
%        \multicolumn{1}{|l}{}                            &   & RI    & Na    & Mg  & Al   & Si   & K   & Ca   & Ba  & Fe             \\ \hline
%         \multicolumn{1}{|c|}{}                           & 1 & 87.1 & 92.8  & 100 & 81.4 & 84.2 & 100 & 90.0 & 100 & 78.5 \\ \cline{2-11} 
%         \multicolumn{1}{|c|}{}                           & 2 & 65.7 & 90.7  & 86.8& 82.8 & 59.2 & 100  & 76.3 &98.6 & 63.1  \\ \cline{2-11} 
%         \multicolumn{1}{|c|}{}                           & 3 & 76.4 & 82.3  & 100  & 76.4 & 52.4 & 100  & 100 & 100 & 82.3  \\ \cline{2-11}
%         \multicolumn{1}{|c|}{}                           & 4 & 84.6 & 53.8  & 23  & 92.3 & 76.9 & 92.3  & 76.9 & 92.3 & 69.2  \\ \cline{2-11}
%         \multicolumn{1}{|c|}{}                           & 5 & 77.7 & 100   & 33.3  & 66.6 & 44.4 & 100  & 55.5 & 100 & 100  \\ \cline{2-11}
%         \multicolumn{1}{|c|}{\multirow{-3}{*}{\tiny Clusters}} & 6 & 58.6   & 82.7& 79.3  & 72.4 & 79.3 & 93.1  & 82.7 & 6.8 & 100  \\ 
%         
%         \hline
%       \end{tabular}
%    \\
%  
%  \end{tabular}
%  \label{tab:execucoes:glass:nb}
% \end{table}

%%%%%%%%%%%%%%%%%%%%%%%%%%%%%%%%%%%%%%%%%%%%%%%%%%%%%%%%%%%%%%%%%%%%%%%%%%%%%%%%%%%%%%%%%%%%%%%%%%%%%%%%%%%%%%%%%%%%%%%%%%%%%%%%%%%%%%%%%%%%%%%%

De acordo com a aplicação do Naive Bayes na base de dados \textbf{Glass} os rótulos são os seguintes:
\begin{itemize}[noitemsep]
 \item ${r_{c_1}=\{ (Mg, [ 2.245 \sim 4.490 ]), (K,[ 0.0 \sim 1.5525 ] ), (Ba,[ 0.0 \sim 0.7875 ] ) \} }$  
 \item ${r_{c_2}=\{ (K,[ 0.0 \sim 1.5525 ] ) \} }$
 \item ${r_{c_3}=\{ (Mg, [ 2.245 \sim 4.490 ]), (K,[ 0.0 \sim 1.5525 ] ), (Ca,[ 8.12 \sim 10.81 ] ), \nonumber \\
 \\ (Ba,[ 0.0 \sim 0.7875 ] ) \} }$  
 \item ${r_{c_4}=\{(Al,[ 1.0925 \sim 1.895 ] ), (K,[ 0.0 \sim 1.5525 ] ), (Ba,[ 0.0 \sim 0.7875 ] ) \} }$
 \item ${r_{c_5}=\{ (Na,[ 14.055 \sim 17.380 ] ), (K,[ 0.0 \sim 1.5525 ] ), (Ba,[ 0.0 \sim 0.7875 ] ), (Fe,[ 0.0 \sim 0.1275 ] ) \} }$
 \item ${r_{c_6}=\{ (Fe,[ 0.0 \sim 0.1275 ] ) \} }$
\end{itemize}


\subsection{CART} \label{cap:resultados:ssec:glass:cart}

Ao utilizar o algoritmo CART logo percebe-se a semelhança com os resultados apresentados na subseção \ref{cap:resultados:ssec:glass:nb}. Apesar dessa semelhança os \textbf{Clusters 4} e \textbf{5} tiveram diferenças nos resultados em comparação ao algoritmo Naive Bayes. 

Ao verificar a \textbf{linha 4} da tabela \ref{tab:execucoes:glass:nb} do Naive Bayes, correspondente ao \textbf{Cluster 4}, os atributos \textbf{Al, K, Ba} apresentaram sempre os mesmos valores, mas já na tabela \ref{tab:execucoes:glass:cart}, também na \textbf{linha 4} de cada execução, só o valor de \textbf{Ba} coincide já os outros atributos tiveram valores mais baixos, fazendo com que eles não participassem da composição do rótulo.

No \textbf{Cluster 5} o atributo \textbf{Na} não faz parte do rótulo, e diferente do Naive Bayes  na tabela \ref{tab:execucoes:glass:nb}, verifica-se que os valores de \textbf{Na} são sempre 100\% de correlação entre os outros atributos. No CART os valores apresentados de \textbf{Na}  nas execuções da tabela \ref{tab:execucoes:glass:cart}, \textbf{linha 5}, são abaixo dos 78\%. Na \textbf{1a. Execução} da tabela \ref{tab:execucoes:glass:cart:1exec} os atributos que compõem o rótulo do \textbf{Cluster 5} apresentam também 100\%, portanto qualquer atributo com valor abaixo de 100\%  não será escolhido para compor o rótulo.

\begin{table}[!h]
\centering
\caption{Resultado da aplicação do algoritmo CART}
\label{tab:rot:glass:cart}
\scalebox{0.9}{
\begin{tabular}{llcrcc} 
\hline \hline
 
\multicolumn{1}{c}{\cellcolor[HTML]{FFFFFF}} & \multicolumn{2}{c}{Rótulos}                & \multicolumn{1}{r}{}               & \\ \cline{2-3}
Cluster                                      & Atributos      & \multicolumn{1}{c}{Faixa} & \multicolumn{1}{c}{Relevância(\%)} & Fora da Faixa & Acurácia Cluster(\%)\\ \hline \hline
                                             & Mg    & [ 2.245 $\sim$  4.490     ]       & 100\%                               & 0 & \\
                                             & K     & [ 0.0 $\sim$  1.5525      ]       & 100\%                               & 0 &\\  
\multirow{-3}{*}{1}                          & Ba    & [ 0.0 $\sim$  0.7875     ]       & 100\%                               & 0 &\multirow{-3}{*}{100\%} \\  \hline
%                                             & petallength    & ] 3.7 $\sim$  5.1 ]       & 88\%                               & 7\\ 
2                                            & K     & ] 0.0 $\sim$  1.5525 ]           & 100\%                               & 0 & 100\%\\  \hline
                                            & Mg     & ]  2.245 $\sim$  4.490  ]              & 100\%                               & 0 &\\ 
                                            & K     & ] 0.0 $\sim$  1.5525 ]               & 100\%                               & 0 &\\  
                                            & Ca     & ] 8.12 $\sim$  10.81 ]       & 100\%                               & 0 &\\ 
\multirow{-3}{*}{3}                          & Ba    & [ 0.0 $\sim$  0.7875     ]       & 100\%                               & 0 & \multirow{-3}{*}{100\%}\\  \hline
4                                           & Ba    & [ 0.0 $\sim$  0.7875     ]       & 92\%                               & 1 & 92,3\% \\  \hline
                                            & K     & [ 0.0 $\sim$  1.5525      ]       & 100\%                               & 0 & \\  
                                            & Ba    & [ 0.0 $\sim$  0.7875     ]       & 100\%                               & 0  &\\   
\multirow{-3}{*}{5}                          & Fe    & [ 0.0 $\sim$  0.1275     ]       & 100\%                               & 0 & \multirow{-3}{*}{100\%} \\  \hline
6                                            & Fe    & [ 0.0 $\sim$  0.1275     ]       & 100\%                               & 0 & 100\% \\  \hline

\hline

\end{tabular}
}
\end{table}

%%%%%%%%%%%%%%%%%%%%%%%%%%%%%%%%%%%%%%%%%%%%%%%%%%%%%%%%%%%%%%%%%%%%%%%%%%%%%%%%%%%%%%%%%%%%%%%%%%%%%%%%%%%%%%

\begin{table}[!h]
\caption{Resultado de 4 (quatro) execuções do algoritmo CART.}
 %\begin{tabular}{ll}
%\rule{0}{50}
\centering
   \subfloat[1a. Execução]{ \label{tab:execucoes:glass:cart:1exec}
   %\scalebox{0.5}{%
   \small\addtolength{\tabcolsep}{-2pt} 
    \begin{tabular}{|cl|c|c|c|c|c|c|c|c|c|}
        \hline \hline
            {\tiny 1a. Exec}     &   & \multicolumn{9}{c|}{\tiny Atributos}                                               \\ \cline{3-11} 
       \multicolumn{1}{|l}{}                            &   & RI    & Na    & Mg  & Al   & Si   & K   & Ca   & Ba  & Fe             \\ \hline
        \multicolumn{1}{|c|}{}                           & 1 & 88.5 & 90.0  & 100 & 92.8 & 84.2 & 100 & 92.8 & 100 & 75.7 \\ \cline{2-11} 
        \multicolumn{1}{|c|}{}                           & 2 & 72.3 & 82.8  & 94.7 & 82.8 & 71.0 & 100  & 77.6 &98.6 & 68.4  \\ \cline{2-11} 
        \multicolumn{1}{|c|}{}                           & 3 & 76.4   & 70.5& 100  & 47.0 & 76.4 & 100  & 100 & 100 & 76.4  \\ \cline{2-11}
        \multicolumn{1}{|c|}{}                           & 4 & 69.2   & 84.6& 76.9  & 61.5 & 69.2 & 76.9  & 76.9 & 92.3 & 84.6  \\ \cline{2-11}
        \multicolumn{1}{|c|}{}                           & 5 & 77.7   & 77.7& 44.4  & 66.6 & 66.6 & 100  & 55.5 & 100 & 100  \\ \cline{2-11}
        \multicolumn{1}{|c|}{\multirow{-3}{*}{\tiny Clusters}} & 6 & 72.4 & 75.8& 68.9  & 72.4 & 75.8 & 86.2  & 86.2 & 51.7 & 100   \\ 
        
        \hline
      \end{tabular}
    %}
    }
 %&
 %\hspace{1cm} %altera o espaçamento entre as tabelas
 \hspace{1cm}
   \subfloat[2a. Execução]{ \label{tab:execucoes:glass:cart:2exec}
 %\scalebox{0.5}{%
   \small\addtolength{\tabcolsep}{-2pt}
    \begin{tabular}{|cl|c|c|c|c|c|c|c|c|c|}
        \hline \hline
            {\tiny 2a. Exec}     &   & \multicolumn{9}{c|}{\tiny Atributos}                                               \\ \cline{3-11} 
       \multicolumn{1}{|l}{}                            &   & RI    & Na    & Mg  & Al   & Si   & K   & Ca   & Ba  & Fe             \\ \hline
        \multicolumn{1}{|c|}{}                           & 1 & 85.7 & 87.1  & 100 & 92.8 & 84.2 & 100 & 92.8 & 100 & 74.2 \\ \cline{2-11} 
        \multicolumn{1}{|c|}{}                           & 2 & 76.3 & 86.8  & 96.0 & 82.8 & 64.4 & 100  & 76.3 &98.6 & 68.4  \\ \cline{2-11} 
        \multicolumn{1}{|c|}{}                           & 3 & 76.4   & 82.3& 100  & 47.0 & 76.4 & 100  & 100 & 100 & 76.4  \\ \cline{2-11}
        \multicolumn{1}{|c|}{}                           & 4 & 76.9   & 84.6& 76.9  & 69.2 & 69.2 & 76.9  & 76.9 & 92.3 & 84.6  \\ \cline{2-11}
        \multicolumn{1}{|c|}{}                           & 5 & 77.7   & 77.7& 44.4  & 66.6 & 66.6 & 100  & 55.5 & 100 & 100  \\ \cline{2-11}
        \multicolumn{1}{|c|}{\multirow{-3}{*}{\tiny Clusters}} & 6 & 72.4 & 75.8& 65.5  & 72.4 & 75.8 & 93.1  & 93.1 & 51.7 & 100  \\ 
        
        \hline
      \end{tabular}
  %}
  %\\  [8ex]
    }
    
    \subfloat[3a. Execução]{ \label{tab:execucoes:glass:cart:3exec}
 %\scalebox{0.5}{%
   \small\addtolength{\tabcolsep}{-2pt} 
    \begin{tabular}{|cl|c|c|c|c|c|c|c|c|c|}
        \hline \hline
            {\tiny 3a. Exec}     &   & \multicolumn{9}{c|}{\tiny Atributos}                                               \\ \cline{3-11} 
       \multicolumn{1}{|l}{}                            &   & RI    & Na    & Mg  & Al   & Si   & K   & Ca   & Ba  & Fe             \\ \hline
        \multicolumn{1}{|c|}{}                           & 1 & 88.5 & 85.7  & 100 & 92.8 & 84.2 & 100 & 92.8 & 100 & 75.7 \\ \cline{2-11} 
        \multicolumn{1}{|c|}{}                           & 2 & 71.1 & 80.2  & 94.7 & 78.9 & 68.4 & 100  & 78.9 &98.6 & 65.7  \\ \cline{2-11} 
        \multicolumn{1}{|c|}{}                           & 3 & 76.4   & 82.3& 100  & 58.8 & 76.4 & 100  & 100 & 100 & 82.3  \\ \cline{2-11}
        \multicolumn{1}{|c|}{}                           & 4 & 76.9   & 84.6& 76.9  & 61.5 & 69.2 & 76.9  & 76.9 & 92.3 & 84.6  \\ \cline{2-11}
        \multicolumn{1}{|c|}{}                           & 5 & 77.7   & 77.7& 44.4  & 66.6 & 66.6 & 100  & 55.5 & 100 & 100  \\ \cline{2-11}
        \multicolumn{1}{|c|}{\multirow{-3}{*}{\tiny Clusters}} & 6 & 72.4 & 68.9& 65.9  & 68.9 & 75.8 & 89.6  & 93.1 & 55.1 & 100   \\ 
        
        \hline
      \end{tabular}
    }
    %&
    \hspace{1cm}
    \subfloat[4a. Execução]{  \label{tab:execucoes:glass:cart:4exec}
       \small\addtolength{\tabcolsep}{-2pt}
    \begin{tabular}{|cl|c|c|c|c|c|c|c|c|c|}
        \hline \hline
            {\tiny 4a. Exec}     &   & \multicolumn{9}{c|}{\tiny Atributos}                                               \\ \cline{3-11} 
       \multicolumn{1}{|l}{}                            &   & RI    & Na    & Mg  & Al   & Si   & K   & Ca   & Ba  & Fe             \\ \hline
        \multicolumn{1}{|c|}{}                           & 1 & 88.7 & 87.1  & 100 & 92.8 & 84.2 & 100 & 92.8 & 100 & 75.7 \\ \cline{2-11} 
        \multicolumn{1}{|c|}{}                           & 2 & 78.9 & 84.2  & 94.7 & 81.5 & 69.7 & 100  & 76.3 &98.6 & 65.7  \\ \cline{2-11} 
        \multicolumn{1}{|c|}{}                           & 3 & 76.4   & 82.3& 100  & 64.7 & 76.4 & 100  & 100 & 100 & 76.4  \\ \cline{2-11}
        \multicolumn{1}{|c|}{}                           & 4 & 76.9   & 84.6& 61.5  & 69.2 & 69.2 & 76.9  & 76.9 & 92.3 & 84.6  \\ \cline{2-11}
        \multicolumn{1}{|c|}{}                           & 5 & 77.7   & 77.7& 44.4  & 66.6 & 66.6 & 100  & 55.5 & 100 & 100  \\ \cline{2-11}
        \multicolumn{1}{|c|}{\multirow{-3}{*}{\tiny Clusters}} & 6 & 72.4 & 68.9& 68.9  & 68.9 & 75.8 & 93.1  & 93.1 & 51.7 & 100  \\ 
        
        \hline
      \end{tabular}
   %\\
    }
    
 %\end{tabular}
 \label{tab:execucoes:glass:cart}
\end{table}


%%%%%%%%%%%%%%%%%%%%%%%%%%%%%%%%%%%%%%%%%%%%%%%%%%%%%%%%%%%%%%%%%%%%%%%%%%%%%%%%%%%%%%%%%%%%%%%%%%%%%%%%%%%%%%
% 
% \begin{table}[!ht]
% \caption{Resultado de ${4(quatro)}$ execuções do algoritmo CART.}
%  \begin{tabular}{ll}
% %\rule{0}{50}
% 
%   
%    %\scalebox{0.5}{%
%    \small\addtolength{\tabcolsep}{-5pt}
%     \begin{tabular}{|cl|c|c|c|c|c|c|c|c|c|}
%         \hline \hline
%             {\tiny 1a. Exec}     &   & \multicolumn{9}{c|}{\tiny Atributos}                                               \\ \cline{3-11} 
%        \multicolumn{1}{|l}{}                            &   & RI    & Na    & Mg  & Al   & Si   & K   & Ca   & Ba  & Fe             \\ \hline
%         \multicolumn{1}{|c|}{}                           & 1 & 88.5 & 90.0  & 100 & 92.8 & 84.2 & 100 & 92.8 & 100 & 75.7 \\ \cline{2-11} 
%         \multicolumn{1}{|c|}{}                           & 2 & 72.3 & 82.8  & 94.7 & 82.8 & 71.0 & 100  & 77.6 &98.6 & 68.4  \\ \cline{2-11} 
%         \multicolumn{1}{|c|}{}                           & 3 & 76.4   & 70.5& 100  & 47.0 & 76.4 & 100  & 100 & 100 & 76.4  \\ \cline{2-11}
%         \multicolumn{1}{|c|}{}                           & 4 & 69.2   & 84.6& 76.9  & 61.5 & 69.2 & 76.9  & 76.9 & 92.3 & 84.6  \\ \cline{2-11}
%         \multicolumn{1}{|c|}{}                           & 5 & 77.7   & 77.7& 44.4  & 66.6 & 66.6 & 100  & 55.5 & 100 & 100  \\ \cline{2-11}
%         \multicolumn{1}{|c|}{\multirow{-3}{*}{\tiny Clusters}} & 6 & 72.4 & 75.8& 68.9  & 72.4 & 75.8 & 86.2  & 86.2 & 51.7 & 100   \\ 
%         
%         \hline
%       \end{tabular}
%     %}
%  &
%  %\hspace{1cm} %altera o espaçamento entre as tabelas
%  
%    
%  %\scalebox{0.5}{%
%    \small\addtolength{\tabcolsep}{-5pt}
%     \begin{tabular}{|cl|c|c|c|c|c|c|c|c|c|}
%         \hline \hline
%             {\tiny 2a. Exec}     &   & \multicolumn{9}{c|}{\tiny Atributos}                                               \\ \cline{3-11} 
%        \multicolumn{1}{|l}{}                            &   & RI    & Na    & Mg  & Al   & Si   & K   & Ca   & Ba  & Fe             \\ \hline
%         \multicolumn{1}{|c|}{}                           & 1 & 85.7 & 87.1  & 100 & 92.8 & 84.2 & 100 & 92.8 & 100 & 74.2 \\ \cline{2-11} 
%         \multicolumn{1}{|c|}{}                           & 2 & 76.3 & 86.8  & 96.0 & 82.8 & 64.4 & 100  & 76.3 &98.6 & 68.4  \\ \cline{2-11} 
%         \multicolumn{1}{|c|}{}                           & 3 & 76.4   & 82.3& 100  & 47.0 & 76.4 & 100  & 100 & 100 & 76.4  \\ \cline{2-11}
%         \multicolumn{1}{|c|}{}                           & 4 & 76.9   & 84.6& 76.9  & 69.2 & 69.2 & 76.9  & 76.9 & 92.3 & 84.6  \\ \cline{2-11}
%         \multicolumn{1}{|c|}{}                           & 5 & 77.7   & 77.7& 44.4  & 66.6 & 66.6 & 100  & 55.5 & 100 & 100  \\ \cline{2-11}
%         \multicolumn{1}{|c|}{\multirow{-3}{*}{\tiny Clusters}} & 6 & 72.4 & 75.8& 65.5  & 72.4 & 75.8 & 93.1  & 93.1 & 51.7 & 100  \\ 
%         
%         \hline
%       \end{tabular}
%   %}
%   \\  [15ex]
%  %\hspace{1cm} %altera o espaçamento entre as tabelas
%  %\rule{0}{50}
%  
%  %\scalebox{0.5}{%
%    \small\addtolength{\tabcolsep}{-5pt}
%      \begin{tabular}{|cl|c|c|c|c|c|c|c|c|c|}
%         \hline \hline
%             {\tiny 3a. Exec}     &   & \multicolumn{9}{c|}{\tiny Atributos}                                               \\ \cline{3-11} 
%        \multicolumn{1}{|l}{}                            &   & RI    & Na    & Mg  & Al   & Si   & K   & Ca   & Ba  & Fe             \\ \hline
%         \multicolumn{1}{|c|}{}                           & 1 & 88.5 & 85.7  & 100 & 92.8 & 84.2 & 100 & 92.8 & 100 & 75.7 \\ \cline{2-11} 
%         \multicolumn{1}{|c|}{}                           & 2 & 71.1 & 80.2  & 94.7 & 78.9 & 68.4 & 100  & 78.9 &98.6 & 65.7  \\ \cline{2-11} 
%         \multicolumn{1}{|c|}{}                           & 3 & 76.4   & 82.3& 100  & 58.8 & 76.4 & 100  & 100 & 100 & 82.3  \\ \cline{2-11}
%         \multicolumn{1}{|c|}{}                           & 4 & 76.9   & 84.6& 76.9  & 61.5 & 69.2 & 76.9  & 76.9 & 92.3 & 84.6  \\ \cline{2-11}
%         \multicolumn{1}{|c|}{}                           & 5 & 77.7   & 77.7& 44.4  & 66.6 & 66.6 & 100  & 55.5 & 100 & 100  \\ \cline{2-11}
%         \multicolumn{1}{|c|}{\multirow{-3}{*}{\tiny Clusters}} & 6 & 72.4 & 68.9& 65.9  & 68.9 & 75.8 & 89.6  & 93.1 & 55.1 & 100   \\ 
%         
%         \hline
%       \end{tabular}
%     
%     &
%     
%        \small\addtolength{\tabcolsep}{-5pt}
%     \begin{tabular}{|cl|c|c|c|c|c|c|c|c|c|}
%         \hline \hline
%             {\tiny 4a. Exec}     &   & \multicolumn{9}{c|}{\tiny Atributos}                                               \\ \cline{3-11} 
%        \multicolumn{1}{|l}{}                            &   & RI    & Na    & Mg  & Al   & Si   & K   & Ca   & Ba  & Fe             \\ \hline
%         \multicolumn{1}{|c|}{}                           & 1 & 88.7 & 87.1  & 100 & 92.8 & 84.2 & 100 & 92.8 & 100 & 75.7 \\ \cline{2-11} 
%         \multicolumn{1}{|c|}{}                           & 2 & 78.9 & 84.2  & 94.7 & 81.5 & 69.7 & 100  & 76.3 &98.6 & 65.7  \\ \cline{2-11} 
%         \multicolumn{1}{|c|}{}                           & 3 & 76.4   & 82.3& 100  & 64.7 & 76.4 & 100  & 100 & 100 & 76.4  \\ \cline{2-11}
%         \multicolumn{1}{|c|}{}                           & 4 & 76.9   & 84.6& 61.5  & 69.2 & 69.2 & 76.9  & 76.9 & 92.3 & 84.6  \\ \cline{2-11}
%         \multicolumn{1}{|c|}{}                           & 5 & 77.7   & 77.7& 44.4  & 66.6 & 66.6 & 100  & 55.5 & 100 & 100  \\ \cline{2-11}
%         \multicolumn{1}{|c|}{\multirow{-3}{*}{\tiny Clusters}} & 6 & 72.4 & 68.9& 68.9  & 68.9 & 75.8 & 93.1  & 93.1 & 51.7 & 100  \\ 
%         
%         \hline
%       \end{tabular}
%    \\
%  
%  \end{tabular}
%  \label{tab:execucoes:glass:cart}
% \end{table}


De acordo com a aplicação do CART na base de dados \textbf{Glass} os rótulos são os seguintes:
\begin{itemize}[noitemsep]
 \item ${r_{c_1}=\{ (Mg, [ 2.245 \sim 4.490 ]), (K,[ 0.0 \sim 1.5525 ] ), (Ba,[ 0.0 \sim 0.7875 ] ) \} }$  
 \item ${r_{c_2}=\{ (K,[ 0.0 \sim 1.5525 ] ) \} }$
 \item ${r_{c_3}=\{ (Mg, [ 2.245 \sim 4.490 ]), (K,[ 0.0 \sim 1.5525 ] ), (Ca,[ 8.12 \sim 10.81 ] ), (Ba,[ 0.0 \sim 0.7875 ] ) \} }$  
 \item ${r_{c_4}=\{ (Ba,[ 0.0 \sim 0.7875 ] ) \} }$
 \item ${r_{c_5}=\{ ( (K,[ 0.0 \sim 1.5525 ] ), (Ba,[ 0.0 \sim 0.7875 ] ), (Fe,[ 0.0 \sim 0.1275 ] ) \} }$
 \item ${r_{c_6}=\{ (Fe,[ 0.0 \sim 0.1275 ] ) \} }$
\end{itemize}



