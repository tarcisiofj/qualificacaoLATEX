\chapter{Resultados}\label{cap:resultados}

Os resultados obtidos aqui neste capítulo foram referentes a aplicação do método de rotulação em ${3(três)}$ bases de dados distintas. Um dos primeiros passos na análise de aprendizagem de máquina é quando o analista prepara os dados para poder utilizar  um método de aprendizagem apropriado. 

Então  a escolha da base de dados também tem influência direta em bons resultados. E sabendo disso a escolha dos conjuntos de dados utilizados nesta pesquisa foi por conta delas apresentarem características diferentes, e também por serem conhecidas, a análise acaba ficando mais clara servindo de amostras a outras base.

\section{Implementação}

Para conseguir gerar os resultados aqui escritos foram feitas implementações utilizando a ferramenta MATLAB \footnote{http://www.mathworks.com/products/matlab/}, onde junto  a ela é possível utilizar suas funções de aprendizado de máquina já prontas. MATLAB possui uma linguagem técnica, e de fácil implementação por já possuir uma gama de funções\footnote{versão: R2016a(9.0.0.341360); 64-bit (glnxa64)} preparadas para aprendizado de máquina. Por esses motivos essa ferramenta foi escolhida para colocar em prática essa pesquisa.

Foram realizados vários testes com o intuito de tentar otimizar resultados e poder comparálos a outras pesquisas já escritas. Seguindo essa linha foi determinado a escolha de ${3(três)}$ bases de dados já conhecidas, onde na implementação de cada uma delas surgiu algumas alterações, dependendo da base, na variável(V), quantidade de faixas(R) e método de discretização. Essas mudanças para cada base serviram para otimizar os resultados. 

Cada base de dados será aplicado dois algoritmos de aprendizado supervisionado que possuem paradigmas diferentes para servir de amostra e poder assim tirar conclusões sobre a rotulação em quaisquer algoritmos supervisionados.  

Os algoritmos utilizados foram o Naive Bayes, sessão \ref{sssec:nbayes}, com paradigma estatístico. E também o algoritmo Classification e Regression Trees - CART, \ref{sssec:cart}, com paradigma simbólico de  árvore de decisão.

\section{Seeds - Identificação de Tipos de Semente}
Essa base de dados da UCI Machine Learning\footnote{http://archive.ics.uci.edu/ml/} é composta por ${7(sete)}$ atributos definindo as características e mais ${1(um)}$ totalizando ${8(oito)}$ atributos, sendo este último um atributo classe, responsável por identificar o tipo de semente. E ainda possui um total de 210 registros classificados em ${3(três)}$ categorias:
\begin{itemize}[noitemsep]
 \item 70 elementos do tipo Kama;
 \item 70 elementos do tipo Rosa;
 \item 70 elementos do tipo Canadian.
\end{itemize}
Na configuração de implementação foi utilizado o método EFD de discretização com divisão em três faixas, ${R=3}$ para todos os atributos, e inserido o valor de variação ${V=3\%}$.


% Please add the following required packages to your document preamble:
% \usepackage{multirow}
% \usepackage[table,xcdraw]{xcolor}
% If you use beamer only pass "xcolor=table" option, i.e. \documentclss[xcolor=table]{beamer}
\begin{table}[!h]
\centering
\caption{Resultado da aplicação do algoritmo Naive Bayes}
\label{tab:rot:seeds:nb}
\begin{tabular}{llrl}
\hline \hline
\multicolumn{1}{c}{\cellcolor[HTML]{FFFFFF}} & \multicolumn{2}{c}{Rótulos}                      & \multicolumn{1}{r}{}            \\ \cline{2-3}
Cluster                                      & Atributos      & \multicolumn{1}{c}{Faixa}       & \multicolumn{1}{c}{Relevância(\%)} \\ \hline \hline
1                                            & area           & ] 12.78 $\sim$  16.14 ]           &    92\%                         \\  \hline
                                             & area           & ] 16.14 $\sim$  21.18 ]           &    95\%                       \\ 
\multirow{-2}{*}{2}                          & lkernel        & ] 5.826 $\sim$  6.675 ]           &    92\%                      \\  \hline
3                                            & perimetro      & [ 12.41 $\sim$  13.73 ]           &    95\%                       \\ \hline \hline
\end{tabular}
\end{table}

Na tabela \ref{tab:rot:seeds:nb} é apresentado os resultados com a execução do algoritmo Naive Bayes. Ela é formada por uma coluna informando os Clusters, os Rótulos sugeridos compostos pelo Atributo e sua Faixa de valor, e também da coluna Relevância. Essa última coluna exibe a resposta do algoritmo em porcentagem, da correlação do atributo em relação aos outros atributos do cluster.

A seleção dos atributos rótulos acontece em relação a variável ${V=3\%}$ variando do atributo de maior relevência. Caso essa variável sofra algum tipo de alteração, os rótulos dos clusters poderão sofrer alterações, pois poderá aumentar  ou diminuir o número de atributos, dependendo do valor inserido em ${V}$. Atravé da tabela XXX é possível analisar todos os valores de relevância gerados para os atributos e analisar qual valor pode-se inserir em ${V}$ para montar o rótulo.

TABELA VET ACERTOS

A tabela \ref{} é formada pelos clusters representado pela as linhas. E as colunas representam os atributos. Essa tabela é fruto do resultado da execução do Naive Bayes em cima dessa base de dados. Pode-se fazer uma análise e notar que algumas características são mais bem correlacionadas que  outras através de seus valores mais altos. 


\begin{table}[!h]
%\centering
\caption{Legenda da Tabela}
 
 
 \subfloat[Tabela da Esquerda]{

   %\centering
   \scalebox{0.5}{%
   \small\addtolength{\tabcolsep}{-4pt}
     \begin{tabular}{|cl|c|c|c|c|c|c|c|}
        \hline
        \cellcolor[HTML]{FFFFFF}                         &   & \multicolumn{7}{c|}{Atributos}                                               \\ \cline{3-9} 
        \multicolumn{1}{|l}{}                            &   & area    & perimetro & compacteness & Lkernel & wkernel & asymetry & lkgroove \\ \hline
        \multicolumn{1}{|c|}{}                           & 1 & 92.8571 & 87.1429   & 50.0000      & 75.7143 & 85.7143 & 60.000   & 657143   \\ \cline{2-9} 
        \multicolumn{1}{|c|}{}                           & 2 & 95.7143 & 91.4286   & 47.1429      & 92.8571 & 90.0000 & 28.5714  & 85.7143  \\ \cline{2-9} 
        \multicolumn{1}{|c|}{\multirow{-3}{*}{Clusters}} & 3 & 91.4286 & 95.7143   & 71.4286      & 85.7143 & 91.4286 & 64.2857  & 58.5714  \\ \hline
      \end{tabular}
    }
  
 }
% \hspace{3cm} %altera o espaçamento entre as tabelas
 
 \subfloat[Tabela da Direita]{

  %\centering
 \scalebox{0.5}{%
   \small\addtolength{\tabcolsep}{-4pt}
   \begin{tabular}{|cl|c|c|c|c|c|c|c|}
    \hline
    \cellcolor[HTML]{FFFFFF}                         &   & \multicolumn{7}{c|}{Atributos}                                               \\ \cline{3-9} 
    \multicolumn{1}{|l}{}                            &   & area    & perimetro & compacteness & Lkernel & wkernel & asymetry & lkgroove \\ \hline
    \multicolumn{1}{|c|}{}                           & 1 & 92.8571 & 87.1429   & 50.0000      & 75.7143 & 85.7143 & 60.000   & 657143   \\ \cline{2-9} 
    \multicolumn{1}{|c|}{}                           & 2 & 95.7143 & 91.4286   & 47.1429      & 92.8571 & 90.0000 & 28.5714  & 85.7143  \\ \cline{2-9} 
    \multicolumn{1}{|c|}{\multirow{-3}{*}{Clusters}} & 3 & 91.4286 & 95.7143   & 71.4286      & 85.7143 & 91.4286 & 64.2857  & 58.5714  \\ \hline
   \end{tabular}
  }
    
 }  
 
\end{table}

\begin{table}[t]
        \begin{minipage}{0.5\textwidth}
            \centering
            \begin{tabular}{r | c c c}
                $+$
                  & 1 & 2 & 3 \\\hline
                1 & 2 & 3 & 4 \\
                2 & 3 & 4 & 5 \\
                3 & 4 & 5 & 6
            \end{tabular}
            \caption{Addition}
        \end{minipage}
        \hfillx
        \begin{minipage}{0.5\textwidth}
            \centering
            \begin{tabular}{r | c c c}
                $\times$
                  & 1 & 2 & 3 \\\hline
                1 & 1 & 2 & 3 \\
                2 & 2 & 4 & 6 \\
                3 & 3 & 6 & 9
            \end{tabular}
            \caption{Multiplication}
        \end{minipage}
\end{table}

Mesmo fazendo várias execuções do algoritmo é possível verificar que mesmo havendo  valores diferentes em algumas tabelas, a correlaçao dos atributos continua a mesma.

explicar resultao, faixa e exibir rotulo






\begin{table}[!h]
\centering
\caption{Resultado da aplicação do algoritmo CART}
\label{tab:rot:seeds:cart}
\begin{tabular}{llrl}
\hline \hline
\multicolumn{1}{c}{\cellcolor[HTML]{FFFFFF}} & \multicolumn{2}{c}{Rótulos}                      & \multicolumn{1}{r}{}            \\ \cline{2-3}
Cluster                                      & Atributos      & \multicolumn{1}{c}{Faixa}       & \multicolumn{1}{c}{Relevância(\%)} \\ \hline\hline
                                             & area           & ] 12.78 $\sim$  16.14 ]         & 91\%          \\  
\multirow{-2}{*}{1}                          & perimetro      & [ 13.73 $\sim$ 15.18 ]          & 94\%          \\ \hline
                                             & area           & ] 16.14 $\sim$  21.18 ]          & 95\%          \\ 
\multirow{-2}{*}{2}                          & perimetro      & ] 15.18 $\sim$  17.25 ]          & 98\%          \\  \hline
                                             & perimetro      & [ 12.41 $\sim$  13.73 ]         & 95\%          \\
\multirow{-2}{*}{3}                          & wkernel        & [ 12.63 $\sim$  3.049 ]         & 97\%          \\ \hline \hline
\end{tabular}
\end{table}

\section{Iris - Identificação de Tipos de Plantas}

\section{Glass - Identificação de Tipos de Vidro}


glass
