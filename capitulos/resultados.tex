\chapter{Resultados}\label{cap:resultados}

Os resultados obtidos aqui neste capítulo foram referentes a aplicação do método de rotulação em ${3(três)}$ bases de dados distintas. Um dos primeiros passos na análise de aprendizagem de máquina é quando o analista prepara os dados para poder utilizar  um método de aprendizagem apropriado. 

Então  a escolha da base de dados também tem influência direta em bons resultados. E sabendo disso a escolha dos conjuntos de dados utilizados nesta pesquisa foi por conta delas apresentarem características diferentes, e também por serem conhecidas, facilitando a análise e servindo de amostra a outras base.

\section{Implementação}

Para conseguir gerar os resultados aqui escritos foram feitas implementações utilizando a ferramenta MATLAB \footnote{http://www.mathworks.com/products/matlab/}, onde junto  a ela é possível utilizar suas funções de aprendizado de máquina já prontas. MATLAB possui uma linguagem técnica, e de fácil implementação por já possuir uma gama de funções\footnote{versão: R2016a(9.0.0.341360); 64-bit (glnxa64)} preparadas para aprendizado de máquina. Por esses motivos essa ferramenta foi escolhida para colocar em prática essa pesquisa.

Foram realizados vários testes com o intuito de tentar otimizar resultados e poder comparálos a outras pesquisas já escritas. Seguindo essa linha foi determinado a escolha de ${3(três)}$ bases de dados já conhecidas, onde na implementação de cada uma delas surgiu algumas alterações, dependendo da base, na variável(V), quantidade de faixas(R) e método de discretização(EWD,EFD). Essas mudanças para cada base servirão para otimizar os resultados. 

Cada base de dados será aplicado dois algoritmos de aprendizado supervisionado que possuem paradigmas diferentes para servir de amostra e poder assim tirar conclusões sobre a rotulação em quaisquer algoritmos supervisionados.  

Os algoritmos utilizados foram o Naive Bayes, sessão \ref{sssec:nbayes}, com paradigma estatístico. E também o algoritmo Classification e Regression Trees - CART, \ref{sssec:cart}, com paradigma simbólico de  árvore de decisão.

\section{Seeds - Identificação de Tipos de Semente}
Essa base foi extraída da UCI Machine Learning\footnote{http://archive.ics.uci.edu/ml/}, composta por ${7(sete)}$ atributos definindo suas características e mais uma definindo sua classificação, sendo este último um atributo classe  responsável por identificar o tipo de semente. Possuindo um total de 210 registros classificados em ${3(três)}$ categorias:
\begin{itemize}[noitemsep]
 \item 70 elementos do tipo Kama;
 \item 70 elementos do tipo Rosa;
 \item 70 elementos do tipo Canadian.
\end{itemize}
Na configuração de implementação foi utilizado o método EFD de discretização com divisão em três faixas, ${R=3}$ para todos os atributos, e inserido o valor de variação ${V=3\%}$.

Na tabela \ref{tab:rot:seeds:nb} e tabela \ref{tab:rot:seeds:cart} são  apresentados os resultados com a execução do algoritmo Naive Bayes e CART respectivamente. Elas são formadas por uma coluna informando os \textbf{Clusters}, \textbf{Rótulos}  compostos pelo \textbf{Atributo} e sua \textbf{Faixa} de valor. Junto também a coluna \textbf{Relevância} exibindo a resposta do algoritmo em porcentagem, da correlação do atributo em relação aos outros atributos do cluster, retirado da tabela \ref{tab:matrelevancia:seeds:nb} e da tabela \ref{tab:matrelevancia:seeds:cart} respectivamente. E por último a coluna \textbf{Elem Fora da Faixa} que mostra a quantidade de elementos que não estão dentro da faixa do rótulo.

\subsection{Naive Bayes}
% Please add the following required packages to your document preamble:
% \usepackage{multirow}
% \usepackage[table,xcdraw]{xcolor}
% If you use beamer only pass "xcolor=table" option, i.e. \documentclss[xcolor=table]{beamer}
\begin{table}[!h]
\centering
\caption{Resultado da aplicação do algoritmo Naive Bayes}
\label{tab:rot:seeds:nb}
\begin{tabular}{llcrc}
\hline  \hline
\multicolumn{1}{c}{\cellcolor[HTML]{FFFFFF}} & \multicolumn{2}{c}{Rótulos}                & \multicolumn{1}{r}{}               & \\ \cline{2-3}
Cluster                                      & Atributos      & \multicolumn{1}{c}{Faixa} & \multicolumn{1}{c}{Relevância(\%)} & Elem fora da Faixa\\ \hline \hline
1                                            & area           & ] 12.78 $\sim$  16.14 ]   & 92\%                               & 14\\  \hline
                                             & area           & ] 16.14 $\sim$  21.18 ]   & 95\%                               & 6\\ 
\multirow{-2}{*}{2}                          & lkernel        & ] 5.826 $\sim$  6.675 ]   & 92\%                               & 6\\  \hline
3                                            & perimetro      & [ 12.41 $\sim$  13.73 ]   & 95\%                               & 5\\ \hline \hline
\end{tabular}
\end{table}



Analisando a coluna rótulo da tabela \ref{tab:rot:seeds:nb}, nota-se que o atributo \textbf{area} aparece tanto no  cluster 1 como também no cluster 2. A técnica envolve não só o rótulo como também a faixa que os valores mais se repetem dentro do atributo. Nesse caso pode-se observar que mesmo o atributo se repetindo entre os clusters. No cluster 2, a faixa de valores são direfentes, sem comentar que também nesse cluster existe outro atributo rótulo, \textbf{lkernel}.

A seleção dos atributos rótulos acontece da diferença da variável ${V=3\%}$ em relação ao atributo de maior relevência. Caso essa variável tenha o valor alterado, os rótulos dos clusters poderão sofrer mudanças, pois poderá aumentar  ou diminuir o número de atributos dos rótulos, dependendo do valor inserido em ${V}$. Através da tabela \ref{tab:matrelevancia:seeds:nb} é possível analisar todos os valores de relevância gerados para os atributos e analisar qual valor pode-se inserir em ${V}$ para montar o rótulo.

\begin{table}[!h]
    
    \caption{Resultado da Correlação dos atributos pelo Naive Bayes; Legenda dos Atributos: (A)area, (B)perimetro, (C)compacteness, (D)Lkernel, (E)Wkernel, (F)asymetry, (G)lkgroove}    
    \centering
   \small\addtolength{\tabcolsep}{+2pt}
    \begin{tabular}{|cl|c|c|c|c|c|c|c|}
        \hline \hline
                                &   & \multicolumn{7}{c|}{Atributos}          \\ \cline{3-9} 
        \multicolumn{1}{|l}{}                            &   & A    & B & C & D & E & F & G \\ \hline
        \multicolumn{1}{|c|}{}                           & 1 & 92.8 & 87.1   & 50.0      & 75.7 & 85.7 & 60.0   & 65.7   \\ \cline{2-9} 
        \multicolumn{1}{|c|}{}                           & 2 & 95.7 & 91.4   & 47.1      & 92.8 & 90.0 & 28.5  & 85.7  \\ \cline{2-9} 
        \multicolumn{1}{|c|}{\multirow{-3}{*}{Clusters}} & 3 & 91.4 & 95.7   & 71.4      & 85.7 & 91.4 & 64.2  & 58.5  \\ \hline
    \end{tabular}
    \label{tab:matrelevancia:seeds:nb} 
\end{table}

A tabela \ref{tab:matrelevancia:seeds:nb} é formada por clusters representado pelas linhas, e colunas representado por atributos. Essa tabela é fruto da implementação do Naive Bayes em cima dessa base de dados, e foi  gerada para auxiliar a retirada dos atributos rótulos. Uma análise pode ser feita através desses dados e ajudar a definir um valor para a variável ${V}$. Percebe-se que algumas características são mais bem correlacionadas que  outras, através de seus valores mais altos. Isso indica o grau de relacionamento entre os atributos após a aplicação do algoritmo. 


\begin{table}[!h]
\caption{Resultado de ${4(quatro)}$ execuções do algoritmo Naive Bayes; Legenda dos Atributos: (A)area, (B)perimetro, (C)compacteness, (D)Lkernel, (E)Wkernel, (F)asymetry, (G)lkgroove}
 \begin{tabular}{ll}
%\rule{0}{50}

  
   %\scalebox{0.5}{%
   \small\addtolength{\tabcolsep}{-4pt}
     \begin{tabular}{|cl|c|c|c|c|c|c|c|}
        \hline \hline
                                 &   & \multicolumn{7}{c|}{Atributos}                                               \\ \cline{3-9} 
       \multicolumn{1}{|l}{}                            &   & A    & B & C & D & E & F & G \\ \hline
        \multicolumn{1}{|c|}{}                           & 1 & 92.8 & 87.1   & 48.5      & 77.1 & 82.8 & 57.1   & 65.7   \\ \cline{2-9} 
        \multicolumn{1}{|c|}{}                           & 2 & 94.2 & 90.0   & 45.7      & 92.8 & 90.0 & 38.5  & 87.1  \\ \cline{2-9} 
        \multicolumn{1}{|c|}{\multirow{-3}{*}{Clusters}} & 3 & 91.4 & 95.7   & 72.8      & 85.7 & 91.4 & 64.2  & 60.0  \\ \hline
      \end{tabular}
    %}
 &
 %\hspace{1cm} %altera o espaçamento entre as tabelas
 
   
 %\scalebox{0.5}{%
   \small\addtolength{\tabcolsep}{-4pt}
   \begin{tabular}{|cl|c|c|c|c|c|c|c|}
        \hline \hline
                                 &   & \multicolumn{7}{c|}{Atributos}                                               \\ \cline{3-9} 
       \multicolumn{1}{|l}{}                            &   & A    & B & C & D & E & F & G \\ \hline
        \multicolumn{1}{|c|}{}                           & 1 & 92.8 & 87.1   & 47.1      & 77.1 & 87.1 & 60.0   & 65.7   \\ \cline{2-9} 
        \multicolumn{1}{|c|}{}                           & 2 & 94.2 & 90.0   & 47.1      & 92.8 & 91.4 & 32.8  & 87.1  \\ \cline{2-9} 
        \multicolumn{1}{|c|}{\multirow{-3}{*}{Clusters}} & 3 & 91.4 & 95.7   & 72.8      & 85.7 & 92.8 & 64.2  & 60.0  \\ \hline
      \end{tabular}
  %}
  \\  [8ex]
 %\hspace{1cm} %altera o espaçamento entre as tabelas
 %\rule{0}{50}
 
 %\scalebox{0.5}{%
   \small\addtolength{\tabcolsep}{-4pt}
   \begin{tabular}{|cl|c|c|c|c|c|c|c|}
        \hline \hline
                                 &   & \multicolumn{7}{c|}{Atributos}                                               \\ \cline{3-9} 
       \multicolumn{1}{|l}{}                            &   & A    & B & C & D & E & F & G \\ \hline
        \multicolumn{1}{|c|}{}                           & 1 & 94.2 & 85.7   & 48.5      & 77.1 & 82.8 & 61.4   & 65.7   \\ \cline{2-9} 
        \multicolumn{1}{|c|}{}                           & 2 & 92.8 & 90.0   & 50.0      & 92.8 & 90.0 & 32.8  & 87.1  \\ \cline{2-9} 
        \multicolumn{1}{|c|}{\multirow{-3}{*}{Clusters}} & 3 & 91.4 & 95.7   & 72.8      & 85.7 & 92.8 & 64.2  & 60.0  \\ \hline
   \end{tabular}
    
    &
    
       \small\addtolength{\tabcolsep}{-4pt}
   \begin{tabular}{|cl|c|c|c|c|c|c|c|}
        \hline \hline
                                 &   & \multicolumn{7}{c|}{Atributos}                                               \\ \cline{3-9} 
       \multicolumn{1}{|l}{}                            &   & A    & B & C & D & E & F & G \\ \hline
        \multicolumn{1}{|c|}{}                           & 1 & 91.4 & 88.5   & 54.2      & 75.7 & 85.7 & 62.8   & 61.4   \\ \cline{2-9} 
        \multicolumn{1}{|c|}{}                           & 2 & 95.7 & 90.0   & 50.0      & 92.8 & 90.0 & 38.5  & 85.7  \\ \cline{2-9} 
        \multicolumn{1}{|c|}{\multirow{-3}{*}{Clusters}} & 3 & 91.4 & 95.7   & 72.8      & 85.7 & 94.2 & 64.2  & 57.1  \\ \hline
   \end{tabular}
   \\
 
 \end{tabular}
 \label{tab:execucoes:seed:nb}
\end{table}

Para conseguir ter uma idéia mais ampla dessas informações, na tabela \ref{tab:execucoes:seed:nb} é exposto o resultado de ${4(quatro)}$ execuções do Algoritmo Naive Bayes, e pode-se constatar que mesmo havendo algumas alterações em seus valores nos atributos em cada execução, a correlação entre os atributos não oferece muita alteração. Como exemplo, o atributo \textbf{area}, possui o melhor grau de correlacionamento em seu grupo, mesmo testado em quatro execuções, como mostrado na tabela \ref{tab:execucoes:seed:nb}.

\subsection{CART}

Já na tabela \ref{tab:rot:seeds:cart}, tem-se o resultado da aplicação de outro algoritmo supervisionado, mas dessa vez com paradigma simbólico. No MATLAB o algoritmo de árvore de decisão utilizado é o CART. 

\begin{table}[!h]
\centering
\caption{Resultado da aplicação do algoritmo CART}
\label{tab:rot:seeds:cart}
\begin{tabular}{llcrc}\hline\hline 

\multicolumn{1}{c}{\cellcolor[HTML]{FFFFFF}} & \multicolumn{2}{c}{Rótulos}                      & \multicolumn{1}{r}{}            \\ \cline{2-3}
Cluster                                      & Atributos      & \multicolumn{1}{c}{Faixa}       & \multicolumn{1}{c}{Relevância(\%)} & Elem fora da Faixa \\ \hline \hline
                                             & area           & ] 12.78 $\sim$  16.14 ]         & 91\%          & 14 \\  
\multirow{-2}{*}{1}                          & perimetro      & [ 13.73 $\sim$ 15.18 ]          & 94\%          & 14\\ \hline
                                             & area           & ] 16.14 $\sim$  21.18 ]          & 95\%         & 6 \\ 
\multirow{-2}{*}{2}                          & perimetro      & ] 15.18 $\sim$  17.25 ]          & 98\%         & 7\\  \hline
                                             & perimetro      & [ 12.41 $\sim$  13.73 ]         & 95\%          & 5 \\
\multirow{-2}{*}{3}                          & wkernel        & [ 2.63 $\sim$  3.049 ]         & 97\%           & 9\\ \hline \hline
\end{tabular}
\end{table}

Pode-se verificar na tabela \ref{tab:rot:seeds:cart} e tabela \ref{tab:rot:seeds:cart} que os clusters 1 e 2 possuem o mesmo conjunto de atributos selecionados no campo de rótulo. Mas isso não implica dizer que os dois grupos são identificados pelo mesmo rótulo. O rótulo é composto pelos atributos e pelas faixas, onde a faixa é escolhida pelo maior número de valores que se repetem. 

Para entender a escolha desses atributos no campo de rótulos, a tabela \ref{tab:matrelevancia:seeds:cart} exibe o resultado gerado na execução do algoritmo em cima da base. Cada valor desses é o resultado da aplicação do algoritmo enquanto o atributo era a classe da vez coforme figura \ref{fig:tecnicamodelocomp}. O atributo de maior valor junto com os atributos com diferença de ${V}$, são escolhidos para ser rótulos. Na linha(cluster) 1 o maior valor é o perimetro, e fazendo a diferença de ${V=3}$ o atributo area entra também como rótulo.




\begin{table}[!h]
    
    \caption{Resultado da Correlação dos atributos pelo CART; Legenda dos Atributos: (A)area, (B)perimetro, (C)compacteness, (D)Lkernel, (E)Wkernel, (F)asymetry, (G)lkgroove}    
    \centering
   \small\addtolength{\tabcolsep}{+2pt}
    \begin{tabular}{|cl|c|c|c|c|c|c|c|}
        \hline \hline
                                &   & \multicolumn{7}{c|}{Atributos}          \\ \cline{3-9} 
        \multicolumn{1}{|l}{}                            &   & A    & B & C & D & E & F & G \\ \hline
        \multicolumn{1}{|c|}{}                           & 1 & 91.4 & 94.2   & 58.5      & 80.0 & 81.4 & 61.4   & 61.4   \\ \cline{2-9} 
        \multicolumn{1}{|c|}{}                           & 2 & 98.5 & 98.5   & 51.4      & 90.0 & 88.5 & 42.8  & 88.5  \\ \cline{2-9} 
        \multicolumn{1}{|c|}{\multirow{-3}{*}{Clusters}} & 3 & 92.7 & 95.7   & 80.0      & 88.5 & 97.1 & 58.5  & 78.5  \\ \hline
    \end{tabular}
    \label{tab:matrelevancia:seeds:cart} 
\end{table}

O resultado do algoritmo Naive Bayes na base de dados \textbf{Seeds} tem como rótulos: 
\begin{itemize}[noitemsep]
 \item ${r_{c_1}=\{ (area, ]12.78 ~ 16.14]) \} }$  
 \item ${r_{c_2}=\{ (area, ]16.14 ~ 21.18]), (Lkernel, ]5.826 ~ 6.675]) \} }$
 \item ${r_{c_3}=\{ (perimetro, [12.41 ~ 13.73]\} }$
\end{itemize}




O resultado do algoritmo Naive Bayes na base de dados \textbf{Seeds} tem como rótulos: 
\begin{itemize}[noitemsep]
 \item ${r_{c_1}=\{ (area, ]12.78 ~ 16.14]) \} }$  
 \item ${r_{c_2}=\{ (area, ]16.14 ~ 21.18]), (Lkernel, ]5.826 ~ 6.675]) \} }$
 \item ${r_{c_3}=\{ (perimetro, [12.41 ~ 13.73]\} }$
\end{itemize}


\section{Iris - Identificação de Tipos de Plantas}


%\begin{table}
%\centering

%\caption{Legenda da Tabela}
 

  
 
%\end{table}

\section{Glass - Identificação de Tipos de Vidro}
