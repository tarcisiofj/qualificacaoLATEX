\chapter{Metodologia / Materiais e Métodos}\label{cap:ferramentas}
Esse capítulo abordará em uma sessão o problema proposto por esse trabalho, e logo em seguida, será apresentado um modelo de resolução. O objetivo ao final deste capítulo é poder resolucionar o problema exibindo seus passos e atribuindo a qualquer outro pesquisador todo o conhecimento necessário para replicar este trabalho através das informações produzidas aqui.

\section{Considerações do Problema}

A abordagem do problema referente a essa proposta de mestrado segue uma linha já pequisada por \cite{Lopes}, que seria o \textbf{Problema de Rotulação}. Esse conceito, rotulação de dados,  já é estudado na literatura na área de aprendizagem não-supervisionada, sessão \ref{ssec:aprendNSup}, onde é comum os algoritmos lidarem com os agrupamentos dos dados, onde os grupos são criados a partir dos graus similaridade entre os elementos.

Muitas pesquisas realizadas na área de rotulação fazem referencia, de fato, a classificação do dados, e não da rotulação nos termos desse trabalho. Ao agrupar um conjunto de elementos por um derterminado critério, esta havendo uma classificação desses elementos escolhidos, mas pouco se sabe, qual é a compreensão desses grupos, já classificados. 

Tem-se então o real problema de rotulação, contudo seria necessário ter um rótulo definido para os grupos classificados para melhor compreender o porquê daquele grupo formado. Esse rótulo seria apresentação dos atributo(s) de maior relevância junto com a faixa, onde estaria nessa faixa, seus valores mais frequentes.

O Problema de Rotulação é formalmente definido como segue abaixo:

%\afterpage{
    \begin{quotation}

        \textit{ Dado um conjunto de clusters ${C=\{c_1,...,c_k | K \geqslant 1\} }$, de modo que cada cluster contém um conjunto de elementos ${c_i=\{\vec{e}_1,..,\vec{e}_{n^{(c_i)}}|n^{(c_i)} \geqslant 1 \}}$ que podem ser representados por um vetor de atributos definidos em ${\mathbb{R}^m }$ e expresso por ${ \vec{e}^{c_i}=(a_1,..,a_m)  }$ e ainda que  com ${ c_i \cap c_{i'}=\{0\} }$ com ${ 1 \leqslant i, i \leqslant K  }$ e ${ i \neq i' }$.
        }\footnotemark 
        \begin{itemize}[noitemsep]
            \item ${K}$ é o número de clusters;
            \item ${c_i}$ é o i-ésimo cluster qualquer;
            \item ${n^{c_i}}$ é o número de elementos do cluster ${c_i}$;
            \item ${\vec{e}_{n^{(c_i)}}}$ se refere ao j-ésimo elemento pertencente ao cluster ${c_i}$;
            \item ${m}$ é a dimensão do problema;
        \end{itemize}
    \footnotetext{Extraída de \cite{Lopes}}
    \end{quotation}
    %\footnotetext{Extraída de \cite{Lopes}}
%}


O estudo deste trabalho aproveita a perspectiva desse problema e cria um rótulo formado por seus atributos de mais relevância junto com os valores mais frequentes dess atribudo.

Como apresentado na sessão \ref{sec:trabcorrel}, o autor foca em rotulação automática de grupos utilizando aprendizagem de máquina supervisionada

