% ----------------------------------------------------------
% Apêndices
% ----------------------------------------------------------

% ---
% Inicia os apêndices
% ---
\begin{apendicesenv}

% Imprime uma página indicando o início dos apêndices
\partapendices

% ----------------------------------------------------------
\chapter{Primeiro Apêncice}
% ----------------------------------------------------------
\label{apendice:1}

Através da pesquisa de \citeonline{Lopes2016} que apresentou um modelo de rotulação de dados, ao qual através de um algoritmo não-supervisionado gera grupos de uma determinada base de dados, e logo após, é aplicado um outro algoritmo com aprendizagem supervisionada nesses grupos para detectar um rótulo para esses grupos. 

Foram realizados testes com as mesmas bases de dados com finalidade de comparar os resultados, mas os testes não foram satisfatórios por causa  da diferença entre os clusters criados, e mesmo utilizando as mesmas técnicas do autor citado acima, não foi o bastante para que os grupos fossem os mesmos, e dessa forma os teste ficaram incompatíveis.

Segue abaixo os testes realizados com o clusters recriados das bases: Seeds, Iris e Glass

\begin{table}[]
\centering
\caption{Rotulação de Dados utilizando a base de dados Seeds.}
\subfloat[Naive Bayes]{
\label{tab:comparativo:seeds:nb}
\scalebox{0.8}{
 \small\addtolength{\tabcolsep}{-2pt} 
\begin{tabular}{|c|c|c|c|c|}
\hline
\rowcolor[HTML]{EFEFEF} 
\hline
Cluster             & Num\_Elem              & Atributo     & Erro & Erro (\%) \\ \hline
1                   &    72                  &  asymetry    & 25   & 34.7\\ \hline
                    &                        & Lkernel        & 1  & 1.64\\  
\multirow{-2}{*}{2} &   \multirow{-2}{*}{61} & lkgroove        & 5  & 8.19\\ \hline 
3                   &    77                  & perimetro       & 10  & 12.98\\ \hline
\multicolumn{3}{|c|}{Total}                             & 41   &\\ \hline
\end{tabular}
} %fim scalebox
} % fim subfloat
\subfloat[CART]{
\label{tab:comparativo:seeds:cart}
\scalebox{0.8}{%
 \small\addtolength{\tabcolsep}{-2pt} 
\begin{tabular}{|c|c|c|c|c|}
\hline
\rowcolor[HTML]{EFEFEF} 
\hline
Cluster             & Num\_Elem              & Atributo     & Erro & Erro (\%)\\ \hline
1                   &    72                  &  perimetro    & 14  & 19.44 \\ \hline
2                   &   61                   & perimetro     & 0  & 0\\ \hline 
3                   &    77                  & perimetro      & 10  & 12.98\\ \hline
\multicolumn{3}{|c|}{Total}                             & 24  & \\ \hline
\end{tabular}
} %fim scalebox
} % fim subfloat

\subfloat[KNN]{
\label{tab:comparativo:seeds:knn}
\scalebox{0.8}{%
 \small\addtolength{\tabcolsep}{-2pt} 
\begin{tabular}{|c|c|c|c|c|}
\hline
\rowcolor[HTML]{EFEFEF} 
\hline
Cluster             & Num\_Elem              & Atributo     & Erro & Erro (\%)\\ \hline
1                   &    72                  &  perimetro    & 14   & 19.44\\ \hline
2                   &   61                   & lkgroove     & 5  & 8.19\\ \hline 
3                   &    77                  & perimetro      & 10 &  12.98\\ \hline
\multicolumn{3}{|c|}{Total}                             & 29  & \\ \hline
\end{tabular}
} %fim scalebox
} % fim subfloat
\subfloat[Resultado de \citeonline{Lopes2016}]{
\label{tab:comparativo:seeds:lopes}
\scalebox{0.8}{%
 \small\addtolength{\tabcolsep}{-2pt} 
\begin{tabular}{|c|c|c|c|c|}
\hline
\rowcolor[HTML]{EFEFEF} 
\hline
Cluster             & Num\_Elem              & Atributo     & Erro & Erro (\%) \\ \hline
                   &                       &  area    & 8   &  11.95 \\
\multirow{-2}{*}{1} &   \multirow{-2}{*}{67} & perimetro        & 9  & 13.64\\  \hline
                    &                        & area        & 12  & 14.64\\
\multirow{-2}{*}{2} &   \multirow{-2}{*}{82} & perimetro        & 10  & 12.2\\ \hline 
                    &                      & perimetro       & 0  & 0\\ 
                    &                        & wkernel        & 3  & 4.92\\  
                    &                      & lkernel       & 1  & 1.64\\ 
\multirow{-4}{*}{3} &   \multirow{-4}{*}{61} & area        & 0  & 0\\  \hline
\multicolumn{3}{|c|}{Total}                             & 43   &\\ \hline
\end{tabular}
} %fim scalebox
} % fim subfloat
\end{table}



\begin{table}[]
\centering
\caption{Rotulação de Dados utilizando a base de dados Iris.}
\subfloat[Naive Bayes]{
\label{tab:comparativo:iris:nb}
\scalebox{0.8}{
 \small\addtolength{\tabcolsep}{-2pt} 
\begin{tabular}{|c|c|c|c|c|}
\hline
\rowcolor[HTML]{EFEFEF} 
\hline
Cluster             & Num\_Elem              & Atributo     & Erro & Erro (\%) \\ \hline
                    &                        & sepallength        & 10  & 26.3\\  
\multirow{-2}{*}{1} &   \multirow{-2}{*}{38} & petalwidth        & 4  & 10.5\\ \hline 
2                   &    62                  & petalwidth       & 19  & 30.6\\ \hline
3                   &    50                  & petalwidth       & 0  & 0\\ \hline
\multicolumn{3}{|c|}{Total}                             & 33   &\\ \hline
\end{tabular}
} %fim scalebox
} % fim subfloat
\subfloat[CART]{
\label{tab:comparativo:iris:cart}
\scalebox{0.8}{%
 \small\addtolength{\tabcolsep}{-2pt} 
\begin{tabular}{|c|c|c|c|c|}
\hline
\rowcolor[HTML]{EFEFEF} 
\hline
Cluster             & Num\_Elem              & Atributo     & Erro & Erro (\%)\\ \hline
1                   &    38                  &  petalwidth    & 4  & 10.5 \\ \hline
2                   &   62                   & petalwidth     & 19  & 30.6\\ \hline 
3                   &    50                  & petalwidth      & 0  & 0\\ \hline
\multicolumn{3}{|c|}{Total}                             & 23  & \\ \hline
\end{tabular}
} %fim scalebox
} % fim subfloat

\subfloat[KNN]{
\label{tab:comparativo:iris:knn}
\scalebox{0.8}{%
 \small\addtolength{\tabcolsep}{-2pt} 
\begin{tabular}{|c|c|c|c|c|}
\hline
\rowcolor[HTML]{EFEFEF} 
\hline
Cluster             & Num\_Elem              & Atributo     & Erro & Erro (\%)\\ \hline
1                   &    38                  &  sepallength   & 10   & 26.3\\ \hline
2                   &   62                   & petalwidth     & 19  & 30.6\\ \hline 
3                   &    50                  & petalwidth      & 0 &  0\\ \hline
\multicolumn{3}{|c|}{Total}                             & 29  & \\ \hline
\end{tabular}
} %fim scalebox
} % fim subfloat
\subfloat[Resultado de \citeonline{Lopes2016}]{
\label{tab:comparativo:iris:lopes}
\scalebox{0.8}{%
 \small\addtolength{\tabcolsep}{-2pt} 
\begin{tabular}{|c|c|c|c|c|}
\hline
\rowcolor[HTML]{EFEFEF} 
\hline
Cluster             & Num\_Elem              & Atributo     & Erro & Erro (\%) \\ \hline
                   &                       &  petalwidth    & 0   & 0 \\
\multirow{-2}{*}{1} &   \multirow{-2}{*}{50} & petallength   & 0  & 0\\  \hline
                   2 &          62              & petallength   & 6  & 9.68\\ \hline
                     &                        & petallength        & 3  & 7.9\\
\multirow{-2}{*}{3} &   \multirow{-2}{*}{38} & petalwidth        & 2  & 5.27 \\ \hline 
\multicolumn{3}{|c|}{Total}                             & 11   &\\ \hline
\end{tabular}
} %fim scalebox
} % fim subfloat
\end{table}


\begin{table}[]
\centering
\caption{Rotulação de Dados utilizando a base de dados Glass.}
\subfloat[Naive Bayes]{
\label{tab:comparativo:glass:nb}
\scalebox{0.8}{
 \small\addtolength{\tabcolsep}{-2pt} 
\begin{tabular}{|c|c|c|c|c|}
\hline
\rowcolor[HTML]{EFEFEF} 
\hline
Cluster             & Num\_Elem              & Atributo     & Erro & Erro (\%)\\ \hline
                  &                    & Ba  & 1 & 1,45\\
\multirow{-2}{*}{1} & \multirow{-2}{*}{69} & Fe  & 11 & 15,94\\ \hline
 &  & Mg  & 2 & 50,00\\
 &  & K   & 2 & 50,00\\
 &  & Ba  & 2 & 50,00\\
\multirow{-3}{*}{2} & \multirow{-2}{*}{4}& Fe  & 0 & 0,00\\ \hline
 &  & RI  & 11 & 35,48\\
 &  & Mg  & 2 & 6,45\\
 &  & Al  & 14 & 45,16\\
 &  & Si  & 6 & 19,35\\
 &  & K   & 1 & 3,23\\
 &  & Ba  & 12 & 38,71\\
\multirow{-7}{*}{3} & \multirow{-7}{*}{31} & Fe  & 0 & 0,00\\ \hline
 &  & RI  & 6 & 16,67\\
 &  & Na  & 11 & 30,56\\
 &  & Mg  & 12 & 33,33\\
 &  & Al  & 17 & 47,22\\
 &  & Si  & 6 & 16,67\\
 &  & K   & 0 & 0,00\\
 &  & Ca  & 1 & 2,78\\
 &  & Ba  & 1 & 2,78\\
\multirow{-9}{*}{4} & \multirow{-9}{*}{36}& Fe  & 8 & 22,22\\ \hline
 &  & Mg  & 2 & 12,50\\
 &  & Al  & 6 & 37,50\\
 &  & Ba  & 1 & 6,25\\
\multirow{-4}{*}{5} & \multirow{-4}{*}{16} & Fe  & 4 & 25,00\\ \hline
 &  & Si  & 6 & 10,34\\
 &  & K   & 0 & 0,00\\
 &  & Ba  & 0 & 0,00\\
\multirow{-4}{*}{6} & \multirow{-4}{*}{58} &Fe& 20 & 34,48\\ \hline
\multicolumn{3}{|c|}{Total}                             & 165  & \\ \hline
\end{tabular}
} %fim scalebox
} % fim subfloat
\subfloat[CART]{
\label{tab:comparativo:glass:cart}
\scalebox{0.8}{%
 \small\addtolength{\tabcolsep}{-2pt} 
\begin{tabular}{|c|c|c|c|c|}
\hline
\rowcolor[HTML]{EFEFEF} 
\hline
Cluster             & Num\_Elem              & Atributo     & Erro & Erro (\%)\\ \hline
 &  & Ba  & 1 & 1,45\\
\multirow{-2}{*}{1} & \multirow{-2}{*}{69} & Fe  & 17 & 24,64\\  \hline
 &  & Mg  & 1 & 25,00\\
 &  & K   & 0 & 0,00\\
 &  & Ba  & 2 & 50,00\\
\multirow{-3}{*}{2} & \multirow{-3}{*}{4} & Fe  & 0 & 0,00\\ \hline
 &  & Mg  & 0 & 0,00\\
 &  & K   & 3 & 9,68\\
&  & Ba  & 22 & 70,97\\
\multirow{-4}{*}{3} & \multirow{-4}{*}{31} & Fe  & 0 & 0,00\\ \hline
 &  & RI  & 7 & 19,44\\
 &  & Na  & 19 & 52,78\\
 &  & Mg  & 20 & 55,56\\
 &  & Al  & 16 & 44,44\\
 &  & Si  & 8 & 22,22\\
 &  & K   & 15 & 41,67\\
 &  & Ca  & 14 & 38,89\\
 &  & Ba  & 2 & 5,56\\
\multirow{-9}{*}{4} & \multirow{-9}{*}{36} & Fe  & 10 & 27,78\\ \hline
 &  & Mg  & 0 & 0,00\\
 &  & Al  & 10 & 62,50\\
 &  & Ba  & 1 & 6,25\\
\multirow{-4}{*}{5} & \multirow{-4}{*}{16} & Fe  & 5 & 31,25\\ \hline
 &  & K   & 12 & 20,69\\
 &  & Ba  & 0 & 0,00\\
\multirow{-3}{*}{6} & \multirow{-3}{*}{58} & Fe  & 23 & 39,66\\ \hline

\multicolumn{3}{|c|}{Total}                             & 208  & \\ \hline
\end{tabular}
} %fim scalebox
} % fim subfloat

\subfloat[KNN]{
\label{tab:comparativo:glass:knn}
\scalebox{0.8}{%
 \small\addtolength{\tabcolsep}{-2pt} 
\begin{tabular}{|c|c|c|c|c|}
\hline
\rowcolor[HTML]{EFEFEF} 
\hline
Cluster             & Num\_Elem              & Atributo     & Erro & Erro (\%)\\ \hline
 &  & Ba  & 1 & 1,45\\
\multirow{-2}{*}{1} & \multirow{-2}{*}{69} & Fe  & 11 & 15,94\\ \hline
 &  & Mg  & 2 & 50,00\\
 &  & K   & 2 & 50,00\\
 &  & Ba  & 2 & 50,00\\
\multirow{-4}{*}{2} & \multirow{-4}{*}{4} & Fe  & 0 & 0,00\\ \hline
 &  & Mg  & 2 & 6,45\\
 &  & Al  & 14 & 45,16\\
 &  & K   & 1 & 3,23\\
 &  & Ba  & 12 & 38,71\\
\multirow{-5}{*}{3} & \multirow{-5}{*}{31} & Fe  & 0 & 0,00\\ \hline
 &  & RI  & 6 & 16,67\\
 &  & Na  & 11 & 30,56\\
 &  & Al  & 17 & 47,22\\
 &  & Si  & 6 & 16,67\\
 &  & K   & 0 & 0,00\\
 &  & Ca  & 1 & 2,78\\
 &  & Ba  & 1 & 2,78\\
\multirow{-8}{*}{1} & \multirow{-8}{*}{36} & Fe  & 8 & 22,22\\ \hline
 &  & Mg  & 2 & 12,50\\
 &  & Ba  & 1 & 6,25\\
\multirow{-3}{*}{5} & \multirow{-3}{*}{16} & Fe  & 4 & 25,00\\ \hline
 &  & K   & 0 & 0,00\\
 &  & Ba  & 0 & 0,00\\
\multirow{-3}{*}{6} & \multirow{-3}{*}{58} & Fe  & 20 & 34,48\\ \hline
\multicolumn{3}{|c|}{Total}                             &  124 & \\ \hline
\end{tabular}
} %fim scalebox
} % fim subfloat
\subfloat[Resultado de \citeonline{Lopes2016}]{
\label{tab:comparativo:glass:lopes}
\scalebox{0.8}{%
 \small\addtolength{\tabcolsep}{-2pt} 
\begin{tabular}{|c|c|c|c|c|}
\hline
\rowcolor[HTML]{EFEFEF} 
\hline
Cluster             & Num\_Elem              & Atributo     & Erro & Erro (\%) \\ \hline
 &  & Ba & 0 & 0\\
 &  & K & 0 & 0\\
 &  & Si & 2 & 2,71\\
\multirow{-2}{*}{1} &   \multirow{-2}{*}{74}& Na & 3 & 4,06\\ \hline
  &  & Fe & 0 & 0\\
 \multirow{-2}{*}{2} &   \multirow{-2}{*}{5} & Ca & 0 & 0\\ \hline
  &  & K & 0 & 0\\
 \multirow{-2}{*}{3} &   \multirow{-2}{*}{19} & Ba & 1 & 5,27 \\ \hline
  &   & K & 0 & 0\\
 &  &  Ba & 1 & 3,13\\
\multirow{-3}{*}{4} &   \multirow{-3}{*}{32} & Ca & 1 & 3,13\\ \hline
 &  & Ba & 0 & 0\\
 &  & K & 0 & 0\\
 &  &  Na & 2 & 3,58\\
 &  & Al & 4 & 7,15\\
\multirow{-5}{*}{5} &   \multirow{-5}{*}{56}  & Mg & 6 & 10,72\\ \hline
  &  & Fe & 0 & 0\\
 \multirow{-2}{*}{6} &   \multirow{-2}{*}{28} & K & 1 & 3,58 \\ \hline
\multicolumn{3}{|c|}{Total}                             & 21   &\\ \hline
\end{tabular}
} %fim scalebox
} % fim subfloat
\end{table}

%\lipsum[50] % Texto qualquer. REMOVER!!

% ----------------------------------------------------------
%\chapter{Perceba que o texto do título desse segundo apêndice é bem grande}
% ----------------------------------------------------------
%\lipsum[51-53] % Texto qualquer. REMOVER!!

\end{apendicesenv}
% ---
