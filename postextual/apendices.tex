% ----------------------------------------------------------
% Apêndices
% ----------------------------------------------------------

% ---
% Inicia os apêndices
% ---
\begin{apendicesenv}

% Imprime uma página indicando o início dos apêndices
\partapendices

% ----------------------------------------------------------
\chapter{Outros resultados de Rotulação }
% ----------------------------------------------------------
\label{apendice:1}

Através da pesquisa de \citeonline{Lopes2016} que apresentou um modelo de rotulação de dados, ao qual através de um algoritmo não-supervisionado gera grupos de uma determinada base de dados, e logo após, é aplicado um outro algoritmo com aprendizagem supervisionada nesses grupos para detectar um rótulo para esses grupos. Mediante isso foram realizados testes com as mesmas bases de dados com finalidade de comparar os resultados, mas os testes não foram satisfatórios, embora algumas bases sejam iguais houve diferença entre os clusters criados. Foram realizados o uso das mesmas técnicas do autor citado acima, todavia não foi o bastante para que os grupos fossem os mesmos, ou seja, grupos definidos não são iguais e por consequência os teste ficaram incompatíveis.

Segue nas tabelas abaixo o comparativo do trabalho do \cite{Lopes2016} com os testes realizados nos clusters recriados das bases: Seeds, Iris e Glass, com o trabalho 

\begin{table}[]
\centering
\caption{Rotulação de Dados utilizando a base de dados Seeds.}
\subfloat[Naive Bayes]{
\label{tab:comparativo:seeds:nb}
\scalebox{0.8}{
 \small\addtolength{\tabcolsep}{-2pt} 
\begin{tabular}{|c|c|c|c|c|}
\hline
\rowcolor[HTML]{EFEFEF} 
\hline
Cluster             & Num\_Elem              & Atributo     & Erro & Erro (\%) \\ \hline
1                   &    72                  &  asymetry    & 25   & 34.7\\ \hline
                    &                        & Lkernel        & 1  & 1.64\\  
\multirow{-2}{*}{2} &   \multirow{-2}{*}{61} & lkgroove        & 5  & 8.19\\ \hline 
3                   &    77                  & perimetro       & 10  & 12.98\\ \hline
\multicolumn{3}{|c|}{Total}                             & 41   &\\ \hline
\end{tabular}
} %fim scalebox
} % fim subfloat
\subfloat[CART]{
\label{tab:comparativo:seeds:cart}
\scalebox{0.8}{%
 \small\addtolength{\tabcolsep}{-2pt} 
\begin{tabular}{|c|c|c|c|c|}
\hline
\rowcolor[HTML]{EFEFEF} 
\hline
Cluster             & Num\_Elem              & Atributo     & Erro & Erro (\%)\\ \hline
1                   &    72                  &  perimetro    & 14  & 19.44 \\ \hline
2                   &   61                   & perimetro     & 0  & 0\\ \hline 
3                   &    77                  & perimetro      & 10  & 12.98\\ \hline
\multicolumn{3}{|c|}{Total}                             & 24  & \\ \hline
\end{tabular}
} %fim scalebox
} % fim subfloat

\subfloat[KNN]{
\label{tab:comparativo:seeds:knn}
\scalebox{0.8}{%
 \small\addtolength{\tabcolsep}{-2pt} 
\begin{tabular}{|c|c|c|c|c|}
\hline
\rowcolor[HTML]{EFEFEF} 
\hline
Cluster             & Num\_Elem              & Atributo     & Erro & Erro (\%)\\ \hline
1                   &    72                  &  perimetro    & 14   & 19.44\\ \hline
2                   &   61                   & lkgroove     & 5  & 8.19\\ \hline 
3                   &    77                  & perimetro      & 10 &  12.98\\ \hline
\multicolumn{3}{|c|}{Total}                             & 29  & \\ \hline
\end{tabular}
} %fim scalebox
} % fim subfloat
\subfloat[Resultado de \citeonline{Lopes2016}]{
\label{tab:comparativo:seeds:lopes}
\scalebox{0.8}{%
 \small\addtolength{\tabcolsep}{-2pt} 
\begin{tabular}{|c|c|c|c|c|}
\hline
\rowcolor[HTML]{EFEFEF} 
\hline
Cluster             & Num\_Elem              & Atributo     & Erro & Erro (\%) \\ \hline
                   &                       &  area    & 8   &  11.95 \\
\multirow{-2}{*}{1} &   \multirow{-2}{*}{67} & perimetro        & 9  & 13.64\\  \hline
                    &                        & area        & 12  & 14.64\\
\multirow{-2}{*}{2} &   \multirow{-2}{*}{82} & perimetro        & 10  & 12.2\\ \hline 
                    &                      & perimetro       & 0  & 0\\ 
                    &                        & wkernel        & 3  & 4.92\\  
                    &                      & lkernel       & 1  & 1.64\\ 
\multirow{-4}{*}{3} &   \multirow{-4}{*}{61} & area        & 0  & 0\\  \hline
\multicolumn{3}{|c|}{Total}                             & 43   &\\ \hline
\end{tabular}
} %fim scalebox
} % fim subfloat
\end{table}



\begin{table}[]
\centering
\caption{Rotulação de Dados utilizando a base de dados Iris.}
\subfloat[Naive Bayes]{
\label{tab:comparativo:iris:nb}
\scalebox{0.8}{
 \small\addtolength{\tabcolsep}{-2pt} 
\begin{tabular}{|c|c|c|c|c|}
\hline
\rowcolor[HTML]{EFEFEF} 
\hline
Cluster             & Num\_Elem              & Atributo     & Erro & Erro (\%) \\ \hline
                    &                        & sepallength        & 10  & 26.3\\  
\multirow{-2}{*}{1} &   \multirow{-2}{*}{38} & petalwidth        & 4  & 10.5\\ \hline 
2                   &    62                  & petalwidth       & 19  & 30.6\\ \hline
3                   &    50                  & petalwidth       & 0  & 0\\ \hline
\multicolumn{3}{|c|}{Total}                             & 33   &\\ \hline
\end{tabular}
} %fim scalebox
} % fim subfloat
\subfloat[CART]{
\label{tab:comparativo:iris:cart}
\scalebox{0.8}{%
 \small\addtolength{\tabcolsep}{-2pt} 
\begin{tabular}{|c|c|c|c|c|}
\hline
\rowcolor[HTML]{EFEFEF} 
\hline
Cluster             & Num\_Elem              & Atributo     & Erro & Erro (\%)\\ \hline
1                   &    38                  &  petalwidth    & 4  & 10.5 \\ \hline
2                   &   62                   & petalwidth     & 19  & 30.6\\ \hline 
3                   &    50                  & petalwidth      & 0  & 0\\ \hline
\multicolumn{3}{|c|}{Total}                             & 23  & \\ \hline
\end{tabular}
} %fim scalebox
} % fim subfloat

\subfloat[KNN]{
\label{tab:comparativo:iris:knn}
\scalebox{0.8}{%
 \small\addtolength{\tabcolsep}{-2pt} 
\begin{tabular}{|c|c|c|c|c|}
\hline
\rowcolor[HTML]{EFEFEF} 
\hline
Cluster             & Num\_Elem              & Atributo     & Erro & Erro (\%)\\ \hline
1                   &    38                  &  sepallength   & 10   & 26.3\\ \hline
2                   &   62                   & petalwidth     & 19  & 30.6\\ \hline 
3                   &    50                  & petalwidth      & 0 &  0\\ \hline
\multicolumn{3}{|c|}{Total}                             & 29  & \\ \hline
\end{tabular}
} %fim scalebox
} % fim subfloat
\subfloat[Resultado de \citeonline{Lopes2016}]{
\label{tab:comparativo:iris:lopes}
\scalebox{0.8}{%
 \small\addtolength{\tabcolsep}{-2pt} 
\begin{tabular}{|c|c|c|c|c|}
\hline
\rowcolor[HTML]{EFEFEF} 
\hline
Cluster             & Num\_Elem              & Atributo     & Erro & Erro (\%) \\ \hline
                   &                       &  petalwidth    & 0   & 0 \\
\multirow{-2}{*}{1} &   \multirow{-2}{*}{50} & petallength   & 0  & 0\\  \hline
                   2 &          62              & petallength   & 6  & 9.68\\ \hline
                     &                        & petallength        & 3  & 7.9\\
\multirow{-2}{*}{3} &   \multirow{-2}{*}{38} & petalwidth        & 2  & 5.27 \\ \hline 
\multicolumn{3}{|c|}{Total}                             & 11   &\\ \hline
\end{tabular}
} %fim scalebox
} % fim subfloat
\end{table}


\begin{table}[]
\centering
\caption{Rotulação de Dados utilizando a base de dados Glass.}
\subfloat[Naive Bayes]{
\label{tab:comparativo:glass:nb}
\scalebox{0.8}{
 \small\addtolength{\tabcolsep}{-2pt} 
\begin{tabular}{|c|c|c|c|c|}
\hline
\rowcolor[HTML]{EFEFEF} 
\hline
Cluster             & Num\_Elem              & Atributo     & Erro & Erro (\%)\\ \hline
                  &                    & Ba  & 1 & 1,45\\
\multirow{-2}{*}{1} & \multirow{-2}{*}{69} & Fe  & 11 & 15,94\\ \hline
 &  & Mg  & 2 & 50,00\\
 &  & K   & 2 & 50,00\\
 &  & Ba  & 2 & 50,00\\
\multirow{-3}{*}{2} & \multirow{-2}{*}{4}& Fe  & 0 & 0,00\\ \hline
 &  & RI  & 11 & 35,48\\
 &  & Mg  & 2 & 6,45\\
 &  & Al  & 14 & 45,16\\
 &  & Si  & 6 & 19,35\\
 &  & K   & 1 & 3,23\\
 &  & Ba  & 12 & 38,71\\
\multirow{-7}{*}{3} & \multirow{-7}{*}{31} & Fe  & 0 & 0,00\\ \hline
 &  & RI  & 6 & 16,67\\
 &  & Na  & 11 & 30,56\\
 &  & Mg  & 12 & 33,33\\
 &  & Al  & 17 & 47,22\\
 &  & Si  & 6 & 16,67\\
 &  & K   & 0 & 0,00\\
 &  & Ca  & 1 & 2,78\\
 &  & Ba  & 1 & 2,78\\
\multirow{-9}{*}{4} & \multirow{-9}{*}{36}& Fe  & 8 & 22,22\\ \hline
 &  & Mg  & 2 & 12,50\\
 &  & Al  & 6 & 37,50\\
 &  & Ba  & 1 & 6,25\\
\multirow{-4}{*}{5} & \multirow{-4}{*}{16} & Fe  & 4 & 25,00\\ \hline
 &  & Si  & 6 & 10,34\\
 &  & K   & 0 & 0,00\\
 &  & Ba  & 0 & 0,00\\
\multirow{-4}{*}{6} & \multirow{-4}{*}{58} &Fe& 20 & 34,48\\ \hline
\multicolumn{3}{|c|}{Total}                             & 165  & \\ \hline
\end{tabular}
} %fim scalebox
} % fim subfloat
\subfloat[CART]{
\label{tab:comparativo:glass:cart}
\scalebox{0.8}{%
 \small\addtolength{\tabcolsep}{-2pt} 
\begin{tabular}{|c|c|c|c|c|}
\hline
\rowcolor[HTML]{EFEFEF} 
\hline
Cluster             & Num\_Elem              & Atributo     & Erro & Erro (\%)\\ \hline
 &  & Ba  & 1 & 1,45\\
\multirow{-2}{*}{1} & \multirow{-2}{*}{69} & Fe  & 17 & 24,64\\  \hline
 &  & Mg  & 1 & 25,00\\
 &  & K   & 0 & 0,00\\
 &  & Ba  & 2 & 50,00\\
\multirow{-3}{*}{2} & \multirow{-3}{*}{4} & Fe  & 0 & 0,00\\ \hline
 &  & Mg  & 0 & 0,00\\
 &  & K   & 3 & 9,68\\
&  & Ba  & 22 & 70,97\\
\multirow{-4}{*}{3} & \multirow{-4}{*}{31} & Fe  & 0 & 0,00\\ \hline
 &  & RI  & 7 & 19,44\\
 &  & Na  & 19 & 52,78\\
 &  & Mg  & 20 & 55,56\\
 &  & Al  & 16 & 44,44\\
 &  & Si  & 8 & 22,22\\
 &  & K   & 15 & 41,67\\
 &  & Ca  & 14 & 38,89\\
 &  & Ba  & 2 & 5,56\\
\multirow{-9}{*}{4} & \multirow{-9}{*}{36} & Fe  & 10 & 27,78\\ \hline
 &  & Mg  & 0 & 0,00\\
 &  & Al  & 10 & 62,50\\
 &  & Ba  & 1 & 6,25\\
\multirow{-4}{*}{5} & \multirow{-4}{*}{16} & Fe  & 5 & 31,25\\ \hline
 &  & K   & 12 & 20,69\\
 &  & Ba  & 0 & 0,00\\
\multirow{-3}{*}{6} & \multirow{-3}{*}{58} & Fe  & 23 & 39,66\\ \hline

\multicolumn{3}{|c|}{Total}                             & 208  & \\ \hline
\end{tabular}
} %fim scalebox
} % fim subfloat

\subfloat[KNN]{
\label{tab:comparativo:glass:knn}
\scalebox{0.8}{%
 \small\addtolength{\tabcolsep}{-2pt} 
\begin{tabular}{|c|c|c|c|c|}
\hline
\rowcolor[HTML]{EFEFEF} 
\hline
Cluster             & Num\_Elem              & Atributo     & Erro & Erro (\%)\\ \hline
 &  & Ba  & 1 & 1,45\\
\multirow{-2}{*}{1} & \multirow{-2}{*}{69} & Fe  & 11 & 15,94\\ \hline
 &  & Mg  & 2 & 50,00\\
 &  & K   & 2 & 50,00\\
 &  & Ba  & 2 & 50,00\\
\multirow{-4}{*}{2} & \multirow{-4}{*}{4} & Fe  & 0 & 0,00\\ \hline
 &  & Mg  & 2 & 6,45\\
 &  & Al  & 14 & 45,16\\
 &  & K   & 1 & 3,23\\
 &  & Ba  & 12 & 38,71\\
\multirow{-5}{*}{3} & \multirow{-5}{*}{31} & Fe  & 0 & 0,00\\ \hline
 &  & RI  & 6 & 16,67\\
 &  & Na  & 11 & 30,56\\
 &  & Al  & 17 & 47,22\\
 &  & Si  & 6 & 16,67\\
 &  & K   & 0 & 0,00\\
 &  & Ca  & 1 & 2,78\\
 &  & Ba  & 1 & 2,78\\
\multirow{-8}{*}{1} & \multirow{-8}{*}{36} & Fe  & 8 & 22,22\\ \hline
 &  & Mg  & 2 & 12,50\\
 &  & Ba  & 1 & 6,25\\
\multirow{-3}{*}{5} & \multirow{-3}{*}{16} & Fe  & 4 & 25,00\\ \hline
 &  & K   & 0 & 0,00\\
 &  & Ba  & 0 & 0,00\\
\multirow{-3}{*}{6} & \multirow{-3}{*}{58} & Fe  & 20 & 34,48\\ \hline
\multicolumn{3}{|c|}{Total}                             &  124 & \\ \hline
\end{tabular}
} %fim scalebox
} % fim subfloat
\subfloat[Resultado de \citeonline{Lopes2016}]{
\label{tab:comparativo:glass:lopes}
\scalebox{0.8}{%
 \small\addtolength{\tabcolsep}{-2pt} 
\begin{tabular}{|c|c|c|c|c|}
\hline
\rowcolor[HTML]{EFEFEF} 
\hline
Cluster             & Num\_Elem              & Atributo     & Erro & Erro (\%) \\ \hline
 &  & Ba & 0 & 0\\
 &  & K & 0 & 0\\
 &  & Si & 2 & 2,71\\
\multirow{-2}{*}{1} &   \multirow{-2}{*}{74}& Na & 3 & 4,06\\ \hline
  &  & Fe & 0 & 0\\
 \multirow{-2}{*}{2} &   \multirow{-2}{*}{5} & Ca & 0 & 0\\ \hline
  &  & K & 0 & 0\\
 \multirow{-2}{*}{3} &   \multirow{-2}{*}{19} & Ba & 1 & 5,27 \\ \hline
  &   & K & 0 & 0\\
 &  &  Ba & 1 & 3,13\\
\multirow{-3}{*}{4} &   \multirow{-3}{*}{32} & Ca & 1 & 3,13\\ \hline
 &  & Ba & 0 & 0\\
 &  & K & 0 & 0\\
 &  &  Na & 2 & 3,58\\
 &  & Al & 4 & 7,15\\
\multirow{-5}{*}{5} &   \multirow{-5}{*}{56}  & Mg & 6 & 10,72\\ \hline
  &  & Fe & 0 & 0\\
 \multirow{-2}{*}{6} &   \multirow{-2}{*}{28} & K & 1 & 3,58 \\ \hline
\multicolumn{3}{|c|}{Total}                             & 21   &\\ \hline
\end{tabular}
} %fim scalebox
} % fim subfloat
\end{table}

%\lipsum[50] % Texto qualquer. REMOVER!!

% ----------------------------------------------------------
\chapter{Características da Implementação}
% ----------------------------------------------------------
\label{apendice:2}
Utilizando como referência o  trabalho de  \cite{Lopes2016} foi utilizado como ferramenta de desenvolvimento o MATLAB\footnote{http://www.mathworks.com/products/matlab/ ; versão: R2016a(9.0.0.341360); 64-bit (glnxa64)}, uma poderosa ferramenta matemática e IDE de desenvolvimento com recursos de aprendizado de máquina em pacotes chamados de \textit{Statistics and Machine Learning Toolbox}. De acordo com a documentação do MATLAB\footnote{https://la.mathworks.com/help/stats/supervised-learning-machine-learning-workflow-and-algorithms.html?lang=en} a  tabela \ref{tab:matlab} exibe quais algoritmos são implementados pela Toolbox.



\begin{table}[!ht]
\centering
\caption{Informações retiradas da documentação do MATLAB v.2016a - Supervised Learning Workflow and Algorithms}
\label{tab:matlab}
\scalebox{0.6}{
\begin{tabular}{p{3cm}|p{4cm}|p{3cm}|p{5cm}|p{5cm}|p{4.5cm}}%p{65}|p{70}|p{100}|p{100}|p{100}|p{100}
\hline
\rowcolor[HTML]{EFEFEF}
Classificador & Suporta multi-classes & Suporta Preditor Categórico & Velocidade de Predição & Utilização de Memória & Interpretabilidade\\ \hline
Decision trees — fitctree & Yes & Yes & Fast & Small & Easy\\ \hline
Discriminant analysis — fitcdiscr & Yes & No & Fast & Small for linear, large for quadratic & Easy\\ \hline
SVM — fitcsvm & No. Combine multiple binary SVM classifiers using fitcecoc. & Yes & Medium for linear. Slow for others. & Medium for linear. All others: medium for multiclass, large for binary. & Easy for linear SVM. Hard for all other kernel types.\\ \hline
Naive Bayes — fitcnb & Yes & Yes & Medium for simple distributions. Slow for kernel distributions or high-dimensional data & Small for simple distributions. Medium for kernel distributions or high-dimensional data & Easy\\ \hline
Nearest neighbor — fitcknn & Yes & Yes & Slow for cubic. Medium for others. & Medium & Hard\\ \hline
Ensembles — fitensemble & Yes & Yes & Fast to medium depending on choice of algorithm & Low to high depending on choice of algorithm. & Hard\\ \hline
\end{tabular}}
\end{table}

Esta tabela exibe algoritmos de aprendizado  supervisionados e suas característica através de resultados  estudados de bases de dados com mais de 7000 observações e 50 classes.

Uma vez decidido pela utilização da \textit{toolbox} e suas característica expostas na tabela \ref{tab:matlab}, foi escolhida para este trabalho um algoritmo por paradigma; simbólico (árvore de decisão com CART), bayesiano (probabilístico Nave Bayes) e analogia ou baseado em instância (KNN). Alguns dos algoritmos mostrado na tabela não foram  utilizados, como:  Discriminant analysis pois é do mesmo tipo bayesiano e Ensembles funciona com várias outras implementações de regressões e classificações com pequenas nuances, e também não foi utilizado por sair do foco desta pesquisa.



\end{apendicesenv}
% ---
