% ---
% RESUMOS
% ---

% RESUMO em português
\setlength{\absparsep}{18pt} % ajusta o espaçamento dos parágrafos do resumo
\begin{resumo}
Frente ao aumento do tráfego de dados em consequência de novas tecnologias, como também a necessidade de mais equipamentos  conectados à rede pedindo por processamento de dados, cada vez mais algoritmos de aprendizado de máquina não-supervisionados estão sendo estudados para obterem bons resultados, na criação de grupos (cluster), em face de obteção de informações úteis desses grupos. A partir desse problema de agrupamento, em grandes volumes de dados, tem-se um grau de dificuldade diretamente proporcional ao crescimeto desse volume. É nesse tema  onde este trabalho atua, muito embora a importância desta proposta de mestrado esteja na interpretação, no entendimento dos grupos e não na criação dos mesmos. Diante o entendimento desses grupos esta pesquisa realiza de forma empírica, ou seja, através de experimentos e testes, a identificação de atributos mais significativos no grupo, junto com faixa de valores que mais se repete a ponto de representá-lo (rotulação).  Dessa forma para a realização da rotulação de grupos de dados a proposta desta pesquisa é utilizar dois algoritmos supervisionados, cada um, com paradigmas diferentes: Naive Bayes (estatístico) e CART (simbólico). E a partir dos testes demonstrar que a rotulação é capaz de representar o grupo  possuindo uma acurácia acima de 70\% de acerto dos valores representados pelo rótulo escolhido.

% Segundo a ABNT, o resumo deve ressaltar o  objetivo, o método, os resultados e as conclusões do documento. A ordem e a extensão  destes itens dependem do tipo de resumo (informativo ou indicativo) e do  tratamento que cada item recebe no documento original. O resumo deve ser  precedido da referência do documento, com exceção do resumo inserido no  próprio documento. (\ldots) As palavras-chave devem figurar logo abaixo do  resumo, antecedidas da expressão Palavras-chave:, separadas entre si por  ponto e finalizadas também por ponto.

  \textbf{Palavras-chaves}: cluster. rotulação. aprendizado supervisionado. 
\end{resumo}

% ABSTRACT in english
\begin{resumo}[Abstract]
 \begin{otherlanguage*}{english}
   This is the english abstract.

   \vspace{\onelineskip}
 
   \noindent 
   \textbf{Keywords}:  cluster. rotulação.
 \end{otherlanguage*}
\end{resumo}
