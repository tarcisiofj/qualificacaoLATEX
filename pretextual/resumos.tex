% ---
% RESUMOS
% ---

% RESUMO em português
\setlength{\absparsep}{18pt} % ajusta o espaçamento dos parágrafos do resumo
\begin{resumo}
Frente ao grande volume e fluxo de dados, algoritmos de aprendizado de máquina são explorados para obterem bons resultados na criação de grupos (cluster) de dados. É nesse contexto onde este trabalho atua, muito embora a importância desta proposta de mestrado esteja na interpretação dos grupos e não na criação dos mesmos. Esta pesquisa realiza de forma empírica, ou seja, através de experimentos e testes a identificação de atributos significativos no grupo a ponto de representá-lo (rotulação). Para isso, utiliza-se dois algoritmos supervisionados, cada um, com paradigmas diferentes: Naive Bayes (estatístico) e CART (simbólico). Através da técnica de correlação de atributos, onde existe uma relação dos dados de entrada com os dados de saída(classe), são testados os dois algoritmos supervisionados em diferentes bases de dados. As respostas encontradas nestes testes serão atributos, junto com suas faixas de valores, representando o grupo e provando que é possível fazer rotulação com Naive Bayes e CART.

% Segundo a ABNT, o resumo deve ressaltar o  objetivo, o método, os resultados e as conclusões do documento. A ordem e a extensão  destes itens dependem do tipo de resumo (informativo ou indicativo) e do  tratamento que cada item recebe no documento original. O resumo deve ser  precedido da referência do documento, com exceção do resumo inserido no  próprio documento. (\ldots) As palavras-chave devem figurar logo abaixo do  resumo, antecedidas da expressão Palavras-chave:, separadas entre si por  ponto e finalizadas também por ponto.

  \textbf{Palavras-chaves}: cluster. rotulação.
\end{resumo}

% ABSTRACT in english
\begin{resumo}[Abstract]
 \begin{otherlanguage*}{english}
   This is the english abstract.

   \vspace{\onelineskip}
 
   \noindent 
   \textbf{Keywords}:  cluster. rotulação.
 \end{otherlanguage*}
\end{resumo}
