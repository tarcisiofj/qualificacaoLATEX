% ---
% RESUMOS
% ---

% RESUMO em português
\setlength{\absparsep}{18pt} % ajusta o espaçamento dos parágrafos do resumo
\begin{resumo}
%Frente ao aumento do tráfego de dados em consequência de novas tecnologias, como também a necessidade de mais equipamentos  conectados à rede passíveis de processamento de dados, cada vez mais algoritmos de aprendizado de máquina estão sendo estudados para  extraírem dados relevantes de grandes volumes de dados. A partir desse problema de interpretação, em grandes volumes de dados, tem-se um grau de dificuldade diretamente proporcional ao crescimeto desse volume. É nesse tema  onde este trabalho atua, no entendimento dos grupos que são formados e não na criação dos mesmos. Diante o entendimento desses grupos esta pesquisa realiza de forma empírica, ou seja, através de experimentos e testes, a identificação de atributos mais significativos no grupo, junto com faixa de valores que mais se repetem a ponto de representá-lo (rotulação).  Dessa forma para a realização da rotulação de grupos de dados a proposta desta pesquisa é utilizar dois algoritmos supervisionados, cada um, com paradigmas diferentes: Naive Bayes (estatístico) e CART (simbólico). E a partir dos testes demonstrar que a rotulação é capaz de representar o grupo. Nos resultados obtemos uma acurácia acima de 70\% de acerto dos valores representados pelo rótulo escolhido.

Com o avanço da tecnologia cada vez mais equipamentos estão se conectando nas redes gerando fluxos e processamento de dados, com isso, mais algoritmos de aprendizado de máquina estão sendo estudados para extraírem informações relevantes desses grandes volumes. Com o grande aumento desse fluxo de dados a interpretação dos mesmos podem ser prejudicada sendo o grau de dificuldade proporcional a esse crescimento. É nesse contexto que essa pesquisa atua, pois alguns algoritmos de aprendizado de máquina criam grupos de dados que possuem algumas característica, e nesse trabalho foi realizado uma pesquisa científica com objetivo de identificar nesses grupos quais são os atributos mais significativos junto com os valores que mais se repetem a ponto de representar o grupo, chamando essa técnica de rotulação. Dessa forma, esta pesquisa utiliza nessa técnica algoritmos supervisionados, já implementados por um software de cálculo numérico (MATLAB), onde pretende-se rotular grupos já criados em diferentes bases de dados exibindo um resultado em porcentagem de acordo com o número de registros que são representados pelo rótulo criado.

% Segundo a ABNT, o resumo deve ressaltar o  objetivo, o método, os resultados e as conclusões do documento. A ordem e a extensão  destes itens dependem do tipo de resumo (informativo ou indicativo) e do  tratamento que cada item recebe no documento original. O resumo deve ser  precedido da referência do documento, com exceção do resumo inserido no  próprio documento. (\ldots) As palavras-chave devem figurar logo abaixo do  resumo, antecedidas da expressão Palavras-chave:, separadas entre si por  ponto e finalizadas também por ponto.

  \textbf{Palavras-chaves}: grupos. rotulação. aprendizado supervisionado. 
\end{resumo}

% ABSTRACT in english
\begin{resumo}[Abstract]
 \begin{otherlanguage*}{english}
   %In the face of increasing data traffic as a result of new technologies, as well as the need for more data-connected equipment, more and more machine learning algorithms are being studied to extract relevant data from large volumes of data. From this problem of interpretation, in large volumes of data, there is a degree of difficulty directly proportional to the growth of this volume. It is in this theme where this work acts, in the understanding of the groups that are formed and not in the creation of the same ones. In the understanding of these groups, this research performs empirically, that is, through experiments and tests, the identification of more significant attributes in the group, along with a range of values ​​that are repeated the most to represent it (labeling). In this way, the proposal of this research is to use two supervised algorithms, each with different paradigms: Naive Bayes (statistical) and CART (symbolic). And from the tests demonstrate that the labeling is able to represent the group. In the results we obtain an accuracy of more than 70 \% of correctness of the values ​​represented by the chosen label.

   
   
   \vspace{\onelineskip}
 
   \noindent 
   \textbf{Keywords}:  cluster. rotulação.
 \end{otherlanguage*}
\end{resumo}
