% ---
% RESUMOS
% ---

% RESUMO em português
\setlength{\absparsep}{18pt} % ajusta o espaçamento dos parágrafos do resumo
\begin{resumo}
%Frente ao aumento do tráfego de dados em consequência de novas tecnologias, como também a necessidade de mais equipamentos  conectados à rede passíveis de processamento de dados, cada vez mais algoritmos de aprendizado de máquina estão sendo estudados para  extraírem dados relevantes de grandes volumes de dados. A partir desse problema de interpretação, em grandes volumes de dados, tem-se um grau de dificuldade diretamente proporcional ao crescimeto desse volume. É nesse tema  onde este trabalho atua, no entendimento dos grupos que são formados e não na criação dos mesmos. Diante o entendimento desses grupos esta pesquisa realiza de forma empírica, ou seja, através de experimentos e testes, a identificação de atributos mais significativos no grupo, junto com faixa de valores que mais se repetem a ponto de representá-lo (rotulação).  Dessa forma para a realização da rotulação de grupos de dados a proposta desta pesquisa é utilizar dois algoritmos supervisionados, cada um, com paradigmas diferentes: Naive Bayes (estatístico) e CART (simbólico). E a partir dos testes demonstrar que a rotulação é capaz de representar o grupo. Nos resultados obtemos uma acurácia acima de 70\% de acerto dos valores representados pelo rótulo escolhido.

%Com o avanço da tecnologia cada vez mais equipamentos estão se conectando nas redes gerando fluxos e processamento de dados, com isso, mais algoritmos de aprendizado de máquina estão sendo estudados para extraírem informações relevantes desses grandes volumes. Com o grande aumento desse fluxo de dados a interpretação dos mesmos podem ser prejudicada sendo o grau de dificuldade proporcional a esse crescimento. É nesse contexto que essa pesquisa atua, pois alguns algoritmos de aprendizado de máquina criam grupos de dados que possuem algumas característica, e nesse trabalho foi realizado uma pesquisa científica com objetivo de identificar nesses grupos quais são os atributos mais significativos junto com os valores que mais se repetem a ponto de representar o grupo, chamando essa técnica de rotulação. Dessa forma, esta pesquisa utiliza nessa técnica algoritmos supervisionados, já implementados por um software de cálculo numérico (MATLAB), onde pretende-se rotular grupos já criados em diferentes bases de dados exibindo um resultado em porcentagem de acordo com o número de registros que são representados pelo rótulo criado.

Com o avanço da tecnologia, cada vez mais equipamentos estão se conectando nas redes, gerando fluxos e processamento de dados. Com isso, mais algoritmos de aprendizado de máquina estão sendo estudados para extraírem informações relevantes desses grandes volumes. Com o grande aumento desse fluxo de dados, a interpretação destes pode ser prejudicada, sendo o grau de dificuldade proporcional a esse crescimento. É nesse contexto que essa pesquisa atua utilizando algoritmos de aprendizado de máquina supervisionados, os quais são algoritmos capazes de aprender através de determinados exemplos ou comportamentos. Neste trabalho realizou-se uma pesquisa científica com o objetivo de identificar em grupos de dados quais são os atributos mais significativos junto aos valores que mais se repetem a ponto de representá-lo, denominando-se essa técnica de rotulação. Dessa forma, utilizou-se técnica de algoritmos supervisionados, que através dos dados de entrada fazem uma correlação com uma saída desejável, e mediante isso, essa técnica é aplicada em todos os atributos para encontrar o mais significativo no cluster. Em seguida, a partir desse atributo mais significativo, utiliza-se um intervalo de dados que possui  maior incidência de valores  compondo o rótulo (atributo/faixa de valor). Nas bases testadas, somente uma dentre as quatro, obteviveram acurácias em alguns \textit{clusters} abaixo de 70\%, mas em todas outras os rótulos tiveram acurácias acima desse valor, indicando que é possível identificar os grupos através dos rótulos encontrados.



%Mediante isso os cada valor de atributo de um \textit{cluster} será saída desejável entre os outros atributos até todos os atributos tenhão seus valores de correlacionamento. Uma vez encontrado o maior valor, este será escolhido como o atributo rótulo,  
%já implementados por um software de cálculo numérico (MATLAB), em que se pretende rotular grupos já criados em diferentes bases de dados, exibindo um resultado em porcentagem de acordo com o número de registros que são representados pelo rótulo criado. 

% Segundo a ABNT, o resumo deve ressaltar o  objetivo, o método, os resultados e as conclusões do documento. A ordem e a extensão  destes itens dependem do tipo de resumo (informativo ou indicativo) e do  tratamento que cada item recebe no documento original. O resumo deve ser  precedido da referência do documento, com exceção do resumo inserido no  próprio documento. (\ldots) As palavras-chave devem figurar logo abaixo do  resumo, antecedidas da expressão Palavras-chave:, separadas entre si por  ponto e finalizadas também por ponto.

  \textbf{Palavras-chaves}: grupos. rotulação. aprendizado supervisionado. 
\end{resumo}

% ABSTRACT in english
\begin{resumo}[Abstract]
 \begin{otherlanguage*}{english}
 With the advancement of technology, more and more equipment is connected in the networks, generating flows and data processing. As a result, more algorithms from other languages ​​are being studied to see the ones for which they are paid. With a great degree of data flow, an interpretation can be impaired, being a degree of difficulty proportional to that growth. This context that this study at the time using the apprentized behaviors of the supervised machine, which is algorithm in aquatic algorithm for behavior samples or behaviors. In this work a scientific investigation was carried out with the objective of identifying the data groups that are more characteristic at the level of the same ones that repeat a point of representation, denominating itself a technique of labeling. In this way, we use the controlled supervision technique, with the entry of the input data of one with the desired output, and through this, the same technique is applied in all the attributes to find the most significant in the cluster. Then use the most meaningful data, use a range of data that has the largest capacity or label. In the tested databases, only one among the four, we obtain accuracy in some clusters below 70\%, but in all other keywords the acurances on this value are being identified in the groups through the labels found.
   
   
   \vspace{\onelineskip}
 
   \noindent 
   \textbf{Keywords}:  cluster. Machine Learning. Supervised Learning. Classification. Labeling.
 \end{otherlanguage*}
\end{resumo}
